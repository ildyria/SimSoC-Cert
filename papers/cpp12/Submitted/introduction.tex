\section{Introduction}
\label{sec:intro}

Type-theoretic settings such as Coq \cite{CoqManualV83,BC04,cpdt}
offer two elementary ways of constructing new objects:
functions and inductive types\footnote{%
Co-inductive types are available as well. 
However, this paper does not depend on issues related to finiteness
of computations:
what is said about inductive types holds as well for co-inductive types.
}. 
%\todo{Inductive are used for datatypes and relations, fixpoints for functions.}
%
For instance, even natural numbers can be inductively characterized 
by the following two rules:

\[
\begin{prooftree}
\using {\coqdocvar{E0}}
\justifies\coqdocvar{even\_i}~ 0
\end{prooftree}
\qquad
\begin{prooftree}
\coqdocvar{even\_i}~ n
\using {\coqdocvar{E2}}
\justifies\coqdocvar{even\_i (S (S n))}
\end{prooftree}
\]


%\coqdocinput{chunk1}

\noindent
Rule names such as \coqdocvar{E1} and \coqdocvar{E2}
serve as canonical justifications for \coqdocvar{even\_i}, 
they are called the \emph{constructors} of the inductive definition.

Now, assume a hypothesis $H$ claiming
that \coqdocvar{even\_i (S (S (S x)))} for some natural number $x$.
Then, by looking at the definition of \coqdocvar{even\_i}, 
we see that only \coqdocvar{E2} could justify $H$,
and we can conclude that \coqdocvar{even\_i (S x)}.
Similarly,  \coqdocvar{even\_i} 1 can be considered as an
absurd hypothesis, since, as \coqdocvar{(S 0)} matches neither
0 nor \coqdocvar{(S (S n))}, 
none of the two possible canonical ways of proving \coqdocvar{even\_i},
namely \coqdocvar{E0} and \coqdocvar{E2} can be used.
Such proof steps are called \emph{inversions},
because they use justifications such as \coqdocvar{E0} and \coqdocvar{E2}
in the opposite way, i.e.,
from their conclusion to their premises. 
Note that \coqdocvar{even\_i} 3, \coqdocvar{even\_i} 5, etc. 
do not immediately yield the absurd by inversion.
However, by iterating the first inversion step, we eventually get
\coqdocvar{even\_i} 1 and then the desired result using a last inversion.
This illustrates that inversion is closer to case analysis than to induction.

Indeed, as we will see below, 
inversion can be decomposed in elementary proof steps,
where the key step is a primitive case analysis on the considered
inductive object (the hypothesis $H$, in our previous example). 
However, this decomposition is very often far from trivial because,
in the general case, rules include several premises,
premises and conclusions may have several arguments and
some of these arguments can be shared.
Still, inversion turns out to be extremely useful in practice.
Well-known instances are related to programming languages,
because their semantics is described using complex inductively defined
relations. 

Note that, it may be worth considering a (recursive) \emph{function}
for defining a predicate, rather than an inductive relation.
For instance, in the Coq syntax, an alternative for even
numbers is as follows:

% Now, assume a goal containing a hypothesis $H$ claiming
% that \coqdocvar{even\_i (S x)} for some natural number $x$.
% Then, by looking at the definition of \coqdocvar{even\_i}, 
% we can conclude that $x$ is \coqdocvar{(S y)}
% for some $y$ satisfying \coqdocvar{even\_i y}.

\medskip
\coqdocinput{chunk11}
\medskip

\noindent
Here \coqdocvar{True} denotes a trivially provable proposition,
and \coqdocvar{False} denotes the absurd proposition.
%
Using \coqdocvar{even\_f} is much simpler in the previous situations:
for instance, \coqdocvar{even\_f (S (S (S x)))} just \emph{reduces} to
\coqdocvar{even\_f (S x)} using computation.
In other words, computation provides inversion for free.
Therefore, one may wonder why we should bother with inductively defined
relations.
Two kinds of answers can be given.

One of them is that an inductive definition allows us 
to focus exactly on the relevant values
whereas, with functional definitions, 
we have to deal with the full domain, 
which can be much bigger in general.
In our example above, 
suppose that we want to prove a statement such as
$\forall n, \mathit{even}\:n \impl P\: n$.
We can always attempt an induction on $n$, 
but this strategy enforces to reason on all numbers, 
including odd numbers.
If $\mathit{even}$ is encoded with \coqdocvar{even\_f},
this is no other option.
However, using \coqdocvar{even\_i}, 
we have the additional opportunity to make an induction on 
(the shape of) $\coqdocvar{even\_i}\;n$,
without needing to bother about odd numbers.

Another issue is that it is not always convenient or even possible to
provide a functional definition of a predicate.
Whenever possible,
a $n$-ary relation $R$ on $A_1 \times \ldots A_n$, % with $n \ge 2$,
is advantageously modeled by a function from $A_1, \ldots A_{n-1}$ to $A_n$.
But it requires $R$ to be functional (deterministic) and moreover,
in type-theoretical settings such as CIC, to be total.
If the relation non-deterministic,
we still can try to 
define it by a function returning either \coqdocvar{True}
or \coqdocvar{False}, as is the case for \coqdocvar{even\_f};
this essentially amounts to provide a decision procedure for 
the intended predicate\footnote{
Note that a 1-ary relation $P$ on $A_1$ is isomorphic to a 
binary relation on $\mathbf{1}\times A_1$,
where $\mathbf{1}$ is a type with exactly one inhabitant.
If $P$ holds for at least two values on $A_1$, 
it can be clearly considered as a non-deterministic 
function from $\mathbf{1}$ to $A_1$.
}.
This is not always possible and, even if we can find such an
algorithm, it may be hindered by undesired encoding tricks,
which will induce additional complications in proofs. 
Moreover, a requirement of formal methods expresses that
high-level definitions and statements should be as clear 
as possible in order to be convincing. 
The inductive style is not always the better than the functional
style, but it is often enough the case so that we cannot
ignore it. 
For technical reasons, it is sometimes worth to consider a 
a functional version and an inductive version of the same notion.
Even if the functional version is much better at inversion-like
proof steps, 
the two versions have to be proved equivalent and there,
the need for inverting the inductive version almost inevitably shows up.


All these considerations are especially relevant in the case
of the operational semantics of programming languages,
either in small-step or in big-step style \cite{nielson}. 
Such semantics define transitions between states,
language constructs and,
very often, additional arguments such as input/output events. 
They are inductively defined, 
with at least one rule for each language construct. 
A tutorial example of a toy (but Turing-complete) language 
formally defined in Coq along these lines
is given in \cite{Pierce:SF}
and routinely used as a teaching support in many universities.
A much more involved example
is the semantics of a fairly large subset of C, as defined in 
the Compcert project \cite{Leroy-Compcert-CACM}.

In the SimSoC-cert project \cite{cpp11}, 
we use this semantics to perform proofs of 
an instruction set simulator for ARM,
which is at the heart of SimSoC~\cite{rapido11}, 
a simulator of embedded systems written in C and C++.
Many inversions are needed in our proofs.

\medskip
The practical need for automating inversion has been identified
many years ago.
The first implementations for Coq and LEGO
are analyzed and explained in
\cite{cornes95automating} for Coq
and \cite{McBride96} for LEGO.
Since then, the main tool proposed to the Coq user is
a tactic called \inversion which,
basically performs a case analysis over a given hypothesis
according to its specific specific arguments,
removes absurd cases,
introduces relevant premises in the environment
and performs suitable substitutions in the whole goal.
%
This tactic works remarkably well,
though it fails in seldom intricate cases,
as reported in mailing lists. 
%
However, the price to pay for its generality
is a high complexity of the formal proof-term underlying
an inversion. 
Does it reflect an unnecessarily complex formalization of a 
(at first sight) rather simple idea?
Anyway, 
beyond slowing down the evaluation of scripts which make
an intensive use of this tactic, 
a practical consequence is that
unpleasantly heavy proof terms can unexpectedly occur in
functions defined in interactive mode.

More importantly, in our opinion, using this tactic
introduces many new hypotheses in the environment.
Their names are automatically generated
and sequel of the script depends on them.
Moreover such introduced hypotheses could be inverted again,
and so on.
This poses a problem of robustness which is very serious
in large developments:
updating the inductive relation or
even minor modifications in another part of the development
may result in a complete renaming 
inside a proof script,
which has then to be debugged line by line.
The situation is better in recent version of Coq, 
since \inversion can optionally be given the names of all hypotheses
to be introduced.
Still, their number and contents is hard to predict,
which makes \inversion hardly usable in high-level tactics.

In \cite{small_inv}, 
the first author introduced a technique 
for performing
so-called \emph{small inversions}. 
This technique is rather flexible and is available in several variants.
Our goal was to demystify the magics behind \inversion
and to propose a practical hand-crafted alternative
to this tactic, 
providing much smaller proof terms as well as
a full control of the user on the behavior of inversion.
The idea was illustrated only on very simple examples
and had to be validated on realistic applications. 

We report here such an experiment, 
in the framework of the SimSoC-cert project introduced above.
It turned out that significant changes had to
be made in order to make the initial idea able
to scale up.
%
The contributions presented here are then:
\begin{itemize}
\item an improvement of the main variant from \cite{small_inv},
  which makes 
  it is both simpler to use and more powerful;
\item its illustration on a significant application,
  which involves an intensive use of inversions on 
  big inductive relations coming from the Compcert project.
\end{itemize}

The concrete setting considered here is the Coq proof assistant,
but the technique can be adapted to any proof assistant based
on the Calculus of Inductive constructions or a similar type theory, 
such as LEGO, Matita or Agda.
The rest of the paper is organized as follows.
Section~\ref{sec:absurd}
recalls the principle of small inversions as introduced in \cite{small_inv}.
Section~\ref{sec:improvement} then explains its limitations
and how to overcome them,
while section~\ref{sec:simsoccert} contains a summary
of the application to SimSoC-cert.
We conclude in section~\ref{sec:conclusion} with a comment
on our achievements and some perspectives.




%%% Local Variables: 
%%% mode: latex
%%% TeX-master: "cpp12"
%%% End: 
