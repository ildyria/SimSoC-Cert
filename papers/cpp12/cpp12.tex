%\documentclass[runningheads]{llncs}
\documentclass{llncs}
%\usepackage{graphicx}
\usepackage{xcolor}
\usepackage{amssymb}
\usepackage{dsfont}
\usepackage{amsmath} 

\usepackage{url}
\usepackage{cancel}
\usepackage{verbatim}
\usepackage{alltt}
\usepackage{graphicx}
\usepackage{prooftree}

\usepackage{latexsym}
\usepackage{xspace}
%\usepackage{draft} REVOIR
\usepackage{comment}
%\usepackage{ntheorem}
%\usepackage{amsthm} % \proof seems to exist in llncs

\newcommand{\revoir}[1]{\mbox{{[*** {\bfseries\large #1} ***]}}}
\newcommand{\nouveau}[1]{\textcolor{blue}{#1}}

\newcommand{\egdef}%
   {\ensuremath{\:\mathrel{\raisebox{-.7ex}%
   {$\stackrel{\rm def}{=\mkern-8mu=}$}}\:}}

\newcommand{\simlight}{\texttt{simlight}\xspace}

\newtheorem{Prog}{Program}
%\newcommand{\idcoq}[1]{\texttt{#1}}
\newcommand{\idcoq}[1]{\ensuremath{\mathit{#1}}}
\newcommand{\impl}{\ensuremath{\Rightarrow}}
\newcommand{\flequiv}{\ensuremath{\leftrightarrow}}
\newcommand{\todo}[1]{\textcolor{blue}{\textsc{[#1]}}}
\newcommand{\XM}[1]{{\color{red} #1}} % Xiaomu's editing
\newcommand{\JFM}[1]{{\color{red} #1}} % JF's editing

% BEGIN For coqdoc 

\newenvironment{coqdoccode}{}{}

\newlength{\coqdocbaseindent}
\setlength{\coqdocbaseindent}{0em}

\newcommand{\coqdef}[3]{#3}
\newcommand{\coqref}[2]{#2}
\newcommand{\coqexternalref}[3]{#3}

% Beginning of a line without any Coq indentation
\newcommand{\coqdocnoindent}{\noindent\kern\coqdocbaseindent}
% Beginning of a line with a given Coq indentation
\newcommand{\coqdocindent}[1]{\noindent\kern\coqdocbaseindent\noindent\kern#1}
% End-of-the-line
\newcommand{\coqdoceol}{\hspace*{\fill}\setlength\parskip{0pt}\par}
% Empty lines (in code only)
\newcommand{\coqdocemptyline}{\vskip 0.4em plus 0.1em minus 0.1em}

% own name
\newcommand{\coqdoc}{\textsf{coqdoc}}

% pretty underscores (the package fontenc causes ugly underscores)
%BEGIN LATEX
\def\_{\kern.08em\vbox{\hrule width.35em height.6pt}\kern.08em}
%END LATEX

% macro for typesetting keywords
\newcommand{\coqdockw}[1]{\texttt{#1}}

% macro for typesetting variable identifiers
\newcommand{\coqdocvar}[1]{\textit{#1}}

% macro for typesetting constant identifiers
\newcommand{\coqdoccst}[1]{\textsf{#1}}

% macro for typesetting module identifiers
\newcommand{\coqdocmod}[1]{\textsc{\textsf{#1}}}

% macro for typesetting module constant identifiers (e.g. Parameters in
% module types)
\newcommand{\coqdocax}[1]{\textsl{\textsf{#1}}}

% macro for typesetting inductive type identifiers
\newcommand{\coqdocind}[1]{\textbf{\textsf{#1}}}

% macro for typesetting constructor identifiers
\newcommand{\coqdocconstr}[1]{\textsf{#1}}

% macro for typesetting tactic identifiers
\newcommand{\coqdoctac}[1]{\texttt{#1}}

% These are the real macros used by coqdoc, their typesetting is 
% based on the above macros by default.

\newcommand{\coqdoclibrary}[1]{\coqdoccst{#1}}
\newcommand{\coqdocinductive}[1]{\coqdocind{#1}}
\newcommand{\coqdocdefinition}[1]{\coqdoccst{#1}}
\newcommand{\coqdocvariable}[1]{\coqdocvar{#1}}
\newcommand{\coqdocconstructor}[1]{\coqdocconstr{#1}}
\newcommand{\coqdoclemma}[1]{\coqdoccst{#1}}
\newcommand{\coqdocclass}[1]{\coqdocind{#1}}
\newcommand{\coqdocinstance}[1]{\coqdoccst{#1}}
\newcommand{\coqdocmethod}[1]{\coqdoccst{#1}}
\newcommand{\coqdocabbreviation}[1]{\coqdoccst{#1}}
\newcommand{\coqdocrecord}[1]{\coqdocind{#1}}
\newcommand{\coqdocprojection}[1]{\coqdoccst{#1}}
\newcommand{\coqdocnotation}[1]{\coqdockw{#1}}
\newcommand{\coqdocsection}[1]{\coqdoccst{#1}}
\newcommand{\coqdocaxiom}[1]{\coqdocax{#1}}
\newcommand{\coqdocmodule}[1]{\coqdocmod{#1}}

% END For coqdoc 

% macros for this paper
\newcommand{\diag}{\coqdocvar{diag}\xspace}
\newcommand{\inversion}{\coqdoctac{inversion}\xspace}
\newcommand{\inv}{\coqdoctac{inv}\xspace}
\newcommand{\casetac}{\coqdoctac{case}\xspace}
\newcommand{\match}{\coqdoctac{match}\xspace}
\newcommand{\with}{\coqdoctac{with}\xspace}
\newcommand{\intac}{\coqdoctac{in}\xspace}
\newcommand{\retac}{\coqdoctac{return}\xspace}
\newcommand{\refine}{\coqdoctac{refine}\xspace}
\newcommand{\Prop}{\coqdockw{Prop}\xspace}
\newcommand{\True}{\coqdocvar{True}\xspace}
\newcommand{\eveni}{\coqdocvar{even\_i}\xspace}
\newcommand{\EZ}{\coqdocvar{E0}\xspace}
\newcommand{\ET}{\coqdocvar{E2}\xspace}
\newcommand{\evenf}{\coqdocvar{even\_f}\xspace}
\newcommand{\prone}{\coqdocvar{pr\_1}\xspace}
\newcommand{\prET}{\coqdocvar{premises\_E2}\xspace}
\newcommand{\EXPR}{\coqdocvar{EXPR}\xspace}
\newcommand{\PROOFEV}{\coqdocvar{PROOF\_EV}\xspace}
\renewcommand{\PROOFEV}{\mathit{PE}}

% For automated inclusion of .tex files generated from coqdoc
\newcommand{\coqdocinput}[1]{\input{#1}}

\newcommand{\union}{\ensuremath{\cup}}

\newenvironment{ml}
  {\begin{alltt}
   \footnotesize} %% 3.12.0
  {\end{alltt}
  }

\newenvironment{coq}
  {\begin{alltt}
   \footnotesize} %% 8.3pl2 (April 2011)
  {\end{alltt}
  }

\newenvironment{humC}
  {\begin{alltt}
   \footnotesize}
  {\end{alltt}
  }



\begin{document}

\title{Scaling up Small Inversions
}
\titlerunning{Scaling up Small Inversions}
% Official order as registered at CPP
\author{Jean-Fran\c{c}ois Monin\inst{1,2}
\and 
Xiaomu Shi\inst{1}
}
\institute{%
Universit\'{e} de Grenoble 1 - LIAMA
\and CNRS - LIAMA
}
\authorrunning{J.-F. Monin, X. Shi}

\maketitle

\begin{abstract}
  When reasoning on formulas involving large-size inductively defined
  relations, such as the semantics of a real programming language,
  many steps require the inversion of a hypothesis. The built-in
  ``inversion'' tactic of Coq can then be used, but it suffers from
  severe controllability and efficiency issues.  A proof-trick called
  small inversions by one of the co-authors provides a part of the
  solution. It is based on a suitable auxiliary diagonal
  predicate. However, many important practical situations are not
  covered by this technique.  We present here an improvement inspired
  by the impredicative encoding of inductive data-structures. Our
  experiments in the SimSoC-Cert project show that this technique
  successfully scales up to proofs of non-trivial programs according
  to the operational semantics of C as defined in Compcert.
\end{abstract}



%-------------------------------------------------------------------------
\section{Introduction}
\label{sec:intro}

Type-theoretic settings such as Coq \cite{CoqManualV83,BC04,cpdt}
offer two elementary ways of constructing new objects:
functions and inductive types\footnote{%
Co-inductive types are available as well. 
However, this paper does not depend on issues related to finiteness
of computations:
what is said about inductive types holds as well for co-inductive types.
}. 
%\todo{Inductive are used for datatypes and relations, fixpoints for functions.}
%
For instance, even natural numbers can be inductively characterized 
by the following two rules:

\[
\begin{prooftree}
\using {\coqdocvar{E0}}
\justifies\coqdocvar{even\_i}~ 0
\end{prooftree}
\qquad
\begin{prooftree}
\coqdocvar{even\_i}~ n
\using {\coqdocvar{E2}}
\justifies\coqdocvar{even\_i (S (S n))}
\end{prooftree}
\]


%\coqdocinput{chunk1}

\noindent
Rule names such as \coqdocvar{E1} and \coqdocvar{E2}
serve as canonical justifications for \coqdocvar{even\_i}, 
they are called the \emph{constructors} of the inductive definition.

Now, assume a hypothesis $H$ claiming
that \coqdocvar{even\_i (S (S (S x)))} for some natural number $x$.
Then, by looking at the definition of \coqdocvar{even\_i}, 
we see that only \coqdocvar{E2} could justify $H$,
and we can conclude that \coqdocvar{even\_i (S x)}.
Similarly,  \coqdocvar{even\_i} 1 can be considered as an
absurd hypothesis, since, as \coqdocvar{(S 0)} matches neither
0 nor \coqdocvar{(S (S n))}, 
none of the two possible canonical ways of proving \coqdocvar{even\_i},
namely \coqdocvar{E0} and \coqdocvar{E2} can be used.
Such proof steps are called \emph{inversions},
because they use justifications such as \coqdocvar{E0} and \coqdocvar{E2}
in the opposite way, i.e.,
from their conclusion to their premises. 
Note that \coqdocvar{even\_i} 3, \coqdocvar{even\_i} 5, etc. 
do not immediately yield the absurd by inversion.
However, by iterating the first inversion step, we eventually get
\coqdocvar{even\_i} 1 and then the desired result using a last inversion.
This illustrates that inversion is closer to case analysis than to induction.

Indeed, as we will see below, 
inversion can be decomposed in elementary proof steps,
where the key step is a primitive case analysis on the considered
inductive object (the hypothesis $H$, in our previous example). 
However, this decomposition is very often far from trivial because,
in the general case, rules include several premises,
premises and conclusions may have several arguments and
some of these arguments can be shared.
Still, inversion turns out to be extremely useful in practice.
Well-known instances are related to programming languages,
because their semantics is described using complex inductively defined
relations. 

Note that, it may be worth considering a (recursive) \emph{function}
for defining a predicate, rather than an inductive relation.
For instance, in the Coq syntax, an alternative for even
numbers is as follows:

% Now, assume a goal containing a hypothesis $H$ claiming
% that \coqdocvar{even\_i (S x)} for some natural number $x$.
% Then, by looking at the definition of \coqdocvar{even\_i}, 
% we can conclude that $x$ is \coqdocvar{(S y)}
% for some $y$ satisfying \coqdocvar{even\_i y}.

\medskip
\coqdocinput{chunk11}
\medskip

\noindent
Here \coqdocvar{True} denotes a trivially provable proposition,
and \coqdocvar{False} denotes the absurd proposition.
%
Using \coqdocvar{even\_f} is much simpler in the previous situations:
for instance, \coqdocvar{even\_f (S (S (S x)))} just \emph{reduces} to
\coqdocvar{even\_f (S x)} using computation.
In other words, computation provides inversion for free.
Therefore, one may wonder why we should bother with inductively defined
relations.
Two kinds of answers can be given.

One of them is that an inductive definition allows us 
to focus exactly on the relevant values
whereas, with functional definitions, 
we have to deal with the full domain, 
which can be much bigger in general.
In our example above, 
suppose that we want to prove a statement such as
$\forall n, \mathit{even}\:n \impl P\: n$.
We can always attempt an induction on $n$, 
but this strategy enforces to reason on all numbers, 
including odd numbers.
If $\mathit{even}$ is encoded with \coqdocvar{even\_f},
this is no other option.
However, using \coqdocvar{even\_i}, 
we have the additional opportunity to make an induction on 
(the shape of) $\coqdocvar{even\_i}\;n$,
without needing to bother about odd numbers.

Another issue is that it is not always convenient or even possible to
provide a functional definition of a predicate.
Whenever possible,
a $n$-ary relation $R$ on $A_1 \times \ldots A_n$, % with $n \ge 2$,
is advantageously modeled by a function from $A_1, \ldots A_{n-1}$ to $A_n$.
But it requires $R$ to be functional (deterministic) and moreover,
in type-theoretical settings such as CIC, to be total.
If the relation non-deterministic,
we still can try to 
define it by a function returning either \coqdocvar{True}
or \coqdocvar{False}, as is the case for \coqdocvar{even\_f};
this essentially amounts to provide a decision procedure for 
the intended predicate\footnote{
Note that a 1-ary relation $P$ on $A_1$ is isomorphic to a 
binary relation on $\mathbf{1}\times A_1$,
where $\mathbf{1}$ is a type with exactly one inhabitant.
If $P$ holds for at least two values on $A_1$, 
it can be clearly considered as a non-deterministic 
function from $\mathbf{1}$ to $A_1$.
}.
This is not always possible and, even if we can find such an
algorithm, it may be hindered by undesired encoding tricks,
which will induce additional complications in proofs. 
Moreover, a requirement of formal methods expresses that
high-level definitions and statements should be as clear 
as possible in order to be convincing. 
The inductive style is not always the better than the functional
style, but it is often enough the case so that we cannot
ignore it. 
For technical reasons, it is sometimes worth to consider a 
a functional version and an inductive version of the same notion.
Even if the functional version is much better at inversion-like
proof steps, 
the two versions have to be proved equivalent and there,
the need for inverting the inductive version almost inevitably shows up.


All these considerations are especially relevant in the case
of the operational semantics of programming languages,
either in small-step or in big-step style \cite{nielson}. 
Such semantics define transitions between states,
language constructs and,
very often, additional arguments such as input/output events. 
They are inductively defined, 
with at least one rule for each language construct. 
A tutorial example of a toy (but Turing-complete) language 
formally defined in Coq along these lines
is given in \cite{Pierce:SF}
and routinely used as a teaching support in many universities.
A much more involved example
is the semantics of a fairly large subset of C, as defined in 
the Compcert project \cite{Leroy-Compcert-CACM}.

In the SimSoC-cert project \cite{cpp11}, 
we use this semantics to perform proofs of 
an instruction set simulator for ARM,
which is at the heart of SimSoC~\cite{rapido11}, 
a simulator of embedded systems written in C and C++.
Many inversions are needed in our proofs.

\medskip
The practical need for automating inversion has been identified
many years ago.
The first implementations for Coq and LEGO
are analyzed and explained in
\cite{cornes95automating} for Coq
and \cite{McBride96} for LEGO.
Since then, the main tool proposed to the Coq user is
a tactic called \inversion which,
basically performs a case analysis over a given hypothesis
according to its specific specific arguments,
removes absurd cases,
introduces relevant premises in the environment
and performs suitable substitutions in the whole goal.
%
This tactic works remarkably well,
though it fails in seldom intricate cases,
as reported in mailing lists. 
%
However, the price to pay for its generality
is a high complexity of the formal proof-term underlying
an inversion. 
Does it reflect an unnecessarily complex formalization of a 
(at first sight) rather simple idea?
Anyway, 
beyond slowing down the evaluation of scripts which make
an intensive use of this tactic, 
a practical consequence is that
unpleasantly heavy proof terms can unexpectedly occur in
functions defined in interactive mode.

More importantly, in our opinion, using this tactic
introduces many new hypotheses in the environment.
Their names are automatically generated
and sequel of the script depends on them.
Moreover such introduced hypotheses could be inverted again,
and so on.
This poses a problem of robustness which is very serious
in large developments:
updating the inductive relation or
even minor modifications in another part of the development
may result in a complete renaming 
inside a proof script,
which has then to be debugged line by line.
The situation is better in recent version of Coq, 
since \inversion can optionally be given the names of all hypotheses
to be introduced.
Still, their number and contents is hard to predict,
which makes \inversion hardly usable in high-level tactics.

In \cite{small_inv}, 
the first author introduced a technique 
for performing
so-called \emph{small inversions}. 
This technique is rather flexible and is available in several variants.
Our goal was to demystify the magics behind \inversion
and to propose a practical hand-crafted alternative
to this tactic, 
providing much smaller proof terms as well as
a full control of the user on the behavior of inversion.
The idea was illustrated only on very simple examples
and had to be validated on realistic applications. 

We report here such an experiment, 
in the framework of the SimSoC-cert project introduced above.
It turned out that significant changes had to
be made in order to make the initial idea able
to scale up.
%
The contributions presented here are then:
\begin{itemize}
\item an improvement of the main variant from \cite{small_inv},
  which makes 
  it is both simpler to use and more powerful;
\item its illustration on a significant application,
  which involves an intensive use of inversions on 
  big inductive relations coming from the Compcert project.
\end{itemize}

The concrete setting considered here is the Coq proof assistant,
but the technique can be adapted to any proof assistant based
on the Calculus of Inductive constructions or a similar type theory, 
such as LEGO, Matita or Agda.
The rest of the paper is organized as follows.
Section~\ref{sec:absurd}
recalls the principle of small inversions as introduced in \cite{small_inv}.
Section~\ref{sec:improvement} then explains its limitations
and how to overcome them,
while section~\ref{sec:simsoccert} contains a summary
of the application to SimSoC-cert.
We conclude in section~\ref{sec:conclusion} with a comment
on our achievements and some perspectives.




%%% Local Variables: 
%%% mode: latex
%%% TeX-master: "cpp12"
%%% End: 

\section{Auxiliary diagonalization function}
\label{sec:absurd}

We illustrate here the main results of \cite{small_inv}
on the example of even numbers.
Here is the corresponding Coq inductive definition.

\medskip
\coqdocinput{chunk21}
\medskip

\noindent
We see that each rule is given by a constructor in a dependent data type
-- also called an inductive predicate or relation because its sort is \Prop.
Therefore, the elementary way to decompose an object of type \eveni $n$
is to use dependent pattern matching.
This is already done by primitive tactics of Coq
such as \coqdockw{case} and \coqdockw{destruct},
which turn out to be powerful enough in many situations, 
when a condition is satisfied:
the conclusion of the current goal fits all arguments of 
the hypothesis to be analyzed by pattern matching.

Let us first illustrate 
dependent pattern matching on even numbers.
Consider a proof $\PROOFEV$ of type $\eveni\;n$
for some natural number $n$.
For each possible constructor, \EZ or \ET, 
we provide a proof term,
respectively $t_\EZ$ and $t_\ET$.
As usual, this term may depend on the arguments 
of the corresponding constructor,
none for \EZ and, say $x$ and $ex$ for \ET.
More importantly for us, $t_\EZ$ and $t_\ET$ may have
different \emph{types}:
the type $P\;n$ of the whole expression depends on $n$;
in the first branch, the type of $t_\EZ$ is $P\;0$ and
in the second branch, the type of $t_\ET$ is $P\; (S\: (S\:x))$.
Therefore, the syntax of the \coqdockw{match} construct
contains a \coqdockw{return} clause with the expected type
of the result $P\;n$ as an argument;
moreover, 
there is also an \coqdockw{in} clause for the type of $\PROOFEV$
which binds $n$:
% moreover, in order to say where this $n$ comes from,
% there is also an \coqdockw{in} clause for the type of $\PROOFEV$:

\coqdocinput{chunk22}
\medskip

\noindent
Most of the time, Coq users do not need to go to this
level of detail: 
in interactive proof mode, 
if $n$ and $P\;n$ are clear from the context,
\casetac $\PROOFEV$ will do the job.
More precisely, if we have an hypothesis $H$ of type $\eveni\;n$
and a desired conclusion of type $P\;n$, 
\casetac $H$ will construct a proof term having the previous
shape and answer with two new subgoals:
one for $P\;0$ and one for $P\; (S\: (S\:x))$,
with $\eveni\;x$ as an additional assumption.

More work is needed precisely when there is no obvious relationship
between the conclusion and the hypothesis to be analyzed.
This happens in particular when $H$ is absurd:
the goal should be discharged whatever is its conclusion.
%This situation is covered in \cite{small_inv} as follows:
This situation is as follows:
the conclusion is converted 
to an expression \diag $V$,
where $V$ is a value coming from $H$ 
and \diag a suitable diagonal function, such that
the dependent case analysis on $H$ provides only trivial subgoals.
In \cite{small_inv}, we take the type of $H$ itself as a trivial
subgoal but here we will take \True.
For example, assume that we want to conclude
$4=7$ from the hypothesis $H: \eveni\;1$.
Our diagonal function is then defined as follows.

\coqdocinput{chunk24}

\vspace*{-.7\baselineskip}
\noindent
Then the conclusion is converted to $\diag\;1$,
and the case analysis on $H$ 
automatically provides two subgoals $\diag\;0$
and $\diag\;(S\: (S\:y))$ for an arbitrary natural number $y$.
Each of these goals reduce to \True, 
and we are done.
The proof term behind this reasoning is very short
($I$ is the standard proof of \True):

\coqdocinput{chunk25}

\vspace*{-.7\baselineskip}
Let us now consider what happens if $H$ is 
$\eveni\;3$ instead of $\eveni\;1$. 
As mentioned in the introduction, 
a first inversion on $H$ will push $\eveni\;1$ in the environment, 
and then we are back to the previous situation.
In \cite{small_inv} we show that in such situations 
the goal can be proved in a different way, by keeping
the same diagonal function in the whole process.
Here the conclusion is convertible to $\diag\;3$ with:

\coqdocinput{chunk26}

\vspace*{-.7\baselineskip}
\noindent
Then the case analysis on $H$ leaves a subgoal for \ET,
since 3 matches $(S\: (S\:n))$.
That is, we have to prove 
$\diag\;(S\: (S\:y))$ with an additional hypothesis $Hy: \eveni\;y$.
A case analysis on $Hy$ yields two subgoals:
$\diag\;2$ and $\diag\;(S\: (S\: (S\: (S\:z))))$,
because $y$ is either $0$ or $(S\: (S\:z))$, and
these 2 subgoals reduce to \True.

This strategy works for arbitrary large odd values,
see \cite{small_inv} for more complex examples.
Measurements on the corresponding proof terms showed
that their size is 1 to 2 orders of magnitude smaller
than with the standard \inversion of Coq.


% \medskip
% \coqdocinput{chunk23}
% \medskip


%%% Local Variables: 
%%% mode: latex
%%% TeX-master: "cpp12"
%%% End:

\section{Handling positive cases}
\label{sec:improvement}

However, 
the technique explained in the previous section 
has to be extended in order to cover more general
situations. 

A first easy improvement makes \diag independent
from the conclusion.
To this effect, we replace it with $(\forall X, X)$ 
in the first branch of \diag.
In our previous example, this yields

\coqdocinput{chunk27}

\noindent
Then the previous proof term 
(\match $H$ \intac $\eveni\;n$ \retac $\diag\;n$ \with 1 $\ldots$)
has the type $\forall X, X$
and then can be successfully applied to any current conclusion.
Alternatively, we can define a general function as follows:

\coqdocinput{chunk29}


\medskip
Next consider the following theorem:

\coqdocinput{chunk28}

\noindent
The proof is by induction on $\eveni\;n$.
In the inductive step, we have to prove $\eveni\;m$
from the induction hypothesis $\eveni\:(n + m) \rightarrow \eveni\;m$
and a new hypothesis $H: \eveni\: (S\: (S\: (n + m)))$.
Intuitively, we want to invert $H$ in order to push $\eveni\:(n + m)$
in the environment. 
The trick given at the end of Section~\ref{sec:absurd}
is then of no help.
However, we can instead adapt \prone as follows:

\coqdocinput{chunk30}

\noindent
Then, applying \prET to $H$ yields a function in continuation passing style.
Its type parameter $X$ is automatically identified to the conclusion
$\eveni\;m$, while $y$ is bound to $n+m$,
so that we get a new goal $\eveni\:(n+m) \rightarrow \eveni\;m$.
That is, we have exactly the expected inversion.
Functions such as \prone and \prET can be seen as inversion
lemmas, but note that their type is the dependent type
expressed by their own \diag.

More generally,
in the case of an inductive type $A(u)$ with parameter $u:U$,
given a constructor of type $\forall \mathbf{p}, A \;p$, 
where $\mathbf{p}$ is a telescope and $\cal P$$:U$ is an expression made
of constructors in the type $U$,
we proceed similarly:
the \match of \diag has a first branch filtering $\cal P$
and returning 
$\forall X: Prop, (\forall \mathbf{p}, X) \rightarrow X$.
If several constructors are possible for $A \:\cal P$,
say $C_1: \forall \mathbf{p_1}, A \:\cal P$,
$\ldots$, $C_n: \forall \mathbf{p_n}, A \:\cal P$,
the inverting lemma corresponding to $A \:\cal P$ will be:

\coqdocinput{chunk19}

\noindent
Remark the close relationship with the impredicative encoding
of data-types in system F.


%%%% END of JF, BEGIN of XM

The next stage to be considered is the case of
an inductive type with more than one parameter.
This raises new issues, because additional identities
between arguments of the premises or the conclusion
of a constructor may occur.
This happens routinely in our main experiment,
to be described below in Section~\ref{sec:simsoccert}.
In order to explain the problems and how to deal with
them in our framework, 
we introduce a toy language.

% To change to this new example is closer to what we have during using CompCert C
% language semantics in SimSoC-Cert project.
% It helps to explain the next problems we met.

The following code is the Coq description of this toy language.
The inductively defined evaluation rule $eval$ has two parameters.
The first one is the input type $tm$, $tm\_const$ and $tm\_plus$ are the
expected cases in pattern matching.
The second parameter is an output of type $val$, which is either nat or bool.
%and it is the extra variable we have to deal with.

\medskip
\coqdocinput{chunk31}
\medskip

In constructor $E\_Plus$,
the two premises share the variables $t1, t2, n1, n2$ with the
conclusion. 
If we use the last solution 
with continuation passing style, as it is presented above,
we are able to keep the premises 
but the relationship between the output values
as specified in the inductive definition will be lost
in the generated subgoal.

In order to solve this issue,
we add a suitable parameter to $X$ 
corresponding to the second parameter of the inductive relation.
The function for extracting the premises of $E\_Plus$ is:

\medskip
\coqdocinput{chunk34}
\medskip

Consider the following example.

\medskip
\coqdocinput{chunk38}
\medskip

By applying $pr\_plus$, $v$ will be equated to $nval~(plus~n1~n2)$ 
according to the rule specified by $E\_plus$.

% Next, consider an extra hypothesis $P\; v$ taking the same variable $v$. 
% This $v$ needs to be assigned to $nval~(plus~n1~n2)$ too.
% So except the inductive predicate we want to invert, 
% all other hypothesis share the same parameters which need to be assigned by
% equivalence relation, should be first revert to the goal.

% The second thing is when other hypothesis contain the same variables.
% Hypothesis $eval (tm\_plus (tm\_const 1) (tm\_const 0)) v$ is
% our inverting target, where $v$ is the output value we want to keep the.
% And hypothesis $P v$ is taking varialbe $v$ as parameter
% We want all 

% In order to have them changed to the same value as in the goal when perform
% applying rules. So we need Coq tactic \coqdockw{revert} to put it in goal before
% apply a corresponding definition.

% The third problem to solve is,
% according to what we have in proof goal we may need to provide different
% diagonalization function for some constructor.


% Then we want to have the equility $n1 = 1$ and $n2 = 0$.
% Normally, we will define a diagonalization function having the similar form 
% as $pr\_plus$ using a $X$ of type $val -> Prop$.

% \medskip
% \coqdocinput{chunk32}
% \medskip

% But the fact is
% applying such function will keep the equivalence between $nval n$ and
% $nval 1$. Then the proof assistant will complain that in goal we have nothing
% in form of $nval n1$ but in form of $nval (n1 + n2)$ with operator $+$.
% Then we know in this case,
% the variable we want to keep is not $nval n1$ but $n1$ which is of type 
% $nat$. But you can't have this $n1$ for free.
% So the diagonalization function will match two parameters: the key parameter
% matching type $tm$ of expected result $tm\_const tc$,
% and the evaluation output $v$ of $nval n$ in order to
% register $n$ independently.

% \medskip
% \coqdocinput{chunk33}
% \medskip

% And the other diagonalization function $pr\_const\_1$ will be used
% when the goal is in normal form $nval n$,
% in this case $pr\_const\_1\_2$ won't be able to help.




%%% Local Variables: 
%%% mode: latex
%%% TeX-master: "cpp12"
%%% End: 

\section{Application to SimSoC-cert}
\label{sec:simsoccert}

SimSoC-Cert~\cite{rapido11,cpp11} aims at certifying the simulator SimSoC, 
which is a complex hardware simulator written in C and C++.
SimSoC is able to simulate ARM and PowerPC architectures and is 
efficient enough to run Linux on both of them at a realistic speed.
The main objective of SimSoC is to help designers of embedded systems:
a large part of the design can be performed on software,
which is is much more convenient, flexible and less expensive
than with real specific hardware components.
However, 
this only makes sense if the simulator is actually faithful to the real
hardware.
Therefore we engaged in an effort to provide a formal certification
of sensitive parts of SimSoC.
More precisely, we consider the Instruction Set Simulator (ISS)
for the ARM, which is at the heart of SimSoC.
This ISS is called Simlight.

To this effect, we provide first defined a formal model in Coq of the ARM
architecture, as defined in the reference manual.
This is essential for defining the reference expected behavior
of SimSoC.
Our second input is the operational semantics of the ISS
encoded in C. 
This program is actually written in a large enough subset of C
called Compcert-C,
which is fully formalized in Coq \cite{Leroy-Compcert-CACM}.

We can then compare the behavior of the ISS encoded in C 
with the expected reference model directly defined in Coq.
To this effect, a projection between the Coq model of the
memory state of Simlight to the states in the reference model
is defined.
Then, correctness statements express that from a 
C memory $m_1$ corresponding to an abstract state $s_1$,
performing the function claimed to represent a given instruction $\cal I$
in Simlight 
will result in a C memory $m_2$ which actually corresponds 
to the abstract state $s_2$ obtained by
running the Coq model of $\cal I$. 
This can be put under the form of a commutative diagram as
schematized in Fig.~\ref{fig:thrm}.
%% HERE THE FIGURE 

\begin{figure}
\hfil\includegraphics[width=.5\linewidth]{theorem.pdf}
\caption{Correctness of the simulation of an ARM operation}
\label{fig:thrm}
\end{figure}

In details,
changes in the C memory model are formally described
by a transition system according to the operational semantics of Compcert-C.
Therefore, the later is used everywhere in the proofs. 
This operational semantics is described by
a big mutual inductive type;
in particular, the evaluation of expressions is defined by
16 constructors, one for each CompCert C expression like assignment or binary
operation.
% And this inductive type evaluation of expression is our target to try the
% improved new inversion.
A typical proof step start from a goal intuitively saying that,
given two C memory states related by a C expression,
as expressed in some hypothesis $H$, 
some commutative diagram holds.
Then $H$ is inverted, which either yields more elementary 
transitions between C memory states, 
with corresponding expected commutative diagrams,
or solves the current diagram if the considered expression
is atomic. 
In general we see that an inversion will result in many
new opportunities to perform inversions.

In our first proofs, using the Coq standard \inversion tactic
resulted in a very weak control on the script.
Finding the right relation to focus on in the hypotheses
was inconvenient.
Interactive execution of the script was also quite slow,
due to the size of the terms generated by \inversion.
And the compilation time for the proof on only one instruction 
took more than one minute --
there are more than one hundred instructions in the ARMv6 architecture.
Moreover
the proof code is fragile in case of changes: 
after \inversion, hypotheses are automatically given similar names 
according to a simple numbering scheme,
so that any modification at the beginning of the proof script
or in auxiliary lemmas 
result in a complete renaming of hypotheses to come in the sequel.
This is quite harmful in practice and constitutes
a serious issue for maintenance.

The following code shows an small excerpt from 
an old proof script in SimSoC-Cert using \inversion.
In this example, we want to find out the relation between memory states
\coqdocvar{m} and \coqdocvar{m'} expressed in the hypothesis $H$,
whose main argument (\coqdocvar{Ecall (Evalof...})
represents the program to be executed.
Its type is given by the inductive relation \coqdocvar{eval\_expr}.
%To achieve this goal, we use \inversion step
%following step by step the definition of semantics of \coqdocvar{eval\_expr}.
We see that many \coqdocvar{inv Hx} are used.
Here \coqdocvar{inv H} denotes \coqdocvar{inversion H; clear H; subst}.

\medskip
\dots \\
\coqdocinput{chunk41}

\coqdocinput{chunk42}
\dots
\medskip

The program for simulating an ARM instruction 
usually contains expressions more complex than in the
example given here.
% We have to invert the predicate so many times to find the relation between
% initial and final memory state. 
This cumbersome work is needed in every instruction proofs,
and there is no clear way to share anything since
the corresponding programs are rather specific,
at least for instructions belonging to different categories.

Thanks to the technique introduced in this paper,
we could define convenient reusable tactics.
First, we defined suitable diagonal-based functions for each
constructor of $eval\_expr$ following the lines given in the
previous section.
Then we packaged them altogether in a high-level tactic named 
$inv\_eval\_expr$ using a Ltac definition. The arguments of this
tactic are the memory states under focus --
in the example above: $m$ and $m'$.
This tactic also contains extra features so that
it is able to:
%
\begin{enumerate}
\item find automatically an hypothesis to be inverted;
\item repeatedly perform our hand-crafted inversion until all constraints
  between two memory state are derived;
\item give meaningful names to the derived constraints;
\item update all other related hypotheses according to the new 
  variables names or values;
\item clean up useless variables and hypotheses.
\end{enumerate}
%
Then all the eighteen \inversion in the example above are solved in just
one step using $inv\_eval\_expr m m'$.

To be fair, let us mention that the robustness issue could,
in principle, be managed using the current version of standard \inversion,
because of allows us 
to give explicit names to introduced variables and hypotheses.
However, due to the complexity of CompCert C semantics, 
providing all these names explicitly turns out to be cumbersome,
and unrealistic when you face dozens of consecutive \inversion 
and about ten names for each \inversion.
Within our framework,
the introduction suitable names 
is automatically performed inside $inv\_eval\_expr m m'$.
Names are chosen according to the contents and in a flexible way, 
so that the evolution of goals is easy to follow.

As a unexpected benchmark, 
when CompCert C semantics has been changed to a new version
in the end of 2011,
we didn't have to change the proofs:
just updating the tactic turned out to be enough. 

Comparing development times is provides additional hints.
In our first try, dedicated to the instruction ADC,
more than two months were spent on the development of the correctness proof.
The inversion technique presented here was not available
at that time, entailing the drawbacks detailed above.
With the new approach, proofs for 4  other instructions
could be finished in only one week. 
The high-level tactic described above required
less than 2 weeks.


Finally, 
we compare the efficiency of the standard Coq \inversion with our new tactic
in Table.~\ref{t:timing}.
The first line is about the whole expression given in the example above. 
The other lines are inversions of specific expressions.
We can observe a gain of about 4 to 5 times.

\begin{table}\centering
\label{t:timing}
\caption{Comparison of the time costs}
\begin{tabular}{|l|c|c|}
\hline
 & standard \inversion & our inversion \\
\hline
Full example &  1.856& 0.404\\
\hline
Ebinop & 0.104&  0.020\\
\hline
Evalof &  0.096& 0.020\\
\hline
Eval &  0.116& 0.020\\
\hline
Evar &  0.108& 0.024\\
\hline
\end{tabular}
\end{table}



%%% Local Variables: 
%%% mode: latex
%%% TeX-master: "cpp12"
%%% End: 

\chapter{Discussion and conclusion}
\label{cpt:concl}

% % JF->XM: general comment: self-defined ---> user-defined

We developed the certification of a part of
an ARM instruction set simulator called \simlight,
using
% the formal representation of the concrete C program
% according to
the operational semantics of the C language provided by the \compcert project.
Correctness proofs were performed under the interactive proof assistant
Coq.
A large part of the Coq specification and of the model of the simulator
were automatically produced from the pseudo-code available in the ARM reference manual.
A Coq proof technique for performing \emph{inversions} was introduced in
order to solve cumbersome proof steps in our work
in a better way than Coq built-in tactics.
Moreover, the size of proof terms generated by our our \hcinv
is much lower than with built-in Coq \inversion,
making Coq type checking and compilation more efficient.
Additionally, we have built a test generator for the ARM instruction decoder,
which generates massive tests covering all ARM instructions.

The following sections contains
an assessment on the usage of operational semantics
in proving the correctness of \simlight and the feasibility of
using this new approach for proving general C programs,
the overall development size of SimSoC-Cert and the TCB.
We conclude with prospects of future work.

\section{Using operational semantics for proving C programs}

The certification technique we applied for \simlight is based on the C operational
semantics provided by \compcert.
The Coq formal representation of the C programs of each ARM instruction
can be obtained from
the instruction pseudo-code intermediate representation AST in two ways:
either by translating it to \compcert C AST,
% JF: the important part of this pretty-printing is more the sharing of types,
% something that  you rightly don't want to explain here...
% moreover it is not part of the comparison since used in the 2 options
% -> let's just forget it
% and pretty-printing it in Coq,
or by translating it to a textual C program,
then parsing it to \compcert C AST using the \compcert C parser.
% and pretty-printing it in Coq.
%
In our experiments, no difference could be observed between the two approaches
-- no information was lost using \compcert C parser.
\compcert C supports a C subset which is rich enough to describe the
operations of ARM instructions.
% % JF -> XM: I see your point but you have to be more careful.
% % OK, we avoid unproved components as far a possible, and
% % the parser is not proved -- it is actually unclear how to get
% % a suitable specification.
% % That said, we are already confident that this parser is good.
% % Here we get additional evidence that this pareser can be trusted
% % but it is not a breakthrough.
% In the future, we can expect other C programs without their own intermediate
% representation AST to use the \compcert C parser directly.

Correctness proofs were performed using the Coq proof assistant.
In this approach to the certification of C programs,
the Coq proof steps in Coq are not simple.
However, we were actually able to consider C programs
having a large size and complex specification,
using the full expressive power of Coq.
%
% % JF: yes we could speak again about ax sem, but we have nothing
% %  more than claims staed inthe intro, since we did not (want to) try.
% than using axiomatic semantics.
%
Our work assesses the feasibility of using operational semantics for
certifying C programs.

Proof steps related to the \compcert C semantics can be simplified
a lot by defining Coq tactics with Ltac (the tactics language).
% % JF: we know we know...
% As an interactive proof assistant, the proof steps require the
% interaction with its user.
Our initial first proof script for ADC instruction contained thousands
of lines of code.  Then, we identified repetitive sequences and
started to define our own proof tactics in the Ltac language,
resulting into much shorter proof scripts.  The second version for ADC
correctness proof was approximately three times smaller than the
first one.
% Repetitive proof steps containing seqences of commands
% can be defined in one general Ltac definition.
In the design of these tactics, we did not seek for generality.
However, since ARM instructions within the same category often have
very similar statements and expressions, our tactics can actually be
reused.
% the user-defined tactics can be reused in all of them.

In Section~\ref{sec:tactic}, we have introduced more general tactics
implemented in SimSoC-Cert, like finding functions in the C memory
model, reusing load/store operations, etc.  Those tactics are not
specific to \simlight, they only deal with \compcert C semantics and
memory operations.  The same holds for our inversion technique: it was
implemented for the needs of SimSoC-Cert as a tactic \hcinv dedicated
to the inductive relations defined in \compcert (see
Section~\ref{ssec:invssc}).  However, these tactics can be reused in
other projects using the same approach to the correctness proof of
\compcert C programs, e.g. the CCCBIP project
which recently started in our group and aims at building a certifying compiler
from a high-level component-based language dedicated to embedded systems
(BIP), with \compcert C as its target. 


% To reason on the C operational
% semantics, \emph{inversions} of the evaluation rules are the essential steps.
% For other projects intend to use \compcert C operational semantics for
% C program certification, e.g. the CCCBIP project, it is possible to
% reuse the tactics defined in SimSoC-Cert.


\section{Hand-crafted inversion}
%%Copy from CPP12 need to be changed 

Our hand-crafted inversion presented in Chapter~\ref{cpt:inv}
was experimented on large proofs relying on big inductive relations
independently defined in the \compcert project.
It played a key role for the success of this approach to correctness proofs
of C programs, and
the extra flexibility provided by \hcinv inversions could be exploited to
produce smaller, more robust and manageable proofs.

It is not yet a fully automatic tactic, like the original \inversion. 
We think that
automation could be realized by interacting with the internals of Coq.
This would be done for efficiency concerns and would not harm
in the cases where the proof can be automatically completed,
or is followed by tactics which do not refer to names produced by inversion.

But in a project with a big size specification like \simsoccert,
where proofs require fine tuning,
interactions between the human and the proof assistant cannot be avoided.
In general, in such situations,
statements involve arbitrarily complex definitions,
so we cannot make the assumption that decision procedures can be used.
The issue is then to provide appropriate mechanisms,
so that writing proof scripts and interacting with the proof assistant
is made easy.
%
We think that our hand-crafted inversion technique is a good tool
in this respect:
it is flexible enough for the user,
practical situations can be managed
with a full control on the script and valuable improvements
of the script are easier to design.


Let us mention another possible application of the technique. Inversion is
sometimes needed to write a function whose properties will be established later
(as opposed to providing a monolithic and exhaustive Hoare-style specification
and along with a VC generator such as Program). In this context, simply using
the proof engine and the inversion tactic tends to generate unmanageably large
terms. We can expect our technique to be very helpful in such situations.

\section{Development size}
\begin{table}[ht]
  \centering
  \begin{tabular}{|l|r@{~}|}
    \hline
    Original ARM ref man (txt)           & 49655 \\
    ARM Parsing to an OCaml AST         & 1068 \\
    Generator (Simgen) for ARM         &   10675 \\
    Generator specifications for SH4      & 737 \\
    General C libraries on ARM         & 1852 \\
    General Coq libraries on ARM         & 1569 \\
    Generated C code for \simlight ARM operations   & 6681 \\
    Generated Coq code for ARM operations   & 2068 \\
    Generated Coq code for ARM decoding  & 592 \\
    Projection   & 857 \\
    Proof script for ADC (2011)    & 3171 \\
    Proof script for ADC (2012)    & 1204 \\
    Definition of \hcinv       &551\\
    Definition of other user-defined tactics      &185\\
%% Xiaomu: please complete
    Proof script for auxiliary functions   & 856 \\
    Proof script for BL (2012)   & 437 \\
    Proof script for LDRB (2012)   & 170 \\
    Proof script for MRS (2012)   & 322 \\
    \hline
  \end{tabular}
  \smallskip
  \caption{Sizes (in number of lines)}
  \label{tab:sizes}
\end{table}

Table~\ref{tab:sizes} shows the size of our development.
The size of the generator is almost the same as the total
number of lines of the generated part for ARMv6.
But note that this is the version redesigned by F. Tuong
in order to be more general,
so that it could be reused with other specific processors.
Currently, besides ARM, it is applied to the SH4 reference manual
where, instead of a specific pseudo-code,
instructions are described using a C syntax.
% There is a good hope that it could be used on other architectures
% \jf{Could be shortened because work by F Tuong.}
% According to different architecture, we will have different size of development.
% But thanks to the experiment of SH4, we are now certain that
% the framework is capable to be used for another processor
% architecture. The only requirement for the object processor is that
% its reference manual should be well structured and can be transformed
% to analyzable text. The most important thing is that it contains formal or
% semi-formal descriptions to be automatically translated to a specific
% intermediate representation.

One can note also that the generated code for the ISS
takes 50\:\% of the Coq formal model,
and almost 70\:\% of the C simulator.
Although the gain may be considered as not that large,
we think that it was worth taking this approach,
given the repetitive nature of instructions.
% Considering the development worthiness,
% the instruction set simulation should be complex enough;
% at least the generated code is more than the specification for such processor.
% As we mentioned in Section~\ref{ssc:arm}, the core of a processor
% simulator is ISS.

About the proof efforts,
the first experiment on the correctness of ADC
%(11 lines of pseudo-code)
costed one month.
The number of Coq lines for the proof script is quite large
(about 3200 for the first version),
especially if we compare with the 11 lines of the corresponding
pseudo-code in the reference manual.
% And the size of the experiment code is quite huge,
% with no optimization or user-defined tactics applied, 3171 loc.
At this stage, we did not develop user-defined tactics.
Now, using \hcinv and other user-defined tactics,
not only maintenability is much improved,
but the development time for a proof is much lower.
Less than one week is needed for an instruction as complicated as ADC.
% Compared to the size of the definition of \hcinv
% and other user-defined tactics,
% the number of lines of code
% of the new version of correctness proof script has been highly
% reduced. And using the new \hcinv with the experiment we obtained,
% the development time has been saved, too.
% To complete a correctness proof of an instruction,
% we need only less than a week,
% which is enough for a quite complicate instruction
% like ADC.
%
% % JF -> XM: you said 12 but I count 11 here!
% % Either the count is wrong, or you forgot an instruction
Until now, 11 instructions were proved correct,
one from each instruction category.
They are given in Table~\ref{tab:prvinst}.
% (the instruction we have proved
% in the category is inside the parenthesis) :
% branch ins truction (BL),
% data processing instruction (ADC),
% Multiply instruction (MUL),
% parallel arithmetic addition and subtraction instruction (QADD16),
% extended instruction (XTAB16),
% miscellaneous arithmetic instruction (CLZ),
% status register accessing instruction (MRS),
% load and store instruction (LDR),
% load and store multiple instruction (LDM),
% semaphore instruction (SWP),
% and exception generating instruction (BKPT).
\begin{table}[ht]
  \centering
  \begin{tabular}{|l|l|}
    \hline
    Category & Instruction name \\
    \hline
    branch & BL \\
    data processing & ADC \\
    multiply & MUL \\
    parallel arithmetic addition and subtraction & QADD16 \\
    extended instruction & UXTAB16 \\
    miscellaneous arithmetic & CLZ \\
    status register access & MRS \\
    load and store & LDR \\
    load and store multiple & LDM \\
    semaphore & SWP \\
    \hline
  \end{tabular}
  \smallskip
  \caption{ARM instructions having a correctness proof}
  \label{tab:prvinst}
\end{table}

\section{Trusted Code Base}

% The trust we may have in our result depends on the faithfulness of its
% statement with relation to the expected behavior of the simulation of
% \texttt{ADC} in \simlight.  It is mainly based on the manually written
% Coq and C library functions, the translators written in OCaml
% described in Section~\ref{sec:overall} (including the
% pretty-printer for Coq), the final phase of the Compcert compiler, and
% the formal definition of $\mathit{proc\_state\_related}$.

Our proofs depend on several tools developed elsewhere:
the Coq proof assistant,
the OCaml compiler and the \compcert C certified compiler.
The TCB of these external tools have to be considered independently.
Regarding Coq, the TCB is essentially its kernel.

Next, the TCB includes the formal version of the ARM reference manual
on which proofs are carried on:
hand-written and automatically produced Coq definitions,
as described in Figure~\ref{fig:arch}.
Alternatively,
automatically produced Coq definitions could be replaced by
the textual reference manual (patched by our bug fixes)
and Coq code generators.
% % JF: no, unless we consider simlight itself, and in that
% %  case we have to add gcc
% the \compcert C AST pretty-printer
% the hand-written Coq, %  and C library functions, %% NO they are proved
% the generated Coq and C representations,
The TCB also includes
the Coq projections from the \compcert C AST representation of \simlight code
to our abstract Coq model.
% Altogether, we have the TCB of
% the correctness statements for relating \simlight to the expected behavior
% as in the formal model.

% % JF -> XM What do you mean? Why the TCB should grow?
% % --> commented out unless something important has to be stated.
% In the future, the development size of correctness proofs will grow
% larger when all the work is done for 147 ARM instructions
% and all the auxiliary functions,
% approximately three times comparing to our TCB.


% \section{Validation}

% \jf{Optionally remove?}

% The validation of new ISS of ARMv6 integrated into SimSoC has been
% introduced in Section ~\ref{ssc:vali}. And a brief speaking on
% Evaluation of Coq specification has been mentioned there, too. In
% general, the C specification can be executed as fast as the older
% hand-written version of ARMv5, but we gain more confidence in the new
% version with the correctness proofs. Although the execution speed is
% very low, the extraction from formal model is still an executable
% and reliable model.

% Using a generator avoids many typo-like errors. However, other kinds
% of errors remain possible.
% Besides the bugs in the documentation which were reported before,
% there are the last bugs we found and fixed
% while trying to boot Linux on the SPEArPlus600 SoC simulator:
% \begin{itemize}
%  \item After the execution of an {\stt LDRBT instruction}, the content
%    of the base register ({\stt Rn}) was wrong. It was due to a bug in
%    the reference manual itself; the last line of the pseudo-code has
%    to be deleted\footnote{This error is fixed in the ARMv7 reference
%      manual, which is now the recommended manual for the ARMv6
%      architecture.}.
%  \item
%    Base register is where the base address stored.  The address may be
%    changed by load/store instructions write-back.  In the operation of
%    such instruction, the write-back to base register should ahead of the
%    certain memory access.  If the memory access fails, the base
%    register then must keep the original value. Such rule is only
%    explained informally.  Our generated ISS manages {\em data aborts}
%    using C++ exception mechanism. As a consequence, moving the
%    statement doing the write-back at the end of the instruction code
%    (and so after any possible {\stt throw}) is sufficient to keep the
%    base register unchanged in case of exception.
%    Some load/store in structure can modify the processor mode; in this
%    case the meaning of base register may be changed. Then write-back
%    must affect a banked version of it instead of current version.
%  \item Additionally, there is a half-word access to an odd address
%    while executing SPEArPlus600 specific code. In this case, the
%    manual indicates that the result is ``unpredictable''. % ...
%  \end{itemize}

\section{Future work}

% % JF -> XM, [GC] general comment:
% % I removed parts wich could be considered as weak or
% % not significant enough.

The next step would be to
extend the work done on \texttt{ADC} and other operations
given in Table~\ref{tab:prvinst} to the full ISS.
% different instruction categories. % JF It is already done!!
% % JF : said above
% The reused parts are the common functions and user-defined tactics,
% which can reduce much of the developing time during proving phase.
% This shorten a lot the current proof script and make it easy
% to understand and to be more generalize.
We are confident that the corresponding work on
the remaining ARM instructions can then be done much faster.
% % JF -> XM : added the following, it would make sense to say
% %  how many lib fun are available / to be done
In particular, a number of lemmas on 14 library functions
are already available.
71 library functions remain to be done.
% % JF: see [GC]
% And it will be also helpful to study the proofs between correctness of
% instructions in the same category.
% The instructions in the same category are constructed
% very similar to each other.
% The common parts then can be
% applied to all instructions in one category.

% Internal functions are described in an informal manner in the
% ARMv6 reference manual.
% No pseudo-code is available for them,
% which means that the corresponding library functions,
% both in the abstract Coq model and in \simlight,
% are written by hand.
% In order to get a suitable \compcert C AST to reason about,
% we use the parser provided in \compcert.
The hand written library functions in \compcert C ASTs are obtained
using the \compcert C parser.
Currently, they are merged with instructions by hand,
and identifiers used in these functions are added manually,
in order to solve a technical issue
stated on page~\pageref{page:libfunast}.
It would be better to build a ``hook'' which automatically finds
the called functions in the parsed ASTs and
generates unused block numbers for the corresponding identifiers.

We also attempted to write a Coq (functional) version of the decoder,
but strong improvements are required to make it usable.
The current version is based on a huge pattern-matching,
%The resulting decoding algorithm is quite weak,
which considers the 32 bits of a binary instruction
in a carefully designed order.
% depending on a roughly concluded decoding order.
% Decoding should be more specific,
% which is one of the very important parts of processor simulator.
We started to design a better version of this decoder,
considering the semantics of bit fields.
% A new version of decoder is planned to decode the binary instruction by the
% bits field separately.
% % [GC] and too technical for a conclusion.
% For example,
% the key bits field $[27:25]$ is $010$ and then
% the instruction must be one of the load/store instruction
% under immediate offset addressing mode;
% or when its value equals to $101$,
% the instruction is a branch instruction.
% After deciding which category it is,
% it should use the other bit field represented parameter
% to decode which concrete instruction it is.
Then, proofs for the decoder could be considered as well
-- automatic extraction tools from the ARM reference manual
are already available.
Finally, the simulation loop
(basically, repeat decoding and running operations) can be be proven.

In another direction, our methodology can be reused on other processors,
such as SH4.

% proof for simlight 2
In the future, \simlight 2 could be considered as well.
\simlight 2 has adopted several optimization methods for a higher simulation
speed. The most important difference is the ``flattening'' method applied
to the instruction set (see Subsection~\ref{sec:codegen}).
Some instructions are merged with their addressing mode,
and the \simlight 2 decoder decodes the instruction and its addressing
mode at the same time. 
Then the C definition is simpler than in \simlight with less function calls.
We expect the proof for this \simlight 2 decoder
to be less difficult than \simlight.
%
Instruction operations in \simlight 2 are essentially the same as for \simlight.
The main optimization used in \simlight 2 
is to specialize some of the parameters according to actually used values.
Therefore, one ARM instruction operation is implemented by
several functions in \simlight 2, instead of just one function in \simlight;
but the code of these functions is essentially the same,
so there is good hope that 
existing correctness lemmas for \simlight could be
restated and generalized in such a way that
instances of them would just be the expected correctness lemmas 
for the corresponding functions in \simlight 2.


% % JF [GC]
% based on automatic
% generation of simulation code and Coq specification for other
% processors.
% The one already considered is SH4.
% In fact, the same approach as the ARMv6 has been followed, and a similar Coq representation can currently be generated from the SH4 manual.
% The other core processor of SimSoC is PowerPC.
% To perform a similar work on PowerPC requires a pre-process replacing using the existing pdf-to-text
% step.
% Because the result of {\stt pdftotxt} is not complete, due to the two column structured documentation, lines are lost or contents
% are disordered.

% % JF : it is not future work
% Besides the correctness proofs on processor instruction operation, the
% other important achievement is building the hand craft version
% \hcinv.
% The technique introduced early in \cite{small_inv} on very small toy examples could be successfully used in a significant application,
% up to suitable extensions in order to conveniently get the premises of
% a constructor in non-absurd cases.
% As in \cite{2013itp}, we do not claim that we have a fully automated tactic, like \inversion.
% Our goal is more modest: providing a hand-crafted inversion technique
% which is flexible enough for the user, so that most practical
% situations can be managed with a full control on the script and
% valuable improvements on robustness.
% Moreover, the extra flexibility provided by hand-crafted inversions can be exploited to produce
% smaller, more manageable proof terms.

% % JF [GC]
% The experiment with SimSoC-Cert correctness proof is introduced in
% Chapter ~\ref{cpt:inv}, which relies on the big inductive relation
% representing the operational semantics of \compcert C defined by
% \compcert project.
Our group recently started another project aiming at
the implementation of certified software written in BIP,
a high-level component-based language dedicated to embedded systems,
with \compcert C as an intermediate target.
We expect the work presented is this thesis to be reused there.
More generally,
our implementation \hcinv dedicated to \compcert
can be re-used in any application of \compcert operational semantics
for proving C programs.
However, it has to be updated accordingly to the new releases of \compcert.

% % JF : it is not future work
% Let us mention another possible application of the technique.
% Inversion is sometimes
% needed to write a function whose properties will be established later (as
% opposed to providing a monolithic and exhaustive Hoare-style specification and
% along with a VC generator such as Program).
% In this context, simply using the proof engine and the \inversion tactic
% tends to generate unmanageably large terms.
% %We can expect our technique could be very helpful in such situations.
% We expect our technique to be very helpful in such situations.

% and the line-count metrics given at the end of \cite{small_inv} makes sense

% reduce developpment time
% ease refactoring a lot


%%% Local Variables:
%%% mode: latex
%%% TeX-master: "thesis"
%%% End:



\bibliographystyle{abbrv}
\bibliography{biblio}

\end{document}


%-------------------------------------------------------------------------


%%% Local Variables: 
%%% mode: latex
%%% TeX-master: "cpp12"
%%% End: 
