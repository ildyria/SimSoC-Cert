\svnidlong
{$HeadURL$}
{$LastChangedDate$}
{$LastChangedRevision$}
{$LastChangedBy$}

% Author: \svnfileauthor; Revision: \svnfilerev; Last changed on: \svnfiledate; 
% URL: \url{\svnkw{HeadURL}}

\begin{thoughts}
\itshape
\hfil -----------------------------------------------------------------------------------\par
\hfil \textbf{Changes on \currfilename}

Author: \svnfileauthor; Revision: \svnfilerev; Last changed on: \svnfiledate
\end{thoughts}

% svn propset svn:keywords 'LastChangedBy LastChangedRevision LastChangedDate HeadURL' thisfile.tex

%%%%%%%%%%%%%%%%%%%%%%%%%%%%%%%%%%%%%%%%%%%%%%%%%%%%%%%%%%%%%%%%%%%%%%%%%%%%%
\subsection{Handling Successful Cases}
\label{sec:improvement}


%\vspace*{-.7\baselineskip}
% Let us now consider what happens if $H$ is 
% $\eveni\;3$ instead of $\eveni\;1$. 
% As mentioned in the introduction, 
% a first inversion on $H$ will push $\eveni\;1$ in the environment, 
% and then we are back to the previous situation.
% In \cite{small_inv} we show that in such situations 
% the goal can be proved in a different way, by keeping
% the same diagonal function in the whole process.
% Here the conclusion is convertible to $\diag\;3$ with:

% \coqdocinput{chunk26}

% \vspace*{-.7\baselineskip}
% \noindent
% Then the case analysis on $H$ leaves a subgoal for \ET,
% since 3 matches $(S\: (S\:n))$.
% That is, we have to prove 
% $\diag\;(S\: (S\:y))$ with an additional hypothesis $Hy: \eveni\;y$.
% A case analysis on $Hy$ yields two subgoals:
% $\diag\;2$ and $\diag\;(S\: (S\: (S\: (S\:z))))$,
% because $y$ is either $0$ or $(S\: (S\:z))$, and
% these 2 subgoals reduce to \True.

% This strategy works for arbitrary large odd values,
% see \cite{small_inv} for more complex examples.
% Measurements on the corresponding proof terms showed
% that their size is 1 to 2 orders of magnitude smaller
% than with the standard \inversion of Coq.

% However, 
% the technique explained in the previous subsection 
% has to be extended in order to cover more general
% situations. 

A first easy improvement makes \diag independent
from the conclusion.
To this effect, we replace it with $(\forall X, X)$ 
in the first branch of \diag.
In our previous example, this yields

\vspace*{.5\baselineskip}

\coqdocinput{chunk27}

\noindent
Then the previous proof term 
(\match $H$ \intac $\eveni\;n$ \retac $\diag\;n$ \with 1 $\ldots$)
has the type $\forall X, X$
and then can be successfully applied to any current conclusion.
Alternatively, we can define a general function as follows:

%\vspace*{.5\baselineskip}

\smallskip
\coqdocinput{chunk29}


\medskip\noindent
Next consider the following theorem:

\smallskip
\coqdocinput{chunk28}
\smallskip

\noindent
The proof is by induction on $\eveni\;n$.
In the inductive step, we have to prove $\eveni\;m$
from the induction hypothesis $\eveni\:(n + m) \rightarrow \eveni\;m$
and a new hypothesis $H: \eveni\: (S\: (S\: (n + m)))$.
Intuitively, we want to invert $H$ in order to push $\eveni\:(n + m)$
in the environment. 
% The trick given at the end of Section~\ref{sec:absurd}
% is then of no help.
We can then adapt \prone as follows:

\smallskip
\coqdocinput{chunk30}

\noindent
Then, applying \prET to $H$ yields a function in continuation passing style.
Its type parameter $X$ is automatically identified to the conclusion
$\eveni\;m$, while $y$ is bound to $n+m$,
so that we get a new goal $\eveni\:(n+m) \rightarrow \eveni\;m$.
That is, we have exactly the expected inversion.
Functions such as \prone and \prET can be seen as inversion
lemmas, but note that their type is the dependent type
expressed by their own \diag.
\medskip

More generally,
let us then invert an hypothesis $H$ having the type $A\: \mathcal{P}$
where $A(u)$ is an inductive type with index $u:U$
and $\mathcal{P}:\;U$ is an expression made
of constructors in the type $U$.
%
%In the case of an inductive type $A(u)$ with index $u:U$,
Given a constructor of type $\forall \mathbf{p}, A \;p$, 
where $\mathbf{p}$ is a telescope 
%and $\mathcal{P}:\;U$ is an expression made of constructors in the type $U$,
we proceed similarly:
the \match of \diag has a first branch filtering $\mathcal{P}$
and returning 
$\forall X: Prop, (\forall \mathbf{p}, X) \rightarrow X$.
If $n$ constructors are possible for $A \:\mathcal{P}$,
say respectively $C_1: \forall \mathbf{p_1}, A \:\mathcal{P}$,
$\ldots$, and $C_n: \forall \mathbf{p_n}, A \:\mathcal{P}$,
the inverting lemma corresponding to $A \:\mathcal{P}$ will be:

\smallskip
\coqdocinput{chunk19}
\smallskip

\noindent
Remark the close relationship with the impredicative encoding
of data-types in system F.


\subsection{Dealing with Constrained Arguments}
\label{sec:constrained-args}

The next stage to be considered is the case of
an inductive type with more than one index.
This raises new issues, because additional identities
between arguments of the premises or the conclusion
of a constructor may occur.
This happens routinely in the inductive definitions
for the operational semantics of C provided by CompCert.
In order to explain the problems and how to deal with
them in our framework, 
we introduce a toy language, together with
an inductively defined evaluation rule $eval$ having two indexes:
the first one is the input type $tm$, $tm\_const$ and $tm\_plus$ are the
expected cases in pattern matching;
the second index is an output of type $val$, which is either nat or bool.
%and it is the extra variable we have to deal with.

\medskip
\coqdocinput{chunk31}
\medskip

In constructor $E\_Plus$,
the two premises share the variables $t1, t2, n1, n2$ with the
conclusion. 
If we use the last solution 
with continuation passing style, as it is presented above,
we are able to keep the premises 
but the relationship between the output values
as specified in the inductive definition will be lost
in the generated subgoal.
%
This issue is handled using an additional argument to $X$ 
corresponding to the second index of the inductive relation.
The function for extracting the premises of $E\_Plus$ is:

\medskip
\coqdocinput{chunk34}
\medskip

\noindent
Now, consider the following examples.

\smallskip
\coqdocinput{chunk38}
\medskip
%
\noindent
In $\coqdocvar{ex1}$, by applying $pr\_plus\_1$, 
$v$ will be equated to $nval~(plus~n1~n2)$ 
according to the rule specified by $E\_plus$.
%
In $\coqdocvar{ex2}$,
we need to analyze at the same time the two arguments of $eval$.
The corresponding premises are extracted using a function $pr\_plus\_1\_2$
having the same body as $pr\_plus\_1$, but whose type is:

\medskip
\coqdocinput{chunk39}
\medskip
%
\noindent
A similar situation happens with $E\_Const$ in the 
two previous examples.
%subgoals generated for $\coqdocvar{ex1}$ and $\coqdocvar{ex2}$.
% to be handled with a corresponding inverting function $pr\_const\_1\_2$.

% \medskip
% Defining an inverting function for each constructor 
% is flexible and convenient for debugging.
% An elegant alternative\footnote{%
% We want to thank the anonymous reviewer who offered this remark.}
% is to merge all of them into a unique
% inverting function managing all cases of the argument(s) under focus.
% For instance, an exhaustive inverting function $pr\_eval\_1\_2$ 
% suitable for $\coqdocvar{ex2}$ has the type:

\medskip
Defining an inverting function for each constructor 
is most convenient for debugging.
However the method is flexible and several such functions can be merged.
In particular, 
an elegant alternative\footnote{%
We want to thank the anonymous reviewer who offered this remark.}
is to provide a unique
inverting function managing all cases of the argument(s) under focus.
For instance, an exhaustive inverting function $pr\_eval\_1\_2$ 
suitable for $\coqdocvar{ex2}$ has the type:

\medskip
\coqdocinput{chunk33}
\medskip

Full definitions as well as additional examples can be found on-line~\cite{hci:examples}.

%%%%%%%%%%%%%%%%%%%%%%%%%%%%%%%%%%%%%%%%%%%%%%%%%%%%%%%%%%%%%%%%%%%%%%%%%%%%%
\subsection{Beating \inversion}
\label{sec:finset}

Let us consider now a predicate defined on a dependent type.
We take intervals $[1...n]$, formalized as $t$ in the standard library \texttt{Fin},
then we restrict them to have an odd length.

\medskip
\coqdocinput{chunk60}
\medskip

\noindent
Finding the premises for the second constructor is a function 
similar to the one provided for $E2$ above:

\medskip
\coqdocinput{chunk61}
\medskip

\noindent
In particular we can easily prove:

\medskip
\coqdocinput{chunk62}
\medskip

\noindent
Standard \inversion happens to fail here.
Note that BasicElim may work (we actually could not succeed)
but would need an additional axiom related to John Major equality.



%%% Local Variables: 
%%% mode: latex
%%% TeX-master: "itp13"
%%% End: 
