\svnidlong
{$HeadURL: https://scm.gforge.inria.fr/anonscm/svn/simsoc-cert/papers/itp13/conclusion.tex $}
{$LastChangedDate: 2013-04-17 11:55:45 +0200 (Wed, 17 Apr 2013) $}
{$LastChangedRevision: 2298 $}
{$LastChangedBy: monin $}

% Author: \svnfileauthor; Revision: \svnfilerev; Last changed on: \svnfiledate; 
% URL: \url{\svnkw{HeadURL}}

\begin{thoughts}
\itshape
\hfil -----------------------------------------------------------------------------------\par
\hfil \textbf{Changes on \currfilename}

Author: \svnfileauthor; Revision: \svnfilerev; Last changed on: \svnfiledate
\end{thoughts}

% svn propset svn:keywords 'LastChangedBy LastChangedRevision LastChangedDate HeadURL' thisfile.tex

%%%%%%%%%%%%%%%%%%%%%%%%%%%%%%%%%%%%%%%%%%%%%%%%%%%%%%%%%%%%%%%%%%%%%%%%%%%%%
\section{Conclusion}
\label{sec:conclusion}


We see no reason why
the technique developed above for performing inversions could not
be automated and implemented in Coq or in proof assistants
based on a similar calculus.
One good motivation for that would be to get terms which are 
much smaller, easier to typecheck,
than with the currently available inversion tactics.
This can be very useful when interactively defining functions
on dependent types, for instance. %\todo{a}{manual already useful there indeed}

But we want to insist first on a much more important feature of
our approach, according to our experience with SimSoC-Cert:
its impact on goals during \emph{interactive} proof development
is \emph{actually controllable}.
We think that having much shorter underlying functions is
helpful in this respect:
they are short enough to be written by hand,
providing an exact view on what is to be generated.
We claim that this feature is especially relevant
to applications which make an intensive use of inversion steps:
in this situation, 
partial automation obtained by programming small controllable building blocks
turns out to be effective,
whereas automation tends to generate a 
response of the proof assistant which is not completely predictible. 
This may not harm too much if the generated goals can be fully discharged
without further interaction,
but this is not the general case.
In particular, this hope is vain when we deal with complex properties,
as in our application.
A better alternative would be to automatically generate 
auxiliary definitions such as \coqdocvar{inv\_field}.
However, we consider that our technique is already useful
and worth to be offered.

In contrast to available techniques \cite{cornes95automating,mcbride00}
we argue against the use of auxiliary equations or disequations:
the latter are better to be cleaned, 
in order to avoid clumsy additional hypotheses, 
which hamper the management of proof scripts;
however, it is not that simple to do.
The brute use of a tactic which performs all possible rewriting
steps, then cleans equalities avalaible in the goal, for instance, 
is not satisfactory
because some equalities already introduced by the user on purpose
could then disappear. 
Therefore, a special machinery is needed in order
to trace equalities coming from the inversion step under consideration
(e.g., the use of \coqdocvar{block} in BasicElim).
Our use of CPS encoding of Leibniz equality, on the other hand,
completely avoids this issue.

Our method was experimented on large proofs relying on 
big inductive relations independently defined in the Compcert project.

The current development can be found on-line~\cite{hci:examples},
as well as examples given in Section~\ref{sec:hci}.

Our group recently started another project dedicated
to a certifying compiler from a high-level component-based language dedicated
to embedded systems (BIP), with CompCert C as its target. 
We expect the work presented here and our high-level tactics
to be reused there.

Let us mention another possible application of the technique.
Inversion is sometimes
needed to write a function whose properties will be established later (as
opposed to providing a monolithic and exhaustive Hoare-style specification and
along with a VC generator such as Program). 
In this context simply using the proof engine and the \inversion tactic
tends to generate unmanageably large terms.
We expect our technique to be very helpful in such situations.


%%% Local Variables: 
%%% mode: latex
%%% TeX-master: "itp13"
%%% End: 
