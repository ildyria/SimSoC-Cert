% ----------------------------------------------------------------------
%                   LATEX TEMPLATE FOR PhD THESIS
% ----------------------------------------------------------------------

% based on Harish Bhanderi's PhD/MPhil template, then Uni Cambridge
% http://www-h.eng.cam.ac.uk/help/tpl/textprocessing/ThesisStyle/
% corrected and extended in 2007 by Jakob Suckale, then MPI-CBG PhD programme
% and made available through OpenWetWare.org - the free biology wiki


%: Style file for Latex
% Most style definitions are in the external file PhDthesisPSnPDF.
% In this template package, it can be found in ./Latex/Classes/
\documentclass[twoside,11pt]{PhDthesisPSnPDF}

\usepackage[utf8]{inputenc}
\usepackage[T1]{fontenc}
\usepackage{lmodern}
%\usepackage[english,french]{babel} % already in PhDthesisPSnPDF
%switching can be done with \selectlanguage{english} \selectlanguage{french}

%: Macro file for Latex
% Macros help you summarise frequently repeated Latex commands.
% Here, they are placed in an external file /Latex/Macros/MacroFile1.tex
% An macro that you may use frequently is the figuremacro (see introduction.tex)
%\include{Latex/Macros/MacroFile1}

\usepackage[toc,page]{appendix}
\usepackage{listings}

\theoremstyle{plain}
\newtheorem{thm}{Theorem}[chapter] % reset theorem numbering for each chapter
\newtheorem{lemma}{Lemma}[chapter] % reset lemma numbering for each chapter

\theoremstyle{definition}
\newtheorem{defn}[thm]{Definition} % definition numbers are dependent on theorem numbers
\newtheorem{exmp}[thm]{Example} % same for example numbers

\usepackage{comment}

% Without resume in French
% \excludecomment{resume}
% With resume in French
\includecomment{resume}

%: ----------------------------------------------------------------------
%:                  TITLE PAGE: name, degree,..
% ----------------------------------------------------------------------
% below is to generate the title page with crest and author name

%if output to PDF then put the following in PDF header
\ifpdf
    \pdfinfo { /Title  (PhD and MPhil Thesis Classes)
               /Creator (TeX)
               /Producer (pdfTeX)
               /Author (Xiaomu SHI xiaomu_shi@hotmail.com)
               /CreationDate (D:20120401000000)  %format D:YYYYMMDDhhmmss
               /ModDate (D:YYYYMMDDhhmm)
               /Subject (xyz)
               /Keywords (add, your, keywords, here) }
    \pdfcatalog { /PageMode (/UseOutlines)
                  /OpenAction (fitbh)  }
\fi


%\title{Formalisation and Proofs for an Instruction Set Simulator}



% ----------------------------------------------------------------------
% The section below defines www links/email for author and institutions
% They will appear on the title page of the PDF and can be clicked
\ifpdf
  \author{\href{mailto:xiaomu_shi@hotmail.com}{Xiaomu SHI}}
%  \cityofbirth{born in XYZ} % uncomment this if your university requires this
%  % If city of birth is required, also uncomment 2 sections in PhDthesisPSnPDF
%  % Just search for the "city" and you'll find them.
  \collegeordept{\href{http://http://edmstii.ujf-grenoble.fr}{Mathématiques, Sciences et Technologies de l'Information, Informatique}}
  \university{\href{http://www.doctoralschool.ofGrenoble}{Universit\'e de Grenoble}} %\marginpar{update href}

  % The crest is a graphics file of the logo of your research institution.
  % Place it in ./0_frontmatter/figures and specify the width
  %\crest{\includegraphics[width=4cm]{logo}}

% If you are not creating a PDF then use the following. The default is PDF.
\else
  \author{Shi Xiaomu}
%  \cityofbirth{born in XYZ}
  \collegeordept{CollegeOrDept}
  \university{Universit\'e de Grenoble}
  %\crest{\includegraphics[width=4cm]{logo}}
\fi

%\renewcommand{\submittedtext}{change the default text here if needed}
\degree{Philosophi\ae Doctor (PhD)}
\degreedate{2013 July}


% ----------------------------------------------------------------------

% turn of those nasty overfull and underfull hboxes
\hbadness=10000
\hfuzz=50pt


%: --------------------------------------------------------------
% User-defined macros

\usepackage{extarrows}
\usepackage{soul}
\usepackage{multirow}
\usepackage{listings}
\setstcolor{red}

\setlength{\marginparwidth}{20mm}
\let\oldmarginpar\marginpar
\renewcommand\marginpar[1]%
  {\-\oldmarginpar[\raggedleft\footnotesize\itshape #1]%
                  {\raggedright\footnotesize\itshape #1}}
\newcommand{\byjf}{\textbf{[JF]}}
\newcommand{\jf}[1]{\textcolor{red}{#1}}
\newcommand{\cjf}[1]{\jf{\byjf{} #1}}
\newcommand{\herejf}{\jf{\itshape [Read until here by JF]}}
\newcommand{\margjf}[2]{\textcolor{blue}{$^{[#1]}$}%
                        \marginpar{\textcolor{blue}{%
                           {\vspace*{-0\baselineskip} $^{#1}$\byjf{} #2
      }}}}
\newcommand{\byxm}{\textbf{[XM]}}
\newcommand{\xm}[1]{\textcolor{cyan}{#1}}
\newcommand{\cxm}[1]{\xm{\byxm{} #1}}
\newcommand{\margxm}[2]{\textcolor{cyan}{$^{[#1]}$}%
                        \marginpar{\textcolor{cyan}{%
                           {\vspace*{-0\baselineskip} $^{#1}$\byxm{} #2
      }}}}

\newcommand{\stt}{\small\tt}


\newcommand{\hide}[1]{}



%:-------------------------- Naming macro -----------------------

\newcommand{\hcinv}{\texttt{hc\_inversion}\xspace}
\newcommand{\inversion}{\coqdockw{inversion}\xspace}
\newcommand{\inv}{\coqdockw{inv}\xspace}
\newcommand{\simlight}{\texttt{Simlight}\xspace}
\newcommand{\why}{\texttt{Why}\xspace}
\newcommand{\whyML}{\texttt{WhyML}\xspace}
\newcommand{\whyCert}{\texttt{WhyCert}\xspace}
\newcommand{\framac}{\texttt{Frama-C}\xspace}
\newcommand{\jessie}{\texttt{Jessie}\xspace}
\newcommand{\compcert}{\texttt{CompCert}\xspace}
\newcommand{\clight}{\texttt{Clight}\xspace}
\newcommand{\simsoc}{\texttt{SimSoC}\xspace}
\newcommand{\simsoccert}{\texttt{SimSoC-Cert}\xspace}

% la ligne ci-dessous est à insérer obligatoirement dans le préambule du document avant \begin{document}

\usepackage[a4paper]{meta-donnees}

%: --------------------------------------------------------------
%:                  FRONT MATTER: dedications, abstract,..
% --------------------------------------------------------------

\begin{document}


%%%%%%%%%%%%%%%%%%%%%%%%%%%%%%%%%%%%%%%%%%%%%%%%%%%%%%
%%             Commandes Meta-données               %%
%%   à renseigner par les auteurs pour générer      %%
%%     la couverture modèle Univ. Grenoble          %%
%%%%%%%%%%%%%%%%%%%%%%%%%%%%%%%%%%%%%%%%%%%%%%%%%%%%%%
%%      Fichier encodé au format ISO-8859-16        %%

%\Sethpageshift{???mm}   %%optionnel : à décommenter si besoin pour ajout d'espace afin de center la couvérture horizontalement (valeur par défaut est -5.5mm)
%\Setvpageshift{???mm}   %%optionnel : à décommenter si besoin pour ajout d'espace afin de center la couvérture verticalement (valeur par défaut est -15.5mm)


%\Universite{}    %%optionnel : à décommenter et à renseigenr si vous voulez changer le non d'université
%\Grade{}         %%optionnel : à décommenter et à renseigenr si vous voulez changer le grade
\Specialite{Informatique}
\Arrete{7 août 2006}
\Auteur{Xiaomu SHI}
\Directeur{Jean-Fran\c{c}ois MONIN}
\CoDirecteur{Vania JOLOBOFF}    %%optionnel : à décommenter et à renseigenr si présence d'un Co-directeur de thèse
\Laboratoire{VERIMAG}
\EcoleDoctorale{Math\'{e}matiques, Sciences et Technologies de l'Information, \mbox{Informatique}}
\Titre{Certification of an \\ Instruction Set Simulator}
%\Soustitre{}      %%optionnel : à décommenter et à renseigenr si présence d'un sous-titre de thèse
\Depot{10 juillet 2013}


% Commande pour création de nouvelles catégories dans le jury:

%\UGTNewJuryCategory{...NomDeLaCategorie...}{...Definition...}

% Exemple \UGTNewJuryCategory{UGTFamille}{Membre de la famille} que nous ajoutons dans la commande \Jury ci-dessous sous la forme \UGTFamille{Jean Rousseau}{(...titre_et_affiliation...s'il_y_en_a...)}


\Jury{
%\UGTPresident{...Civilité, Prénom\_et\_Nom...}{...titre\_et\_affiliation...}
%\UGTPresidente{...Civilité, Prénom\_et\_Nom...}{...titre\_et\_affiliation...}

\UGTExaminateur{M. Yves Bertot}{Directeur de Recherche, INRIA Sophia-Antipolis}  
%% 1er examinateur
\UGTRapporteur{Mme Sandrine Blazy}{Professeur, IRISA}
%% 1er rapporteur
\UGTCoDirecteur{M. Vania Joloboff}{Directeur de Recherche, LIAMA}
%% Co-Directeur de thèse s'il y en a
\UGTExaminateur{M. Xavier Leroy}{Directeur de Recherche, INRIA Rocquencourt}     
%% second examinateur
\UGTExaminateur{M. Laurent Maillet-Contoz}{Ingénieur, STMicroelectronics}
%\UGTExaminatrice{...Civilité, Prénom\_et\_Nom...}{...titre\_et\_affiliation...}
\UGTRapporteur{M. Claude Marché}{Directeur de Recherche, INRIA Saclay %
- Île-de-France et LRI}      %% second rapporteur
\UGTDirecteur{M. Jean-François Monin}{Professeur, Université de Grenoble 1 UJF}
%% Directeur de thèse

%% 3ème examinateur
\UGTExaminateur{M. Frédéric Rousseau}{Professeur, Université de Grenoble 1 UJF}
%% 4ème examinateur


%\UGTInvite{...Civilité, Prénom\_et\_Nom...}{...titre\_et\_affiliation...}
%\UGTInvitee{...Civilité, Prénom\_et\_Nom...}{...titre\_et\_affiliation...}
}

\MakeUGthesePDG    %% très important pour générer la couvérture de thèse


%\language{english}

% sets line spacing
\renewcommand\baselinestretch{1.2}
\baselineskip=18pt plus1pt


%: ----------------------- generate cover page ------------------------

%\maketitle  % command to print the title page with above variables


%: ----------------------- cover page back side ------------------------
% Your research institution may require reviewer names, etc.
% This cover back side is required by Dresden Med Fac; uncomment if needed.

% \newpage
% \vspace{10mm}
% 1. Reviewer: Name

% \vspace{10mm}
% 2. Reviewer:

% \vspace{20mm}
% Day of the defense:

% \vspace{20mm}
% \hspace{70mm}Signature from head of PhD committee:



%: ----------------------- abstract ------------------------

% Your institution may have specific regulations if you need an abstract and where it is to be placed in the document. The default here is just after title.

%\include{abstract}

% The original template provides and abstractseparate environment, if your institution requires them to be separate. I think it's easier to print the abstract from the complete thesis by restricting printing to the relevant page.
% \begin{abstractseparate}
%   \input{Abstract/abstract}
% \end{abstractseparate}


%: ----------------------- tie in front matter ------------------------

\frontmatter
%\include{dedication}
%\include{acknowledgement}


%: ----------------------- contents ------------------------

\setcounter{secnumdepth}{3} % organisational level that receives a numbers
\setcounter{tocdepth}{3}    % print table of contents for level 3
\tableofcontents            % print the table of contents
% levels are: 0 - chapter, 1 - section, 2 - subsection, 3 - subsection


%: ----------------------- list of figures/tables ------------------------


\listoffigures	% print list of figures

\listoftables  % print list of tables


%: ----------------------- glossary ------------------------

% Tie in external source file for definitions: /0_frontmatter/glossary.tex
% Glossary entries can also be defined in the main text. See glossary.tex
%\include{glossary}

%\begin{multicols}{2} % \begin{multicols}{#columns}[header text][space]
%\begin{footnotesize} % scriptsize(7) < footnotesize(8) < small (9) < normal (10)

%\printnomenclature[1.5cm] % [] = distance between entry and description
%\label{nom} % target name for links to glossary

%\end{footnotesize}
%\end{multicols}



%: --------------------------------------------------------------
%:                  MAIN DOCUMENT SECTION
% --------------------------------------------------------------

% the main text starts here with the introduction, 1st chapter,...
\mainmatter

%\renewcommand{\chaptername}{} % uncomment to print only "1" not "Chapter 1"


%: ----------------------- subdocuments ------------------------

% Parts of the thesis are included below. Rename the files as required.
% But take care that the paths match. You can also change the order of appearance by moving the include commands.
\chapter*{Acknowledgements\markboth{}{}}

First and foremost, 
I would like to express my sincere gratitude toward 
Prof. Jean-Fran\c{c}ois Monin,
my supervisor, for his guidance and kind support.
He inspired me greatly during writing my thesis.
I sincerely thank Mr. Vania Joloboff
and the whole SimSoC-Cert team
for rendering their help during the period of the project work.
I also wish to give many thanks to my thesis reviewers
Prof. Sandrine Blazy and Mr. Claude March\'{e} for their
patient reading and constructive suggestions,
and to all the other jury members Mr. Yves Bertot, Mr. Xavier Leroy, 
Mr. Laurent Maillet-Contoz, and Prof. Fr\'{e}d\'{e}ric Rousseau,
who kindly accepted to take part in the jury.
It has been a wonderful time and valuable opportunity 
to work in the FORMES group with
people from all over the world sharing their knowledge and ideas.
Special thanks to the Dr. Jianqi LI and the others from Tsinghua University
for giving me helps in daily life and a good working environment.
Last but not least I am grateful to my parents for their supports during
my studies.


\section{Introduction}
\label{sec:intro}

Type-theoretic settings such as Coq \cite{CoqManualV83,BC04,cpdt}
offer two elementary ways of constructing new objects:
functions and inductive types\footnote{%
Co-inductive types are available as well. 
However, this paper does not depend on issues related to finiteness
of computations:
what is said about inductive types holds as well for co-inductive types.
}. 
%\todo{Inductive are used for datatypes and relations, fixpoints for functions.}
%
For instance, even natural numbers can be inductively characterized 
by the following two rules:

\[
\begin{prooftree}
\using {\coqdocvar{E0}}
\justifies\coqdocvar{even\_i}~ 0
\end{prooftree}
\qquad
\begin{prooftree}
\coqdocvar{even\_i}~ n
\using {\coqdocvar{E2}}
\justifies\coqdocvar{even\_i (S (S n))}
\end{prooftree}
\]


%\coqdocinput{chunk1}

\noindent
Rule names such as \coqdocvar{E1} and \coqdocvar{E2}
serve as canonical justifications for \coqdocvar{even\_i}, 
they are called the \emph{constructors} of the inductive definition.

Now, assume a hypothesis $H$ claiming
that \coqdocvar{even\_i (S (S (S x)))} for some natural number $x$.
Then, by looking at the definition of \coqdocvar{even\_i}, 
we see that only \coqdocvar{E2} could justify $H$,
and we can conclude that \coqdocvar{even\_i (S x)}.
Similarly,  \coqdocvar{even\_i} 1 can be considered as an
absurd hypothesis, since, as \coqdocvar{(S 0)} matches neither
0 nor \coqdocvar{(S (S n))}, 
none of the two possible canonical ways of proving \coqdocvar{even\_i},
namely \coqdocvar{E0} and \coqdocvar{E2} can be used.
Such proof steps are called \emph{inversions},
because they use justifications such as \coqdocvar{E0} and \coqdocvar{E2}
in the opposite way, i.e.,
from their conclusion to their premises. 
Note that \coqdocvar{even\_i} 3, \coqdocvar{even\_i} 5, etc. 
do not immediately yield the absurd by inversion.
However, by iterating the first inversion step, we eventually get
\coqdocvar{even\_i} 1 and then the desired result using a last inversion.
This illustrates that inversion is closer to case analysis than to induction.

Indeed, as we will see below, 
inversion can be decomposed in elementary proof steps,
where the key step is a primitive case analysis on the considered
inductive object (the hypothesis $H$, in our previous example). 
However, this decomposition is very often far from trivial because,
in the general case, rules include several premises,
premises and conclusions may have several arguments and
some of these arguments can be shared.
Still, inversion turns out to be extremely useful in practice.
Well-known instances are related to programming languages,
because their semantics is described using complex inductively defined
relations. 

Note that, it may be worth considering a (recursive) \emph{function}
for defining a predicate, rather than an inductive relation.
For instance, in the Coq syntax, an alternative for even
numbers is as follows:

% Now, assume a goal containing a hypothesis $H$ claiming
% that \coqdocvar{even\_i (S x)} for some natural number $x$.
% Then, by looking at the definition of \coqdocvar{even\_i}, 
% we can conclude that $x$ is \coqdocvar{(S y)}
% for some $y$ satisfying \coqdocvar{even\_i y}.

\medskip
\coqdocinput{chunk11}
\medskip

\noindent
Here \coqdocvar{True} denotes a trivially provable proposition,
and \coqdocvar{False} denotes the absurd proposition.
%
Using \coqdocvar{even\_f} is much simpler in the previous situations:
for instance, \coqdocvar{even\_f (S (S (S x)))} just \emph{reduces} to
\coqdocvar{even\_f (S x)} using computation.
In other words, computation provides inversion for free.
Therefore, one may wonder why we should bother with inductively defined
relations.
Two kinds of answers can be given.

One of them is that an inductive definition allows us 
to focus exactly on the relevant values
whereas, with functional definitions, 
we have to deal with the full domain, 
which can be much bigger in general.
In our example above, 
suppose that we want to prove a statement such as
$\forall n, \mathit{even}\:n \impl P\: n$.
We can always attempt an induction on $n$, 
but this strategy enforces to reason on all numbers, 
including odd numbers.
If $\mathit{even}$ is encoded with \coqdocvar{even\_f},
this is no other option.
However, using \coqdocvar{even\_i}, 
we have the additional opportunity to make an induction on 
(the shape of) $\coqdocvar{even\_i}\;n$,
without needing to bother about odd numbers.

Another issue is that it is not always convenient or even possible to
provide a functional definition of a predicate.
Whenever possible,
a $n$-ary relation $R$ on $A_1 \times \ldots A_n$, % with $n \ge 2$,
is advantageously modeled by a function from $A_1, \ldots A_{n-1}$ to $A_n$.
But it requires $R$ to be functional (deterministic) and moreover,
in type-theoretical settings such as CIC, to be total.
If the relation non-deterministic,
we still can try to 
define it by a function returning either \coqdocvar{True}
or \coqdocvar{False}, as is the case for \coqdocvar{even\_f};
this essentially amounts to provide a decision procedure for 
the intended predicate\footnote{
Note that a 1-ary relation $P$ on $A_1$ is isomorphic to a 
binary relation on $\mathbf{1}\times A_1$,
where $\mathbf{1}$ is a type with exactly one inhabitant.
If $P$ holds for at least two values on $A_1$, 
it can be clearly considered as a non-deterministic 
function from $\mathbf{1}$ to $A_1$.
}.
This is not always possible and, even if we can find such an
algorithm, it may be hindered by undesired encoding tricks,
which will induce additional complications in proofs. 
Moreover, a requirement of formal methods expresses that
high-level definitions and statements should be as clear 
as possible in order to be convincing. 
The inductive style is not always the better than the functional
style, but it is often enough the case so that we cannot
ignore it. 
For technical reasons, it is sometimes worth to consider a 
a functional version and an inductive version of the same notion.
Even if the functional version is much better at inversion-like
proof steps, 
the two versions have to be proved equivalent and there,
the need for inverting the inductive version almost inevitably shows up.


All these considerations are especially relevant in the case
of the operational semantics of programming languages,
either in small-step or in big-step style \cite{nielson}. 
Such semantics define transitions between states,
language constructs and,
very often, additional arguments such as input/output events. 
They are inductively defined, 
with at least one rule for each language construct. 
A tutorial example of a toy (but Turing-complete) language 
formally defined in Coq along these lines
is given in \cite{Pierce:SF}
and routinely used as a teaching support in many universities.
A much more involved example
is the semantics of a fairly large subset of C, as defined in 
the Compcert project \cite{Leroy-Compcert-CACM}.

In the SimSoC-cert project \cite{cpp11}, 
we use this semantics to perform proofs of 
an instruction set simulator for ARM,
which is at the heart of SimSoC~\cite{rapido11}, 
a simulator of embedded systems written in C and C++.
Many inversions are needed in our proofs.

\medskip
The practical need for automating inversion has been identified
many years ago.
The first implementations for Coq and LEGO
are analyzed and explained in
\cite{cornes95automating} for Coq
and \cite{McBride96} for LEGO.
Since then, the main tool proposed to the Coq user is
a tactic called \inversion which,
basically performs a case analysis over a given hypothesis
according to its specific specific arguments,
removes absurd cases,
introduces relevant premises in the environment
and performs suitable substitutions in the whole goal.
%
This tactic works remarkably well,
though it fails in seldom intricate cases,
as reported in mailing lists. 
%
However, the price to pay for its generality
is a high complexity of the formal proof-term underlying
an inversion. 
Does it reflect an unnecessarily complex formalization of a 
(at first sight) rather simple idea?
Anyway, 
beyond slowing down the evaluation of scripts which make
an intensive use of this tactic, 
a practical consequence is that
unpleasantly heavy proof terms can unexpectedly occur in
functions defined in interactive mode.

More importantly, in our opinion, using this tactic
introduces many new hypotheses in the environment.
Their names are automatically generated
and sequel of the script depends on them.
Moreover such introduced hypotheses could be inverted again,
and so on.
This poses a problem of robustness which is very serious
in large developments:
updating the inductive relation or
even minor modifications in another part of the development
may result in a complete renaming 
inside a proof script,
which has then to be debugged line by line.
The situation is better in recent version of Coq, 
since \inversion can optionally be given the names of all hypotheses
to be introduced.
Still, their number and contents is hard to predict,
which makes \inversion hardly usable in high-level tactics.

In \cite{small_inv}, 
the first author introduced a technique 
for performing
so-called \emph{small inversions}. 
This technique is rather flexible and is available in several variants.
Our goal was to demystify the magics behind \inversion
and to propose a practical hand-crafted alternative
to this tactic, 
providing much smaller proof terms as well as
a full control of the user on the behavior of inversion.
The idea was illustrated only on very simple examples
and had to be validated on realistic applications. 

We report here such an experiment, 
in the framework of the SimSoC-cert project introduced above.
It turned out that significant changes had to
be made in order to make the initial idea able
to scale up.
%
The contributions presented here are then:
\begin{itemize}
\item an improvement of the main variant from \cite{small_inv},
  which makes 
  it is both simpler to use and more powerful;
\item its illustration on a significant application,
  which involves an intensive use of inversions on 
  big inductive relations coming from the Compcert project.
\end{itemize}

The concrete setting considered here is the Coq proof assistant,
but the technique can be adapted to any proof assistant based
on the Calculus of Inductive constructions or a similar type theory, 
such as LEGO, Matita or Agda.
The rest of the paper is organized as follows.
Section~\ref{sec:absurd}
recalls the principle of small inversions as introduced in \cite{small_inv}.
Section~\ref{sec:improvement} then explains its limitations
and how to overcome them,
while section~\ref{sec:simsoccert} contains a summary
of the application to SimSoC-cert.
We conclude in section~\ref{sec:conclusion} with a comment
on our achievements and some perspectives.




%%% Local Variables: 
%%% mode: latex
%%% TeX-master: "cpp12"
%%% End: 

% \include{related}
% \include{simsoc}
\chapter{Background}
\label{cpt:bg}

\newcommand{\toyl}{\textit{ese}\xspace}

%\jf{Short introduction in 2 or 3 sentences}
This chapter provides a short introduction to the scientific background
of our work:
operational semantics,
to define the meaning of programs;
the Coq proof assistant,
which is used to define the formal model of ARMv6 architecture and to perform
correctness proofs;
finally \compcert, which contains an operational semantics of C formalized in Coq,
and is then the basis of
the formal model that we use for the instruction set simulator \simlight.
We pay a particular attention on underspecified behaviours:
this happens when different compilation strategies may
provide different behaviours for the same program, as is the case for C.
Such issues are illustrated on a very simple toy language, \toyl.
% which \compcert C semantics follows.
% Then a introduction on Coq, which is the language
% used to define the formal model of ARMv6 architecture and to perform
% the correctness proof.
% The last section gives a brief idea on \compcert project and
% which parts we have used to support our correctness proof.

\selectlanguage{french}
\section*{Résumé}

\begin{resume}
Ce chapitre contient une courte introduction au cadre scientifique
dans lequel notre travail a été developpé.
On commence par quelques notions de sémantique opérationnelle,
permettant de définir la signification des programmes.
On présente ensuite l'assistant à la preuve Coq,
que nous avons utilisé pour définir notre modèle formel de l'architecture
ARMv6 et d'effectuer des preuves de correction.
Nous terminons par \compcert,
qui fournit notamment une sémantique opérationnelle de C
formalisée en Coq --
c'est l'ingrédient essentiel que nous utilisons pour produire un modèle formel
du simulateur d'instructions \simlight.

Une attention particulière est portée aux comportement sous-spécifiés :
cela se produit lorsque différentes stratégies de compilation peuvent
aboutir à des comportements différents pour un même programme,
ce qui est le cas avec le langage C.
Pour illustrer ce genre de problèmes,
nous introduisons un langage jouet, \toyl,
contenant des expressions avec effet de bord.
  
\end{resume}

\selectlanguage{english}

\section{Operational Semantics}
\label{sec:opsem}
In computer science, there are three traditional ways to express how programs
perform computations:
% evaluate: % JF: "evaluate" seems correct but I feel more secure with "perform computations"
axiomatic semantics, denotational semantics, and
operational semantics.  Formal semantics are important because it can
give an abstract, mathematical,
and standard interpretation of a programming language.
It helps to understand what a program written in this language does
and to verify what we expect from the program.
In a few words:
\begin{itemize}
\item Denotational semantics constructs mathematical objects which
  describe the meaning of expressions of the language using stateless
  partial \emph{functions}.
  All observably distinct programs have distinct denotations.
\item Operational semantics is more concrete because it is based
  on states. However, in contrast with a low-level implementation,
  operational semantics considers abstract states.
  The behavior of a piece of program corresponds to a transition
  between abstract states.
  This transition relation allows us to define the execution of programs
  by a mathematical computation \emph{relation}.
  This approach is quite convenient for proving the correctness
  of compilers, using operational semantics for the source and target
  languages (and, possibly intermediate languages).
  Operational semantics is used in \compcert to define the execution
  of C programs,
  or more precisely programs in the subset of C considered by the
  \compcert project.
  The work presented in this thesis is based on this approach.
\item Axiomatic semantics describes the effect of programs by
  assertions. A well-known example is Hoare logic.
  It is one of the most popular approaches for proving the
  correctness of programs.
\end{itemize}

% The distinctions between the three broad classes of approaches can
% sometimes be vague, but
% all known approaches to formal semantics use
% the above techniques, or some combination of them.
A good tutorial on programming language semantics is
Benjamin C. Pierce's \emph{Software Foundation}%
\footnote[1]{http://www.cis.upenn.edu/~bcpierce/sf/}.
It is mainly dedicated to operational semantics
and it contains an introduction to Hoare Logic.
The material presented in this tutorial is formalized
in the Coq proof assistant.
Another interesting introduction can be found in \cite{nielson1992semantics}.
It is more detailed than \emph{Software Foundation},
but it is not supported by a proof assistant.


Operational semantics can be used to reliably prove results
on a programming language.
Operational semantics can be presented in two styles.
Small-step semantics,
often known as structural operational semantics,
is used to describe how the single steps of computations evaluate.
The other is big-step semantics, or natural semantics,
which returns the final results of an execution in one big step.
The corresponding transition relation is defined by rules,
according to the syntactic constructs of the language,
in a style which is inspired by natural deduction.

The book \cite{nielson1992semantics} explains
that the choice between small-step semantics
and big-step semantics depends on the objective.
They sometimes can be equivalent.
But in general, they provide different views of the same language % evaluations
and we have to choose an appropriate one for a particular usage.
Moreover, some language constructs can be hard or even impossible to define
with one of these semantics
whereas it is easy with the other style.
In general, when big-step semantics can be used,
it is simpler to manage than small-step semantics.

% \newcommand{\ANum}{\ensuremath{\mathit{A\!Num}\xspace}}
% \newcommand{\APlus}{\ensuremath{\mathit{A\!Plus}\xspace}}
% \newcommand{\AMinus}{\ensuremath{\mathit{A\!Minus}\xspace}}
% \newcommand{\AMult}{\ensuremath{\mathit{A\!Mult}\xspace}}

In order to illustrate some issues on operational semantics
and its different flavors
which are important for us,
let us consider a simple language called \toyl,
for expressions with side-effects.
This language allows us to present some typical issues of C language,
related to the the evaluation order of expressions and statements.
The ISO-C standard that mentions the evaluation order of
expressions with side-effect on the same object is undefined, for example:

\begin{alltt}
  i = ++i + 1;
  a[i++] = i;
\end{alltt}

Several orders are allowed for each of the previous assignments,
because they include two side effects on variable \texttt{i} --
according to ISO-C standard, there are two ``sequence points'' in them.
% before and after the statement.

Other examples are given by Brian Campbell in the
CerCo project~\cite{campbell2012executable},
in order to show that the evaluation order constraints in C
are very lax and not uniform.

\begin{alltt}
  x = i++ && i++;
  x = i++ & i++;
\end{alltt}

Our toy language \toyl is designed to illustrate similar issues.
The constructs of \toyl are:
constants $C~n$, where $n$ is a natural number,
a unique variable $V$,
the addition $P\:\toyl~\toyl$ of two arguments of type \toyl,
and the assignment of the variable with a value expressed by an \toyl.
Its abstract syntax is as follows.

\begin{figure}[h]
% $$a,~b~::=~\ANum~n~|~\APlus~a~b~|~\AMinus~a~b~|~\AMult~a~b$$
$$\toyl~::=~C~n~|~V~|P~\toyl~\toyl~|~A~\toyl~$$
\caption{Syntax of toy language \toyl}
\label{fig:syn}
\end{figure}

% \newcommand{\bsarr}{\ensuremath{\overset{bs}{\rightarrow}}}
% \newcommand{\ssarr}{\ensuremath{\overset{ss}{\rightarrow}}}
\newcommand{\bsarr}{\ensuremath{\xrightarrow{\;bs\;}}}
\newcommand{\ssarr}{\ensuremath{\xrightarrow{\;ss\;}}}

The semantics in big-step style is inductively defined using the following
rules. The parameter $state$ of type natural number
is introduced here to store the current value of $V$.
After an evaluation, a new $state$ is returned. The evaluation takes an initial
state and an expression to compute, and returns a new state and
a natural number which is the evaluation result.
The notation $\bsarr$ means ``evaluates to''.

\medskip
\begin{figure}[h]
\begin{equation}
\frac{}{st,~C~n~\bsarr~st',~n}
\end{equation}
\begin{equation}
\frac{}{st,~V~\bsarr~st,~st}
\end{equation}
\begin{equation}
\frac{st,~e_1~\bsarr~st',~n_1~\qquad st',~e_2\bsarr~st'',n_2}{st,~P~e_1~e_2~\bsarr~st'',~(n_1~+~n_2)}
\end{equation}
\begin{equation}
\frac{st,~e~\bsarr~st',~n_1~\qquad n_1,~V~\bsarr~st'',n_2}{st,~A~e~\bsarr~st'',~n_2}
\label{eq:bsassign}
\end{equation}
%$$\frac{st,~e~\bsarr~st',~n}{st,~A~e~\bsarr~n,~n}$$
\caption{Big-step operational semantics of the toy language \toyl}
\label{fig:bssem}
\end{figure}

%%Add simplified assignment rule
Rule~\ref{eq:bsassign} is for assignment.
A simpler and equivalent version is:
\begin{equation}
\frac{st,~e~\bsarr~st',~n}{st,~A~e~\bsarr~n,~n}
\label{eq:bsassign_simpl}
\end{equation}
The version given in rule~\ref{eq:bsassign}
is closer to
the small-step semantics to be presented later,
which exposes an explicit evaluation order.
To this effect, 
the assignment is split into two parts:
evaluating the right-hand side
then putting the result into the left-hand side.

% Note that an assignment is split into two parts:
% evaluation of the right-hand side
% then putting the result into the left-hand side.

\newcommand{\progtwo}{\ensuremath{P\;V\;(P\;(A\;(C\;1))\;(A\;(C\;2)))}\xspace}
%
For instance, from the state where \texttt{V} contains 0,
the expression in C syntax\\
\mbox{}\hfil\texttt{V + ((V = 1) + (V = 2))}\\
evaluates to 3, with a final state where \texttt{V} contains 2.
This expression is formalized by the term
$\progtwo$,
and the previous statement is formalized by:
$$0, \progtwo \bsarr 2, 3.$$
This statement is proved by systematic applications
of the rules given in Figure~\ref{fig:bssem}.
The proof is driven by the shape of the expression.
Each constructor ($C$, $V$, $P$, $A$) is handled by a specific rule
and leads to premises with smaller expressions (in this language),
which means that
the execution will terminate for any expression.
Moreover, the semantics defined here is deterministic;
the evaluation order is leftmost and
innermost. This is expressed by the following lemma:
%
\begin{lemma}
If $st,~t~\bsarr~st'~v$, and $st,~t~\bsarr~st''~v'$, then $v~=~v'~$ and $st~=~st''$.
\label{lem:dettoyl}
\end{lemma}

Using big-step semantics, we can also describe a non-deterministic
system by
adding one rule for right to left evaluation to offer another evaluation order:
\begin{equation}
\frac{st,~e_2~\bsarr~st',~n_2~\qquad st',~e_1\bsarr~st'',n_1}{st,~P~e_1~e_2~\bsarr~st'',~(n_1~+~n_2)}
\end{equation}

Then the output of the evaluation cannot be predicted:
the same expression can return different states and results.
For instance, we have
\begin{center}
$0, \progtwo \bsarr 2, 3$\\
$0, \progtwo \bsarr 1, 3$\\
$0, \progtwo \bsarr 2, 5$\\
$0, \progtwo \bsarr 1, 4$\\
\end{center}

Next, the following description gives
the small-step operational semantics rules of the same toy language.
This time, the small-step rules take an expression of type \toyl
and the initial state which stores the
current value of variable $V$, and return the reduced expression
and the new state.
The symbol $\ssarr$ means ``reduces to in one small step''.

\medskip
\begin{figure}[h]
\begin{equation}
\frac{}{V,~st~\ssarr~(C~st),~st}
\label{eq:ssV}
\end{equation}
\begin{equation}
\frac{}{(P~(C~n_1)~(C~n_2)),~st~\ssarr~(C~(n_1~+~n_2)),~st}
\label{eq:ssCC}
\end{equation}
\begin{equation}
\frac{e_1,~st~\ssarr~e_1',~st'}{(P~e_1~e_2),~st~\ssarr~(P~e_1'~e_2),~st'}
\label{eq:ssP}
\end{equation}
\begin{equation}
\frac{e_2,~st~\ssarr~e_2',~st'}{(P~(C~n_1)~e_2),~st~\ssarr~(P~(C~n_1)~e_2'),~st'}
\label{eq:ssPC}
\end{equation}
\begin{equation}
\frac{}{(A~(C~n)),~st~\ssarr~V,~n}
\label{eq:ssAC}
\end{equation}
\begin{equation}
\frac{e,~st~\ssarr~e',~st'}{(A~e),~st~\ssarr~(A~e'),~st'}
\label{eq:ssA}
\end{equation}
\caption{Small-step operational semantics of the toy language \toyl}
\label{fig:sssem}
\end{figure}

In small-step semantics,
two rules (\eqref{eq:ssP} and \eqref{eq:ssPC}) are needed
to define the leftmost and innermost evaluation order.
And there is no rule for reducing a single constant.
From the number of rules, we see that the definition of
deterministic computations with a given evaluation order
is more complex
with small-step operational semantics
than with big-step semantics.
%The rules above give a deterministic semantics.

We can also have a non-deterministic small-step semantics
by modifying one of the rules of the plus operation
to remove the leftmost and innermost order:
changing rule \eqref{eq:ssPC} in Figure~\ref{fig:sssem} into:
\begin{equation}
\frac{e_2,~st~\ssarr~e_2',~st'}{(P~e_1~e_2),~st~\ssarr~(P~e_1'~e_2),~st'}
\end{equation}

Considering the set of possible executions allowed by the
non-deterministic semantics, we have more
results by using small-step semantics than using big-step semantics.
Taking the same example \progtwo as above,
the possible executions in small-step semantics are:

\begin{center}
$0, \progtwo \ssarr 3, 1$\\
$0, \progtwo \ssarr 3, 2$\\
$0, \progtwo \ssarr 4, 1$\\
$0, \progtwo \ssarr 5, 2$\\
$0, \progtwo \ssarr 6, 2$\\
\end{center}

%\jf{The last result is obtained by performing [COMPLETE]}
The last result is obtained by performing the assignment $A~(C~1)$,
then the assignment $A~(C~2)$; at this point, the value stored in the state
equals $2$. Next, performing plus in any order will compute the
result of 6 and the state still stores $2$.
On the other hand,
the big-step semantics fails to express that 6 can be returned.

In contrast with big-step semantics,
the sequence corresponding to an assignment
(evaluation the right-hand side,
then putting the result into the left-hand side)
can actually be interrupted
when we consider small-step semantics,
and the evaluation of another sub-expression can then occur.

In general,
big-step semantics is not the right approach for dealing with non-deterministic
executions or under-specified semantics,
because it is not able to cover all the possible execution cases.

Note that \compcert includes a big-step deterministic semantics
and a small-step non-deterministic semantics for \compcert C.

% \jf{add executions of the example above, showing we have more results than bs.
% conclude bs is not the right approach for non-deterministic executions,
% under-specified semantics. For instance, \compcert C includes bs determinist semantics,
% and ss non-deterministic semantics}

% \hide{
% The future of program verification is to connect machine-verified
% source programs to machine-verified compilers, and run the object
% code on machine-verified hardware. To connect the verifications end to
% end, the source language should be specified as a structural
% operational semantics represented in a log- ical framework; the target
% architecture can also be specified that way. Proofs of source code can
% be done in the logical framework, or by other tools whose soundness is
% proved w.r.t. the operational semantics specification; these may be in
% safety proofs via type-checking, correctness proofs via Hoare Logic,
% or (in source languages designed for the purpose) correctness proofs
% by a more expressive proof theory.  The compiler -- if it is an
% optimizing compiler -- will be a stack of phases, each with a well
% specified operational semantics of its own. There will be proofs of
% (partial) correctness of each compiler phase, or witness-driven
% recognizers for correct compilations, w.r.t. the operational
% semantics’s that are inputs and outputs to the phases.
% }

\section{Coq}

\subsection{Short introduction}

%\jf{Some details can be given for a non-specialist.}
%
Coq\cite{coqart} is an interactive theorem prover,
implemented in OCaml.
It allows the expression of mathematical assertions,
mechanically checks proofs of these assertions, helps to find formal proofs,
and extracts a certified program from the constructive proof
of its formal specification.
% Coq works within the theory of the calculus of inductive constructions,
% a derivative of the calculus of constructions.
% Coq implements a dependently typed functional programming language.
Coq can also be presented as a dependently typed $\lambda$-calculus
(or functional language).
Here we just illustrate the syntax on simple examples.
For a detailed presentation,
the reader can consult \cite{coqmanual} or \cite{coqart}.
%\jf{Use appropriate fonts.}
\begin{itemize}
\item $fun~(n:nat)~\Rightarrow ~n$ is the identity function on natural numbers;
  its type is written as $nat~\rightarrow ~nat$. Function application is not written
  as $f(x)$ but $f\; x$, or $(f\; x)$ if grouping is needed.
  With several arguments, the syntax is $f~x~y$ or $(f~x~y)$ instead of $f(x,~y)$.
\item We can write definitions as follows:

  $${\tt Definition~idn~:=~fun~(n:~nat)~\Rightarrow ~n.}$$

  An equivalent and more common syntax for this definition is:

  $${\tt Definition~idn~(n:~nat)~:=~n.}$$

  For instance, the application of $idn$ to 3 is written $(idn 3)$ and
  this term reduces to 3.
\item $fun~(X:~Type)~(n:~X)~\Rightarrow ~n$ is the \emph{polymorphic}
  identity function on an arbitrary type $X$;
  its type is written $\forall ~X:~Type,~X~\rightarrow ~X$.

  $${\tt Definition~id~(X:~Type)~(n:~X)~:=~n.}$$

  Note that it expects 2 arguments, for instance, we can write $(id~nat~3)$.
  Like most of functional programming languages, Coq can also perform type inference.
  If we define $id$ as following:

  $${\tt Definition~id~\{X:~Type\}~(n:~X)~:=~n.}$$

  The application can be just written as $id~3$. Coq can get the explicit $X$ from
  the type of $3$.
\item
 % \jf{Give an example of a function with 1st order dependent types.
 %  Could be $\forall n:nat, n> 0 \rightarrow nat$; do NOT put all the code
 %  o the fct, just a squeletton }
   A dependent type is a type that depends on a value.
   It is very flexible to use, as to refine the type of a function without
   including the whole specification.
   A very simple example is to define a predecessor with only the rule for
   case 0:

   $${\tt \forall ~n:~nat,~n~>~0~\rightarrow ~nat}$$

\item Coq also includes inductive types, as explained in the next subsection.
\end{itemize}

A proof term of type
$\forall~n:~nat,~P\, n~\rightarrow Q\, n$ is $fun~(n:nat),~P~n~\rightarrow ~Q~n$
is a function which takes a natural number $n$ and a proof of $P~n$ as arguments
and returns $Q~n$.
In general, proofs are functions and checking the correctness of a proof
boils down to type-checking.
% \jf{Explain that proofs are functions. E.g., a proof of
%  $\forall n:nat, P\, n \rightarrow Q\, n $ is a function
% which takes as arguments a natural number $n$,
% a proof of $P \,n$ and returns a proof of $Q\,n$.}

Coq is not an automated theorem prover:
the logic supported by Coq (CiC\footnote{%
Calculus of Inductive Constructions.}%
) includes arithmetic;
therefore it is too rich to be decidable.
However, type-checking (in particular, checking the correctness of a proof) is decidable.
As full automation is not possible for finding proofs, human interaction is essential.
The latter is realized by \emph{scripts},
which are sequences of commands for building a proof step by step.
Coq also provides
built-in tactics implementing various decision procedures
for suitable fragments of CiC
and a language called \texttt{Ltac} which can be used for
automating the search of proofs and shortening scripts.

%\jf{Which can be used for subgoals which are in a decidable fragment of CiC.}
% \jf{Give examples} % Well, no need because you don't use this feature

\subsection{Inductive definitions}

\newcommand{\toylvar}[1]{\texttt{#1}}

To make a better illustration, we use the same toy language \toyl~\ref{fig:syn}
as in the previous section.
Here we show how to inductively define its syntax and its big-step
operational semantics:

\begin{alltt}
Inductive tm : Type :=
  | C : nat -> tm      (* constant *)
  | V : tm             (* the unique variable *)
  | P : tm -> tm -> tm (* plus *)
  | A : tm -> tm       (* assignment *)
\end{alltt}

%\renewcommand{\coqdocvar}[1]{\texttt{#1}}
%
An inductive definition can handle recursive specifications of types;
it defines how it is constructed.
The type \toylvar{tm} is the type of
the toy language \toyl which can be
a constant (constructor \toylvar{C} associated with a natural number of type
\toylvar{nat}), a (unique) variable (constructor \toylvar{V}),
or one of the following two operations: plus
(two expressions of type \toylvar{tm} connected by the constructor \toylvar{P})
or assignment
(constructor \toylvar{A} with an expression of type \toylvar{tm} as input)

Then the inductive definition below gives the annotated inductive type
to describe the deterministic evaluation relation of the corresponding \toyl in
big-step style.
The type of the evaluation \toylvar{eval} is a relation, describing the
transition from
an input expression \toylvar{tm} and a state
to a new state and an evaluation result of type \toylvar{nat},
a natural number.
Each clause is defined according to a rule in Figure~\ref{fig:bssem}.

\begin{alltt}
Inductive eval : state -> tm -> state -> nat -> Prop :=
  | E_Const : forall s n,
      eval s (C n) s n
  | E_Var : forall s,
      eval s V s s
  | E_Plus : forall s t1 n1 s' t2 n2 s'',
      eval s t1 s' n1 ->
      eval s' t2 s'' n2 ->
      eval s (P t1 t2) s'' (n1 + n2)
  | E_Assign : forall s s' s'' t n1 n2,
      eval s t s' n1 ->
      eval n1 V s'' n2 ->
      eval s (A t) s'' n2.
\end{alltt}

%\begin{alltt}
%Inductive eval : state -> tm -> state -> nat -> Prop :=
%  | E_Const : forall s n,
%      eval s (C n) s n
%  | E_Var : forall s,
%      eval s V s s
%  | E_Plus : forall s t1 n1 s' t2 n2 s'',
%      eval s t1 s' n1 ->
%      eval s' t2 s'' n2 ->
%      eval s (P t1 t2) s'' (n1 + n2)
%  | E_Assign : forall s s' t n,
%      eval s t s' n ->
%      eval s (A t) n n
%\end{alltt}

\subsection{Proofs and tactics}

In order to show a concrete
proof using Coq proof assistant, we recall Lemma~\ref{lem:dettoyl}
which claims the big-step operational semantics of \toyl is deterministic.
we first formalize the corresponding statement as follows:
%assertion named \coqdocvar{aevalR\_deterministic}:

\begin{alltt}
Lemma eval_deterministic:
  forall st t st' st'' v v',
  eval st t st' v ->
  eval st t st'' v' ->
  (v = v') \(\wedge\) (st' = st'').
\end{alltt}

It states that,
with the same initial state \toylvar{st} and expression \toylvar{t},
evaluating the big-step semantics defined in Figure~\ref{fig:bssem}
will return the same results and the same new states.
Then we use Coq in an interactive way to verify this statement.
The general idea is to make an induction on \toylvar{eval st t st' v},
name as hypothesis \toylvar{ev1}.
According to the rules in the inductive definition of \toylvar{eval},
there are four cases to consider. Under each case of \toylvar{ev1},
we also have to consider the corresponding derivation of hypothesis \toylvar{ev2}
of type \toylvar{eval st t st'' v'}.
The proof script contains a sequence of tactics to lead Coq to
perform all these steps,
checking the correctness of the claims we made.
Here is a short explanation on some basic and frequently used tactics:
\begin{itemize}
\item
  \coqdockw{intros} moves the quantifiers
  and hypotheses from the goal to the context of assumptions.
\item
  \coqdockw{induction} does case analysis for inductively defined types.
  Induction hypotheses are automatically put into context.
\item
  \coqdockw{inversion} derives the constraints on variables according to the inductive
  definition corresponding to the hypothesis that is inverted.
\item
  \coqdockw{reflexivity} checks that the left-hand side and the right-hand side of
  an equational goal are convertible.
\item
  \coqdockw{rewrite} performs replacement according to an equational hypothesis.
\end{itemize}

The following code from \coqdockw{Proof} to \coqdockw{Qed}
provides a formal proof of the determinism of big-step semantics of \toyl stated above.
\begin{alltt}
Proof.
  intros until v'; intros ev1 ev2.
  generalize dependent v'.
  generalize dependent st''.
  induction ev1.
    (*Case "C"*)
    intros;
    inversion ev2; subst; split; try reflexivity.
    (*Case "V"*)
    intros;
    inversion ev2; subst; split; try reflexivity.
    (*Case "P"*)
    intros;
    inversion ev2; subst; split;
    apply IHev1_1 in H2; destruct H2 as [Heqn1 Heqst1];
    rewrite Heqst1 in IHev1_2;
    apply IHev1_2 in H5; destruct H5 as [Heqn2 Heqst2];
    [rewrite Heqn1; rewrite Heqn2; reflexivity | exact Heqst2].
    (*Case "A"*)
    intros;
    inversion ev2; subst;
    apply IHev1 in H1;
    destruct H1; rewrite H; split; reflexivity.
Qed.
\end{alltt}

\subsection{Interactive proof assistant vs automated theorem prover}

An interactive proof assistant, such as Coq, requires man-machine
collaboration to develop a formal proof.
Human input is needed to create appropriate auxiliary defintions,
choose the right inductive property and, more generally,
to define the architecture of the proof.
Automation is used for non-creative proof steps
and checking the correcntess of the resulting formal proof.
A rich logic can be handled in an interactive proof assistant
for a variety of problems.

On the other hand, fully automated theorem provers were developed.
They can perform the proof tasks automatically, that is,
without additional human input.
%then less human work is required to write interactive proving tactics.
%It is sure that the
Automated theorem prover can be efficient in some cases.
But problems appear to be inevitable:
% JF: the real argument is below, this one looks strange.
% except first-order logic other logic such as
% higher-order logic, theorem proving for them is not well implemented.
%There also exists  the decidability problem in automate theorem prover.
if we are able to automatically prove a formula,
it means that it belongs to a decidable (or at least semi-decidable) class of problems.
%The underlying logics need to be decidable.
It is well-known that decidable logics are much less powerful, expressive
and convenient than higher-order logic.
Then the range of problems we can model with an automated theorem prover is smaller
than with an interactive proof assistant.
In practice, both approaches are important
in the fields of computer science and mathematical logic.
Here in our project, a rich logical system is needed,
in order to manage the complexity of the specification and of the proofs.
% JF: not relevant here.
% We choose to use the interactive proof assistant Coq.
% More comparisons of different technologies are in Section~\ref{sec:cersimsoc}.

\subsection{Applications}

Georges Gonthier (of Microsoft Research, in Cambridge, England) and
Benjamin Werner (of INRIA) used Coq to create a surveyable proof of
the four color theorem, which was completed in September, 2004~\cite{gonthier2008formal}
Based on this work, a significant extension to Coq was developed,
which is called Ssreflect (which stands for ``small scale reflection''). Despite
the name, most of the new features added to Coq by Ssreflect are
general purpose features, which is useful not merely for the computational
reflection style of proof.

The same technology was then used for the formal verification
of an important result from finite group theory, the ``odd theorem''.
A simplified proof has been published in two books: (Bender \&
Glauberman 1995), which covers everything except the character theory,
and (Peterfalvi 2000, part I) which covers the character theory. This
revised proof is still very hard, and is longer than the original
proof, but is written in a more leisurely style.  A fully formal
proof, checked with the Coq proof assistant, was announced in
September, 2012 by Georges Gonthier and fellow researchers at Microsoft
Research and INRIA.\cite{gonthier2013engineering}

\compcert\cite{ccc} is a formally verified optimizing compiler for a subset of
the C programming language which currently targets PowerPC, ARM and
32-bit x86 architectures.
% This project, led by Xavier Leroy,
% started officially in 2005, funded by the French institutes ANR and
% INRIA.
The compiler is specified, programmed, and proved in Coq. It
aims to be used for programming embedded systems requiring
reliability. The performance of its generated code is often close to
that of gcc (version 3) at optimization level O1, and is always better
than that of gcc without optimizations.

\section{CompCert}
\label{sec:compcert}

In a previous section (Sec~\ref{sec:gi}),
we mentioned that we use results of the \compcert project in order
to link the formal representation of ARMv6 architecture with the C representation
of this architecture in \simlight.
Now we introduce \compcert in more detail.
\compcert is a verified compiler for the C programming language provided by INRIA
\cite{ccc}.
It has a long translation chain of eleven steps, from C source code into assembly
code. Every internal representation has its own syntax and semantics defined in
Coq. % done by \compcert group.
It is formally verified in the following sense:
the produced assembly code is proved to behave exactly the same as the input C program,
according to a formally defined operational semantics of these languages.

The 2 first languages considered in the \compcert translation chain,
\compcert C and \clight, are actually subsets of the C language.
Like C, \compcert C is non-deterministic: for some expressions cause side-effect
and have more than one evaluation order.
On the other hand, all expressions of \clight are pure.
Assignments and function calls in \clight are treated not as statements but as expressions.
The reason why we choose \compcert C rather than \clight to represent \simlight
is that
% it is the closest representation to C language comparing to the next languages,
% such as \clight, which is also a subset of C and a simplified version of \Compcert C.
it is much more user-friendly and convenient.
% \margjf{1}{Don't agree, see next comment}%
% In order to take every possible execution state into account and the
% compatibility with \simlight (the c code will be implemented in the \simlight
% simulator), \compcert C is used instead of \clight.
% If we take \clight to be the language representing \simlight, the code to be
% considered is much less readable and less convenient to generate (refer to
% the translation of our own, not to use \compcert compiler).
Indeed, as \clight expressions are pure and deterministic,
a number of auxiliary variables have to be introduced in order
to manage intermediate states.

Here we present a small example of a C program to illustrate the last point.
The original C code is as:
% \begin{alltt}
% void main(int a, int b, int c)
% {
%   a = f(a + b, b + 1, c);
%   b = f(a, b, c);
%   c = b;
% }
% \end{alltt}
\begin{alltt}
void main(int x, int y)
\{
  int a;
  int b;
  int v;
  a = f(f1(v, f2(x, y)), f3(a, 1), f4(b, 3));
\}
\end{alltt}
All the function calls (\texttt{fx}) are side-effect free operations.
Then using \compcert compiler, we are able to generate the \compcert C
and \clight representations.
The \compcert representation is exactly the same as the original C
code in this case. But the \clight representation is quite different,
with the introduction of additional temporary variables
(which are different from local variables, they do not reside in memory).
% \begin{alltt}
% void main(int a, int b, int c)
% \{
%   register int $2;
%   register int $1;
%   $1 = f(a + b, b + 1, c);
%   a = $1;
%   $2 = f(a, b, c);
%   b = $2;
%   c = b;
% \}
% \end{alltt}
\begin{alltt}
void main(int x, int y)
\{
  int a;
  int b;
  int v;
  register int $5;
  register int $4;
  register int $3;
  register int $2;
  register int $1;
  $1 = f2(x, y);
  $2 = f1(v, $1);
  $3 = f3(a, 1);
  $4 = f4(b, 3);
  $5 = f($2, $3, $4);
  a = $5;
\}
\end{alltt}
%$
The proof based on these two representations can be expected to
have the same complexity, because the complexity of the proof
work is caused by the C memory model. Using either of them will face
the same memory model (this will be detailed in
Chapter~\ref{cpt:correct}).
The transition corresponding to the evaluation of a given high-level
expression (as the one given above) will anyway be decomposed in smaller
transitions,
either if we use the more complicated semantics of \compcert C
on the original shorter expression,
or if we use the simpler semantics of \clight
on the corresponding longer \clight expression.
Therefore, we don't expect a real gain in using \clight rather than \compcert C
at the proof stage,
while we would lose readability and convenience in the C code.

% \jf{Disagree because we consider only one execution anyway.
% The point is more that if \clight is the target, the code to be
% considered is much less readable un less convenient to generate.
% Indeed, as \clight expressions are pure and deterministic,
% a number of auxiliary variables have to be introduced in order
% to manage intermediate states.
% [Here you should add a simple example.]
% About the work required in proofs, we can expect a similar complexity.
% The transition corresponding to the evaluation of a given high-level
% expression (as the one given above) will anyway be decomposed in smaller
% transitions,
% either if we use the more complicated semantics of \compcert-C
% on the original shorter expression,
% or if we use the simpler semantics of \clight
% on the corresponding longer \clight expression.
% Therefore, we don't expect a real gain in using \clight rather than \compcert-C
% at the proof stage,
% while we would lose readability and convenience in the C code.
% }

In \simsoccert, we use two parts of \compcert C.
The first is the \compcert basic library. It defines data types for words, half-words,
bytes etc., and bitwise operations and lemmas to describe their properties.
In our Coq model, we also use these low level representations
and operations to describe the ISS (Instruction Set Simulator) model.
The second is the \compcert C language (its syntax and semantics),
from which we get a formal model of \simlight.
In our correctness proofs, wa can then analyze its behaviour step by step
and compare it with our Coq model of ARM.

% Because these independency from \compcert project, we have to update our project
% due to the version changes in \compcert. Reporting the changes from \compcert
% version 1.9 to the latest 1.11, there are three big parts we have to mention:
% \begin{itemize}
% \item
% The library for float number has been changed entirely. Althought we don't use
% float number in \simsoccert, the basic inductively defined type \coqdocvar{type}
% has one constructor \coqdocvar{Tfloat} which has parameter of type
% \coqdocvar{float}. Then the compilation option has to include this library path,
% and it has also to be added in the Coq \coqdockw{loadpath}.
% \item
% The type of memory model is
% \end{itemize}

\subsection{CompCert library}
\label{ssec:cclib}
%\newcommand{\int}{}

In \compcert, a reusable basic library
on machine integers (type \texttt{int}) and bitwise operations
is formally defined in Coq.
The type \texttt{int} is based on type \texttt{Z} from the Coq standard library, with
a proof to guarantee that the range of the value is between 0 and the modulus.
Parameterized by \texttt{wordsize} of type \texttt{nat} (natural number),
the integer module can be instantiated to \texttt{byte}, \texttt{int64}, and so on.
This module also supports the conversion between the types \texttt{int}, \texttt{Z},
and \texttt{nat}.

Applicative finite maps are the main data structure used in
the memory state and the global/local environment descriptions.
There are two basic types, a \texttt{Tree} and a \texttt{Map},
from which a number of maps and trees can be derived.
The difference between the two is:
for \texttt{Tree} the result of the $\langle get\rangle$ operation is an option type:
if there is no data associated with the key, \texttt{None} is returned.
For type \texttt{Map}, $\langle get\rangle$ always returns a data.
If there is no data associated, a default value will be returned, which is
given at initialization time.
These two data structures are based on the abstract signature radix-2 search tree.
And the derived trees and maps are named by their keys which can be integer or positive.
The \texttt{Tree} is used to define the global and the local environments,
which gather memory information,
and map the reference identifier to data information.
Since the environment corresponds to a memory contents,
no information can be obtained if a nonexistent address is given.
On the contrary, the memory contents is represented
by a \texttt{Map} indexed by an integer.
If a block in memory has not been allocated,
it should return a default value \texttt{Undefined} by any visit.

%Applicative finite maps are the main data structure used in this
%  project.  A finite map associates data to keys.  The two main operations
%  are [set k d m], which returns a map identical to [m] except that [d]
%  is associated to [k], and [get k m] which returns the data associated
%  to key [k] in map [m].  In this library, we distinguish two kinds of maps:
%- Trees: the [get] operation returns an option type, either [None]
%  if no data is associated to the key, or [Some d] otherwise.
%- Maps: the [get] operation always returns a data.  If no data was explicitly
%  associated with the key, a default data provided at map initialisation time
%  is returned.
%
%  In this library, we provide efficient implementations of trees and
%  maps whose keys range over the type [positive] of binary positive
%  integers or any type that can be injected into [positive].  The
%  implementation is based on radix-2 search trees (uncompressed
%  Patricia trees) and guarantees logarithmic-time operations.  An
%  inefficient implementation of maps as functions is also provided.



\subsection{CompCert C semantics}

\label{ssec:ccc}

% \jf{Explain later in this subsection (at the end?)
% what is the impact of these restrictions on \simlight:
% unavailable features can be (and are) completely avoided?
% If some of them are used: which ones?
% how is it managed in proofs?
% If some answers are too much detailed here, or need
% maaterial to be introduced later,
% just mention the point and refer to the place
% where it will be dealt with}.

\compcert C is a large subset of C language.
Here are some limitations in this subset.
\begin{itemize}
\item Types: most of the types in C90 \cite{C90} are supported,
  except the following points.
  \begin{enumerate}
  \item Unprototyped function type $(int f())$
    and function type with variable number of arguments $(int f(...))$.
    But it is possible to declare (not define) an external function of
    the latter.
  \item A structure can not have an unknown sized array type as the last
    element. The size information must be known.
  \end{enumerate}
\item Wide char and wide string.
\item Type cast does not support pointer to float.
%\item The in-memory representation of pointer is opaque, can only be examined by
%  a 32bits word.
\item Specify bit fields in unions are not supported.
%\item In-line assembly code is not supported.
\item For the switch statement, \texttt{case} and \texttt{default} must appear.
  And the \texttt{default} must occur at last.
\item The only available external functions are \texttt{printf}, \texttt{malloc}, \texttt{free},
  \texttt{\_\_builtin\_annot} and \texttt{\_\_bultin\_annot\_val}. The other external functions can be
declared but not implemented. One external function will generate an event trace.
It says the result of the external function is computed by operating system,
not the \compcert C code.
\item Every program must have a \texttt{main} function declared.
\end{itemize}


%%% BEGIN moved to Chapter 6
% % XM
% % When we want to obtain the \compcert C representation of ARMv6 model,
% % there are two ways. First, use the \compcert C provided converter to generate \compcert
% % C code from \simlight C file, which is not a verified translation step in \compcert.
% % JF -> XM : warning, what you wrote was misunderstood
% % JF
% The \compcert C representation of ARMv6 can be presented in two ways:
% in textual form or using an AST (Abstract Syntax Tree).

% Two options can then be considered.
% The first is to use the converter provided by \compcert C to generate \compcert
% C code from \simlight C file, which is not a verified translation step in \compcert.
% Or, we translate from the ARM internal representation
% AST to \compcert C AST.
% If we use the first method, the unsupported things above
% will not be controlled. The generated code may lose information without warning.
% So the second transformation is in use.
% The second transformation also has weakness. The ARM internal AST only contains the
% ISS model. The function body of library functions is not included,
% which means the generated code has no definition of these library functions.
% We have to add them manually, or improve the feature of the transformation by
% invoking the former.
% Whichever transformation we choose, we have to give a fake main function.
% This is because our correctness proof takes each instruction operation as one program.
% To build a \compcert C program, we must have the main entry point in the global
% environment.
%%% END moved to Chapter 6

%JF: I'm HERE

% XM
% The semantics definition has two aspects.
% JF
\compcert provides two operational semantics for \compcert C:
% XM
% one is in small-step strategy; the other is in big-step.
% JF
one is non-deterministic, in small-step style;
the other is detailed, in big-step style.
In our case, the big-step semantics is enough for correctness proofs
% JF -> XM : the previous sentence is not enough. We already duscussed on that point.
% Have no time to write smtg here, anyway you should think about it again
% and maybe add a small paragraph later (before or after submmission).
% Remove the margf after reading this.
% Well, look at "important remark" at the end of this file...

% JF: next sentence repeated below
% The semantics is described as a transition system on the memory model.
% JF: said earlier -> removed
% The evaluation Statement and expression is deterministic.
% Said in 2.1
% The evaluation is represented using relations for a better proof induction.
%%Coq had better support for proof induction over relations than over function definitions
The formal operational semantics is described as a transition system
on memory states written as follows:
$$G,E~\vdash ~\langle\textrm{expression}\rangle,~M~\xLongrightarrow{t}~v,~M'.$$
Here $G$ represents the global environment of the whole program;
$E$ is the local environment;
$M$ and $M'$ are memory states and $t$ is a trace of I/O events;
$v$ is a returned value.

In \compcert C, expressions can be categorized into 15 cases,
% XM
% and we have 13 of them that are used in our correctness proofs.
% JF
13 of them are used in our correctness proofs.
Some of them are similar to the ones for \clight and are already listed
in \compcert papers \cite{lerbla08}.
Inference rules that are different from \clight are presented
in Fig.~\ref{fig:evalexpr}.
\begin{figure}
\begin{minipage}[b]{1\linewidth}
\centering
%\small

%  eval_call: forall e m rf rargs ty t1 m1 rf' t2 m2 rargs' vf vargs
%                       targs tres fd t3 m3 vres,
%       eval_expr e m RV rf t1 m1 rf' -> eval_exprlist e m1 rargs t2 m2 rargs' ->
%       eval_simple_rvalue ge e m2 rf' vf ->
%       eval_simple_list ge e m2 rargs' targs vargs ->
%       classify_fun (typeof rf) = fun_case_f targs tres ->
%       Genv.find_funct ge vf = Some fd ->
%       type_of_fundef fd = Tfunction targs tres ->
%       eval_funcall m2 fd vargs t3 m3 vres ->
%       eval_expr e m RV (Ecall rf rargs ty) (t1**t2**t3) m3 (Eval vres ty)
\begin{equation}
\label{eq:funcall}
\frac
{\begin{array}{c}
G,E\vdash rf~M~\xLongrightarrow{t1}~rf',M1\qquad
G,E\vdash rarg^*~M1~\xLongrightarrow{t2}~rarg'^*,M2\\
G,E\vdash M2~rf' \Rightarrow vf\qquad
\texttt{find\_funct}~(G,vf)~=~\lfloor fd\rfloor\\
\vdash M2~fd~varg^* \xLongrightarrow{t3}vres,M3
\end{array}
}
{G,E\vdash M~\langle\textrm{\texttt{Call}}\rangle\xLongrightarrow{t1**t2**t3}vres,M3}
\end{equation}
\end{minipage}

\begin{minipage}[b]{1\linewidth}
\centering
   % eval_assign: forall e m l r ty t1 m1 l' t2 m2 r' b ofs v v' t3 m3,
   %    eval_expr e m LV l t1 m1 l' -> eval_expr e m1 RV r t2 m2 r' ->
   %    eval_simple_lvalue ge e m2 l' b ofs ->
   %    eval_simple_rvalue ge e m2 r' v ->
   %    sem_cast v (typeof r) (typeof l) = Some v' ->
   %    assign_loc ge (typeof l) m2 b ofs v' t3 m3 ->
   %    ty = typeof l ->
   %    eval_expr e m RV (Eassign l r ty) (t1**t2**t3) m3 (Eval v' ty)
\begin{equation}
\frac
{\begin{array}{c}
G,E\vdash l~M\xLongrightarrow{t1}l',M1\qquad
G,E\vdash r~M1\xLongrightarrow{t2}r',M2\\
G,E\vdash l'~M2\Rightarrow (b, ofs)\qquad
G,E\vdash r'~M2\Rightarrow v\\
cast(v,typeof(l),typeof(r))=~\lfloor v'\rfloor\\
store (G,~typeof(l),~M2,~(b,ofs),~v)=~\lfloor M3\rfloor\\
\end{array}
}
{G,E\vdash (l=r)~M\xLongrightarrow{t1**t2**t3}v',M3}
\end{equation}
\end{minipage}
\caption{Some rules for \compcert C operational semantics}
\label{fig:evalexpr}
\end{figure}

The first rule in Fig~\ref{fig:evalexpr} is for evaluating a function call.
The evaluation is quite different from the rule for \clight.
Not only that \clight expressions are side-effect free,
but \compcert C separates memory state
transformation from evaluating simple expressions in order to preserve memory
state.
A function call can be evaluated in three steps:
evaluating the function referenced by identifier \texttt{rf} to get where it
is stored;
evaluating the function arguments \texttt{rargs} to get their values;
finding the function definition \texttt{fd} in the environment;
then evaluating the function call using \texttt{eval\_funcall}.

The second rule in Fig~\ref{fig:evalexpr} is the evaluation of an assignment.
In \clight, an assignment is not an expression but a statement
because \clight expressions are pure.

In \simlight, the interpretation uses a subset of C features
which is as simple as possible.
% JF -> XM: important remark
This is not only to satisfy to \compcert C restrictions,
but also to avoid ambiguous situations where an expression could
have different behaviours.
This way, the bigstep semantics of \compcert C is sufficient.
% JF -> XM 2 vrbs in the next sentence, -> cannot understand
% in order to preserve most of the C representations
% would be used to perform proofs.
% XM
% As the \compcert C restrictions being introduced in the beginning of this section,
% there are still some of them can not be avoided in \simlight.
% JF
However, some features outside of \compcert C
occur in the current version of \simlight:
external functions, which are used in many places
to perform I/O subsystem communications.
Currently, those external functions are represented by axioms.
As a future improvement,
it will be better to use internal functions instead.

% JF: next parag a bit low-level, and not really related to background
% Each instruction is included in one file
% in order to perform correctness proofs individually.
% These files contains no \texttt{main} declaration.
% During translation, we add automatically a empty \texttt{main} in every file.

%%% Local Variables:
%%% mode: latex
%%% TeX-master: "thesis"
%%% End:

\include{formal}
\include{interp}
\include{framework}
\include{correctness}
\newcommand{\coqvartt}[1]{\texttt{#1}}

\chapter{Designing our own inversion}
\label{cpt:inv}
% DONT REMOVE THIS REMINDER: biffer avec \st
%\cjf{Comments by JF will be presented in this way.}\margjf{1}{Or this way in the margin}

In correctness proofs of ARM instructions, which involve
the large-size inductively defined relation coming from \compcert C semantics,
many steps require inverting a hypothesis to 
perform a case analysis and extract all useful constraints from the hypothesis.
The Coq built-in tactic \inversion is usually considered to be the right choice
in such situations.
But using it made us suffering from severe controllability,
maintenance and efficiency issues.
To circumvent these issues, we propose an inversion technique based on the
combination of an antidiagonal argument and the impredicative encoding
of inductive data-structures,
which we are going to introduce in this chapter.
Part of the material presented of this chapter has been published in~\cite{2013itp}.

\selectlanguage{french}
\section*{Résumé}

\begin{resume}
Dans les preuves de correction des instructions ARM,
qui reposent sur des relations définies inductivement de grande taille,
issues de la sémantique de \compcert C,
de nombreuses étapes consistent à inverser une hypothèse pour effectuer
une analyse de cas et extraire toutes les contraintes utiles contenues 
dans cette hypothèse.
La tactique Coq standard \inversion est généralement considérée comme
étant le bon choix dans de telles situations.
Cependant son utilisation nous a posé de graves problèmes de contrôlabilité,
de maintenance et d'efficacité.
Pour les résoudre, nous proposons une technique d'inversion basée sur
la combinaison d'un argument antidiagonal et d'un codage imprédicatif
des structures de données inductives,
qui fait l'objet de ce chapitre.
Le matériel présenté ici a été partiellement publié dans~\cite{2013itp}.
\end{resume}

\selectlanguage{english}

\section{Why a new inversion}
\label{sec:ninv}
\subsection{Inversion tactic in Coq}
During the development of a proof, 
if a hypothesis is an instance of an inductive predicate
and we want to derive the consequences of this hypothesis,
the general logical principle to be used is called \emph{inversion}.
To this effect, the Coq proof assistant provides
a useful tactic called \inversion \cite{coqmanual}
which is available in several variants.

An inversion is a kind of forward reasoning step that allows
for users to extract all useful information contained in a hypothesis.
It is a case analysis over
a given hypothesis according to its specific arguments,
that removes absurd cases,
introduces relevant premises in the environment and performs suitable
substitutions in the whole goal.
The practical need for automating inversion has been identified 
many years ago and most proof assistants (Isabelle, Coq, Matita,...) 
provide an appropriate mechanism.

% \margjf{3}{to be moved as a remark, after explaining the central line}
% \jf{When some inversion steps are to be used repeatedly,
% appropriate inversion lemmas can be prepared using}
% \coqdockw{Derive Inversion};
% the generated inversion principle can be used with
% the \coqdockw{inversion ... using} tactic.

% If \coqvartt{I} is the inductive predicate and \margjf{2}{unfinished sentence}


%When you want to perform \inversionon a hypothise \coqvartt{H},
%which has form \textit{I t}. And \textit{I} is

\subsection{Issue from \compcert C semantics}
%Almost, need to re-read
\compcert C semantics is a quite big and complex inductive relation. Each 
constructor describes the memory state transformation of an expression,
statement, or function.
In the theorems we aim at proving,
ARM instructions are represented by C functions containing
a sequence of statements
which can be decomposed into complicated expressions.
As soon as
we want to discover the relation between memory states before and after
evaluating an expression, we have to invert hypotheses of operational semantics to
follow the clue given by its definition.
To perform such inverting we can use
\inversion. But each use of \inversion will go one step only.

For illustration, we present here a small excerpt from an old
proof script in \simsoccert using \inversion, which belongs to
the ADC instruction. It sets the CPSR with the value of SPSR.
The pseudo-code from the ARM reference manual is just
\texttt{CPSR = SPSR}.
The corresponding C code is a call to function \texttt{copy\_StatusRegister},
which sets CPSR field by field by the values from SPSR.
Lemma \texttt{same\_cp\_SR} states that the C memory state of the simulator
and the corresponding formal representation of ARM processor state
evolve consistently during this assignment.
\begin{alltt}
Lemma same_copy_SR :
  forall e m l b s t m' v em,
  proc_state_related m e (Ok tt (mk_semstate l b s)) ->
  eval_expression (Genv.globalenv prog_adc) e m expr_cp_SR t m' v ->
  forall l b,
  proc_state_related m' e
   (Ok tt
       (mk_semstate l b (Arm6_State.set_cpsr s (Arm6_State.spsr s em)))).
   ).
\end{alltt}

After a couple of introductions and other administrative steps,
we get the following goal, where \texttt{cp\_SR} is unfolded in hypothesis H.
\texttt{cp\_SR} is the identifier of \compcert C representation, which
calls to the function \texttt{copy\_StatusRegister} with arguments CPSR as
setting destination and SPSR as source.

\begin{alltt}
  l' : local
  b' : bool
  a' : expr
  H : eval_expr (Genv.globalenv prog_adc) e m RV
         (Ecall (Evalof (Evar copy_StatusRegister T14) T14)
            (Econs
               (Eaddrof
                  (Efield (Ederef (Evalof (Evar proc T3) T3) T6)
                     adc_compcert.cpsr T7) T8)
               (Econs
                  (Ecall (Evalof (Evar spsr T15) T15)
                     (Econs (Evalof (Evar proc T3) T3) Enil) T8) Enil))
            T12) t m' a'
  ============================
   proc_state_related m' e st'
\end{alltt}

Then we have to invert \texttt{H} and similar generated hypotheses until
all constructors used in its type are exhausted.
Here 18 consecutive inversions are needed. Using \inv tactic invented by
\compcert, 
which performs standard \inversion, clearing the inverted hypothesis,
and rewriting of all auxiliary equations,
the sequel of the script started as follows:

\begin{alltt}
  inv H. inv H4. inv H9. inv H5. inv H4. inv H5. 
  inv H15. inv H4. inv H5. inv H14. inv H4. inv H3. 
  inv H15. inv H5. inv H4. inv H5. inv H21. inv H13.
  ...
\end{alltt}

The old proof script includes a lot of code in this style,
which makes the size of the code huge and hard to manage.

% \margjf{1}{Provide some details, in 1 or 2 lines, explaining
% which inductive relations are involved.
% Refer to a previous chapter explaining Compcert-C semantics}
% For a complex expression, we have to push \inversion many
% times. Thereofre, we get proof scripts which are heavily filled with calls to 
% \inversion.
% \cjf{\bfseries \fbox{Good place for a picture}}

Another problem is the management of names.
A single \inversion will derive
a dozen of variables and hypotheses according to the corresponding constructor
in \compcert C formal semantics.
% A complex one like \coqvartt{eval\_call} above,
% inverting a hypothesis of this kind will return nineteen new variables 
% and eight new hypotheses.
% Giving names will be boring and repetitive.
With Coq built-in \inversion, their names are automatically generated
using consecutive numbers.
This makes proof scripts highly dependent on such names.
Such a feature is already not very good when writing the proof,
because of the heavy use of inversions and the large number of new names
generated each time.
More importantly, 
the maintenance of proof scripts becomes a terribly awful task:
each use of those uncontrolled names has to be revisited
either when the formal definition of the \compcert-C semantics
changes (upgrading from \compcert 1.8 to 1.9 for instance)
or when the algorithm of Coq for name generation is changed
(this happened from Coq 8.3 to 8.4).
To provide an idea of the burden,
in our first experience using Coq built-in \inversion,
the complete correctness proof on instruction ADC
resulted in a file containing 2500 lines of proof scripts.
Moreover, designing (and maintaining) the scripts was made uncomfortable by
the compilation time of this file more than one minute 
most of the time was spend on \inversion.
Given that there are more than one hundred instructions in ARMv6 ISS,
we considered it as urgent to find a replacement for Coq built-in
\inversion.


% \subsection*{Disadvantage of the traditional inversion}

% So we conclude the disadvantage of using original \inversion:
% \begin{itemize}
% \item 
% \end{itemize}


\section{Design of \hcinv}
%\margjf{2}{Say what hcf stands for: ``hand crafted'' smtg?}
%Here \texttt{hc} is a shorthand for \texttt{hand-crafted}.
Here \hcinv stands for \emph{hand-crafted inversion}.

% \subsection{Inspiration: small inversion}

% %most is copy from paper small_inv, need to change
% Small inversion is a proof trick introduced in ~\cite{small_inv}.
% It is able to perform the same as tactic \inversion in some cases.

% \cjf{OK, you actually do what I wanted from Section \ref{sec:gen-design}.
% So here, you should introduce the plan and make clear where is your
% contribution.
% BTW, an important part of your contrib is the implementation using LTAC.
% The whole chapter should be longer and indeed can be much longer.
% }

% \cjf{\itshape(Obsolete comment) This subsection should present the trick, so that the whole section
% is self-contained.
% Focus on the main point, no need to detail recursive applications for instance.
% And add what is insufficient in \cite{small_inv}.
% Take material from the submitted paper!
% }

% \jf{$\vdots$}

% \cjf{This conclusion should be put later, after Section \ref{sec:gen-design} at least}.
% %
% This technique is flexible and is available in several variants.
% Comparing to regular \inversion, small inversion has also strong
% elimination\margjf{3}{Explain in more detail, or remove}.
% The differences between standard \inversion and small
% inversion are: when we want to reduce recursively absurd hypotheses, standard
% \inversion will repeat yields the recursive case until reach the
% absurd case; small inversion is more intuitive
% to change the goal directly to what it will become according to definition.
% The printed version of proof terms of small inversion already shows a big
% advantage from lines of code and the produced .vo file.
% More examples are in ~\cite{small_inv}.

\subsection{General design concept and example}
\label{sec:gen-design}

Small inversion is a proof trick introduced in ~\cite{small_inv}.
It is able to perform the same as tactic \inversion in some cases.

>From the idea of small inversion\cite{small_inv}, we have built a more
powerful inversion through several improvements and validated it to
realistic applications.  The following examples introduce our
development step by step.  To make it easy to understand, we choose a
well known example about even defined for Peano's natural number. Its
inductive definition is :

\begin{alltt}
Inductive even_i : nat -> Prop :=
  | E0 : even_i 0
  | E2 : forall n, even_i n -> even_i (S (S n)).
\end{alltt}

As explained in ~\cite{small_inv}, the main idea is to build the
corresponding auxiliary diagonalization function.

% \cjf{You often use sentence like the previous one, with a \emph{it}
% which refers to nothing.
% I think that you mean something like:\\
% ``As explained in ~\cite{small_inv}, the way is to build the
% corresponding auxiliary diagonalization function''.\\
% Or better:\\
% ``As explained in ~\cite{small_inv}, the main idea is to build the
% corresponding auxiliary diagonalization function''.
% }

First, the inductive predicate \coqvartt{even\_i} is a dependent data type.

% \margjf{5}{say why you write that: interactive construction of \texttt{match} is more convenient}
Using primitive tactics \coqdockw{case} or
\coqdockw{destruct} is powerful enough
to perform dependent pattern matching on an assumption of type
\coqvartt{even\_i n}
when the conclusion of the current goal shares the
same arguments as the hypothesis to be case analyzed.
If not so,
one cannot return the desired new goal with the converted arguments
by using only \coqdockw{case} or \coqdockw{destruct}

% \cjf{``powerful enough''... for what? Do you mean:}

% \jf{In order to perform dependent pattern matching on an assumption
% of type \coqvartt{even\_i n},
% we can use
% the primitive tactics \coqdockw{case} or \coqdockw{destruct},
% which automatically find suitable terms for the \coqdockw{return}
% clause of the \coqdockw{match} construct.}
%
Assume there are two proof terms \coqvartt{t0} and \coqvartt{t2} for
constructors \coqvartt{E0} and \coqvartt{E2}. The two proof terms have
different types. The type of \coqvartt{t0} is \coqvartt{P 0},
the type of \coqvartt{t2} is \coqvartt{P (S (S n))}.
Therefore, the syntax of the \coqdockw{match} construct contains a
\coqdockw{return}
clause with the expected type of the result \coqvartt{P n} as an argument;
moreover, there
is also an \coqdockw{in} clause for the type of \coqvartt{H} which binds
\coqvartt{n}:

% \noindent\cjf{A sentence is missing here, for introducing the next term in the \texttt{allttt} paragraph.}
%
\begin{alltt}
match H in even_i n return P n with
  | E0 => t0
  | E2 e ex => t2
end
\end{alltt}

Assuming a hypothesis \coqvartt{H} of type \coqvartt{even\_i n}
and a conclusion of type \coqvartt{P n}, both sharing variable \coqvartt{n},
then applying a case analysis
on \coqvartt{H} will build a proof term in the same form
as the code above and generate two new sub-goals \coqvartt{P 0} and
\coqvartt{P (S (S x))} with the additional assumption \coqvartt{even\_i x}.

%\st{When}
Sometimes, there is no obvious relation between the hypothesis and conclusion.
For example, %\st{we have a}
consider the following lemma: $even\_i~1 \rightarrow 3=4$,
%\st{The inductive predicate} \coqvartt{even\_i} \st{has parameter} \coqvartt{1}.
%\st{But}
where the conclusion ($3 = 4$)
%\st{has nothing to do with parameter} \coqvartt{1}.
is not related to the argument of \coqvartt{even\_i} (1).
As mentioned before, our interactive destruct works only if the hypothesis we
want to destruct and the conclusion share the same argument.
In order to fix this, we have to convert the conclusion of the current goal into
a function of $1$. We define a diagonalization function
\coqvartt{diag} which matches the key parameter and returns the
conclusion of the current goal:

\begin{alltt}
let diag x :=
  match x with
    | 1 => 3 = 4
    | _ => True
  end in
  match H in even_i n return diag n with
    | E0 => I
    | E2 _ _ => I
  end
\end{alltt}
%\cjf{{\rm\coqvartt{concl}} concl \rm is not a constant \(\rightarrow\) use a different font}

Then a case analysis on \coqvartt{H} will return two sub-goals: \coqvartt{diag 0},
and \coqvartt{diag (S (S y))} ending up with a proof term
for \coqvartt{True}.

% \cjf{Revoir: diag 1 is not proved, it has to be  (and actually is) convertible
% with the conclusion of the current goal.
% Reprhase the whole previous explanation, saying the interactive destruct
% works if the goal has the shape
% \texttt{even 1 |- something 1}}

%% \herejf %%

However, the technique explained in the previous section has to be extended in
order to cover more general situations.

The first improvement we have to provide is to make the diagonalization function
independent from specific conclusion if we want it to be used for any possible
goal. We use $\forall X:Prop$ instead of a specific conclusion to hook the
current conclusion.
Then the previous diagonalization function will be replaced.
% by:\\
%\begin{center}
%$diag~x~:=~\coqdockw{match}~x~\coqdockw{with}~1 \Rightarrow \forall(X:Prop),~X\mid \_ \Rightarrow True~\coqdockw{end}$\\
%\end{center}
Then together with the previous proof term of type $\forall X,X$,
it is able to apply any conclusion:
\begin{alltt}
let diag x :=
  match x with
    | 1 => forall X : Prop, X
    | _ => True
  end in
  match H in even_i n return diag n with
    | E0 => I
    | E2 _ _ => I
  end
\end{alltt}

The second is to consider a positive case.
Let us consider the following theorem as an example,\\
\begin{center}
$\forall n~m, \coqvartt{even\_i n} \rightarrow \coqvartt{even\_i (n+m)}
\rightarrow \coqvartt{even\_i m}.$
\end{center}
The proof is led by induction on \coqvartt{even\_i n}.
According to the constructor of inductive type \coqvartt{even\_i},
induction generates two
sub-goals: \coqvartt{even\_i (0 + m)} and \coqvartt{even\_i (S (S (n + m))}.
The first is easy to solve.
Then an induction hypothesis will be added to the local context:
$even\_i~(n + m) \rightarrow even\_i~m$.
If we want to continue, we need a link from
\coqvartt{S (S (n + m))} to \coqvartt{n + m}, and it is exactly
the second constructor \coqvartt{E2} of inductive type \coqvartt{even\_i}.
So we expect our technique could also express the premise of the focused
constructor.
We propose a new \coqvartt{diag} function and proof
term defined as follows:
\begin{alltt}
let diag x :=
  match x with
    | S (S y) => forall X: Prop, (even_i y -> X) -> X
    | _ => True
  end in
  match H in even_i n return diag n with
    | E2 p e => fun X k => k e
    | _ => I
  end
\end{alltt}

Then, applying the new technique in current hypothesis $H:~even\_i(S(S(n~+~m)))$
yields a function in continuation passing style.
The type parameter \coqvartt{X} identified to the conclusion
\coqvartt{even\_i m}; then \coqvartt{y} binds to \coqvartt{n + m},
and the goal converts to $even\_i~(n + m) \rightarrow even\_i~m$.
That is exactly what we expected.
Our inversion function can be seen as inversion lemmas, but their type is the
dependent type expressed by their own \coqvartt{diag}.
The difference between our diagonalization function and the Coq built-in
\coqdockw{Derive Inversion} will be introduced at the end of this section.

To summarize this new diagonalization function, when there is an inductive type
$\coqvartt{I}(\coqvartt{t})$,
where \coqvartt{t} is the parameter of type \coqvartt{T},
and $\coqvartt{C}_i$ is a constructor of \coqvartt{I} depending on
parameter $\coqvartt{t}_i$ of type \coqvartt{T},
\coqvartt{pi} is the premise in constructor $\coqvartt{C}_i$,
$\mathcal{P}$ consists of a constructor of type \coqvartt{T},
we want to filter.
Then a constructor of the inductive type \coqvartt{I(t)} containing $\mathcal{P}$
can be expressed like
$\coqvartt{C}_i:\forall ~\coqvartt{p}_i,~\coqdocvar{I}~\mathcal{P}$.
And \coqvartt{HI} is the hypothesis of type \coqvartt{I(t)} we want to invert.
In the general case, we have to consider if there are more than one possible
constructors containing $\mathcal{P}$, like constructor $C_i$, $C_j$, etc.
The inverting lemma corresponding to $\coqvartt{A}\mathcal{P}$ is:

\begin{alltt}
let diag x :=
  match x with
    | P => forall X: Prop, (forall pi, X) ... (forall pj, X) -> X
    | _ =>  True
  end in
  match HI in I t return diag t with
    | Ci ei => fun X ki => ki ei
    ...
    | Cj ej => fun X kj => kj ej
    | _ => I
  end
\end{alltt}
Remark the close relationship with the impredicative encoding of data types in
system~F.

Next, we consider more than one parameter in an inductive type.
The difference when we have more parameters is that using the previous inverting
strategy, the identifiers for the same variable in premise and conclusion
cannot be related.
This problem was discovered when applying our inverting technique
to the \simsoccert project.
Let us introduce a new example in order to explain the problem properly.
Here is a toy language that accepts two operations:
\coqvartt{tm\_const} and \coqvartt{tm\_plus}.
The output type \coqvartt{val} is a natural number
or a Boolean.
The evaluation (\coqvartt{eval}) takes an argument of type \coqvartt{tm}
and returns a value of type \coqvartt{val}.
The Coq code is as follows:

\begin{alltt}
Inductive tm : Type :=
  | tm_const : nat -> tm
  | tm_plus : tm -> tm -> tm.

Inductive val : Type :=
  | nval  : nat -> val
  | bval  : bool -> val.

Inductive eval : tm -> val -> Prop :=
  | E_Const : forall n,
      eval (tm_const n) (nval n)
  | E_Plus : forall t1 t2 n1 n2,
      eval t1 (nval n1) ->
      eval t2 (nval n2) ->
      eval (tm_plus t1 t2) (nval (plus n1 n2)).
\end{alltt}

In the inductive type \coqvartt{eval},
the constructor \coqvartt{E\_Plus} has four
variables: \coqvartt{t1}, \coqvartt{t2}, \coqvartt{n1}, and \coqvartt{n2}.
The premises and the conclusion share these variables.
Without special care
we lose the information of relationship of sharing.

Let us consider a theorem,
$$\forall v,~\coqvartt{eval}(\coqvartt{tm\_plus}(\coqvartt{tm\_const}~1)~(\coqvartt{tm\_const}~0))~v~\rightarrow ~v~=~\coqvartt{nval}~1.$$
The diagonalization function corresponding to the previous method is:
\begin{alltt}
match x with
  | tm_plus tc1 tc2 =>
      forall X: Prop,
      (forall n1 n2, eval tc1 (nval n1) -> eval tc2 (nval n2) -> X) -> X
  | _ => True
end
\end{alltt}
But then, the fact that \coqvartt{v} should be
\coqvartt{nval (plus n1 n2)} is not recorded.
The solution is to add a parameter to \coqvartt{X} to keep this identification
after evaluation. The modified diagonalization function for the constructor
\coqvartt{E\_Plus} is:
\begin{alltt}
match x with
  | tm_plus tc1 tc2 =>
      forall X: tm -> Prop,
        (\(\forall\) n1 n2, eval tc1 (nval n1) ->
                  eval tc2 (nval n2) -> X (nval (plus n1 n2))) -> X v
  | _ => True
end
\end{alltt}

This example also introduces another problem we had not foreseen: a
constructor may have more than one diagonalization function.
Considering the same theorem as above, after inverting
\coqvartt{E\_Plus}, the current proof goal is:
\begin{alltt}
n1 : nat
n2 : nat
e1 : eval (tm_const 0) (nval n1)
e2 : eval (tm_const 1) (nval n2)
============================
 nval (n1 + n2) = nval 1
\end{alltt}
We expect inverting e1 and e2 can give us the nat value of \coqvartt{n1} and
\coqvartt{n2}.
Without any consideration, we defined the diagonalization function
for \coqvartt{E\_Const} like this,
\begin{alltt}
match t with
  | tm_const n => forall (X: val -> Prop), X (nval n) -> v
  | _ => True
end
\end{alltt}
It chooses to keep the value for type \coqvartt{val}.
Then we notice in current conclusion there is no \coqvartt{nval n1} or
\coqvartt{nval n2} but \coqvartt{nval (n1 + n2)}.
The previous diagonalization function is not able to get the value of
\coqvartt{n1} or \coqvartt{n2}.
The diagonalization function should focus on a variable of type \coqvartt{nat}
instead of \coqvartt{val}.
The pattern matching should match both input and output parameters of
\coqvartt{eval}.
\begin{alltt}
match t, v with
  | tm_const tc, nval n => forall (X: nat -> Prop), X tc -> n
  | _, _ => True
end
\end{alltt}

In summary, the diagonalization function is defined depending upon
what conclusion we have.
When we have a conclusion like in this example, we choose
the second diagonalization function.
If the conclusion contains only \coqvartt{nval n}, we can choose the first one.

% % JF -> XM: only relevant to \cite{small_inv}, obsolete here.
% The differences between standard \inversion and small
% inversion are: when we want to reduce recursively absurd hypotheses, standard
% \inversion will repeat yielding the recursive case until reach the
% absurd case; small inversion is more intuitive
% to change the goal directly to what it will become according to definition.
% The printed version of proof terms of small inversion already shows a big
% advantage from lines of code and the produced .vo file.
% More examples are in ~\cite{small_inv}.

% \subsection {Improvement during correctness proving}

% The development of \hcinv step by step applied to inverting one constructor
% of type eval\_expr in \compcert-C semantics. It is eval\_call which specifies
% the big-step evaluation of function call.

% First try: we use similar technique as small inversion.
% When it is the default case, we return a series conjunction of equalities
% in order to keep the consistency of parameter values.
% Else, it returns \coqvartt{False}.

% \subsection{Key parameters decision}

% A very important point in this inversion technique is the key parameter. It is
% the one we focus on to decide which case is it in the corresponding inductive
% definition.
% How to choose this key parameter is the subject of this section.
% The type of a inductive relation indicates the necessary parameters.
% And the case analyze always depend on one of these parameters.
% Then such parameter is the main key we are looking for.
% In the given example \coqvartt{eval}, the key parameter is of type \textbf{tm}.
% When it is \coqvartt{tm\_const}, it is in case \coqvartt{E\_Const}.
% When it is \coqvartt{tm\_plus}, it is in \coqvartt{E\_Plus}.
% But in some situation, only one key parameter is not enough.



\subsection{Using our hand-crafted inversion in SimSoC-Cert}
\label{ssec:invssc}
% \cjf{Much more details needed.
% Insist on the gain in \emph{manageability} and on \emph{flexibility},
% in a framework where full automation is not reachable,
% so we need interactive proofs with good control.
% Explain the ``howto'' write high-level tactics.}

We use the new inversion to define a new inversion tactic \coqvartt{inv\_[expr]} 
for inductive type \coqvartt{eval\_expr} in \compcert.
The semantics of \compcert C tells us how the memory state 
is transformed by evaluating expressions (Section~\ref{ssec:ccc}).
Like explained in the previous subsection,
an auxiliary function has to be defined for each constructor of
\coqvartt{eval\_expr}.

First, we define the diagonal-based function for each constructor
of $eval\_expr$, following the lines given in the previous section.
For example, 
the evaluation of a field is defined in \compcert by the following rule.
\begin{alltt}
Inductive eval_expr :
  env -> mem -> kind -> expr -> trace -> mem -> expr -> Prop :=
  ...
  | eval_field : \(\forall\) e m a t m’ a’ f ty,
      eval_expr e m RV a t m’ a’ ->
      eval_expr e m LV (Efield a f ty) t m’ (Efield a’ f ty)
\end{alltt}

We then define (observe that 2 variables and 1 hypothesis will be generated):
\begin{alltt}
Definition inv_field {g} {e} {m} {ex} {t} {m'} {ex'}
  (ee:eval_expr g e m LV ex t m' ex') :=
  let diag e ex ex' m m' :=
    match ex with
      | Efield a b c =>
        \(\forall\) (X:expr->Prop),
        (\(\forall\) t a', eval_expr g e m RV a t m' a' -> X (Efield a' b c)) -> X ex'
      | _ => True
    end in
    match ee in (eval_expr _ e m _ ex _ m' ex')
             return  diag e ex ex' m m'  with
      | eval_field _ _ _ t _ a' _ _ H1 =>   fun X k => k t a' H1
      | _ => I
    end.
\end{alltt}

Every instruction contains a quite complex expression.
If we want to find the
relation between the memory states affected by these expressions,
we have to invert many times
even if we use the new \hcinv.
These steps are repetitive, applying the right diagonal-based functions
with the same pair of memory states as parameters to the focused hypothesis.

Using the \texttt{match goal} construct of LTac,
we can define a high-level tactic for each inductive type,
gathering all the functions defined for its constructors.
For example, the inversion tactic for $\coqvartt{eval\_expr}$ contains:

\begin{alltt}
Ltac inv_eval_expr m m' :=
  ...
  let t1_:=fresh "t" in
  let v1_:=fresh "v" in
  let ev_ex1 := fresh "ev_ex" in
  ...
  match goal with
  ...
    | [ee: eval_expr ?ge ?e m LV (Efield ?a ?f ?ty) ?t m' ?a' |- ?cl] =>
      apply (inv_field ee); clear ee; intros t1_ a1_ ev_ex1; intros;
      inv_eval_expr m m'
\end{alltt}

This tactic has two arguments $\coqvartt{m}$ and $\coqvartt{m'}$,
corresponding to C memory states.
The first \texttt{intros} introduces the 3 generated components 
with names respectively prefixed by \coqdoccst{t}, \coqdoccst{v} and \coqdoccst{ev\_ex}.
The second \texttt{intros} is related to previously reverted 
hypotheses, their names are correctly managed by Coq.
The tactic proceeds as follows:
\begin{itemize}
\item 
it automatically finds the hypothesis we want
to invert by matching the targetted memory states;
\item 
related hypotheses are reverted;
\item 
the right auxiliary function is called
(all auxiliary functions are gathered in the tactic);
\item 
meaningful names are given to derived variables and hypotheses;
\item 
all other related hypotheses are updated according to the new names
and new values;
\item 
useless variables and hypotheses are cleaned up ;
\item 
the steps above are repeated until all transitions between
the two targetted memory states are discovered. 
\end{itemize}
\noindent
We name this tactic \coqvartt{inv\_eval\_expr};
all inversions on hypotheses of type \coqvartt{eval\_expr} are replaced by
\coqvartt{inv\_eval\_expr}.
%
For example, 18 standard \inv were used in the old proof script of 
lemma \texttt{same\_copy\_SR}.
With the high-level tactic, the 18 \inv can be replaced by one step:
\texttt{inv\_eval\_expr~m~m'}.

Inverting a hypothesis of type \coqvartt{eval\_expr} may introduce new
hypotheses on internal memory states according to the premises 
in the definition of the constructor.
% % JF: the way you put sentences one after the other is
% % hard (impossible) to follow. I don't understand what you do
% %  - introducing an new topic
% %  - explain the previous sentence, or add details related to the 
% %  - or ?
% %  You should use "joining words" such as "therefore", "moreover", "but"
% %  And also, use structure in the paragraphs (new paragraphs, itemize, etc.). 
% % Here I guess something, it may be wrong. Please check.
% % I also suggest that you give more code in \coqvartt{inv\_eval\_expr}
% % in order to illustrate your informal comments (in contrast with
% % a conference paper, you have room). 
%
% % I dont' understand if your "next" here refers to something wich is
% % computed in the tactic and automatically inverted,
% % or if it is soemthing for the next interaction of the user
% % (ie.e., in the script of the lemma).
% % Intuitively I believe that it is the 2nd. 
% 
% To find which is the next to invert is not random. 
% This is because of automatic name giving process. 
% reference [AAA] (used below)
The automatic naming scheme in our tactic provides useful clues
which are helpful in the script of a proof.
%
% % JF: I'm confused here. Do you mean that we get an explicit equality,
% % i.e., a hypothesis containing an equality ? 
% % And then you use a rewrite step ?
% % Hard to believe. I tend to think that such equalities are implicit,
% % they come from identifications of arguments in the inductive relation
% % which is inverted.
% % It corresponds to base cases if the inductive relation.
%
% To invert those hypotheses containing internal memory state,
% sometimes we get an equality between two states
% $\coqvartt{m}_{i1}~=~\coqvartt{m}_{i2}$.
% Then $\coqvartt{m}_{i1}$ will be replaced by $\coqvartt{m}_{i2}$.
%
Sometimes, inverting a hypothesis will identify two memory states
$\coqvartt{m}_{i1}$ and $\coqvartt{m}_{i2}$.
Then $\coqvartt{m}_{i1}$ is automatically replaced by $\coqvartt{m}_{i2}$.
%
% % XM (hard to understand for JF)
% If the memory state we use to find the
% hypothesis to invert is replaced by its following state,
% it is not possible to continue our automatic process.
% The inversion order of hypotheses is go backward in order to avoid the memory
% state parameter is replaced. 
% % JF (Again I try something which may be wrong)
%
Such replacements trouble the automatic process in our tactic,
because the first memory state $\coqvartt{m}_{i1}$ is used
for finding the next hypothesis to be inverted.
This issue is solved by inverting hypotheses in backward order.

% % JF redundant with [AAA] above
% Note that the names are not explicitly given in the script,
% which would be cumbersome.
% They are generated in our tactic.

% % JF redundant
% This ad-hoc tactic is a wrapper for all inverting diagonalization functions.
%
Our \hcinv makes it possible to have a convenient automatic naming algorithm
because the arguments that need to be named are fixed and
are known directly from the inductive type definition itself. 
It does not work with standard inversion because,
other than the arguments and premise of the inductive definition itself,
extra equalities may be introduced and 
% % JF : don't understand
% the order preservation of their order may not accurate.
hypotheses may be reordered in a way which is not under our control.




\subsection{Comparing \hcinv with Coq built-in inversions}
\label{ssc:adinv}


\sloppy
There are three Coq built-in tactics that can achieve inverting the
hypothesis of current proof goal.  
They are the standard
\inversion, \coqdoctac{Derive Inversion}, 
and \coqdoctac{dependent induction/destruction}.
We already discussed the tactic \inversion. 
The tactic \coqdoctac{Derive Inversion} allows
the user to first automatically generate an inversion principle according to
an inductive type and then to apply it to inverting target.
The tactics \coqdoctac{dependent induction} and 
\coqdoctac{dependent destruction}
are another option for inverting inductive predicate instances and
potentially doing induction at the same time. They are based on
BasicElim of Conor McBride~\cite{mcbride96}
and work by abstracting each argument of an inductive instance 
by a variable and constraining it by suitable equalities.
The usual induction and destruct tactics can then be applied
to the abstracted instance and after rewriting of the equalities,
we get the expected goals.
\fussy

\paragraph{Ease of use.}
If we compare these three options, without considering
the issues on name control,
\coqdoctac{Derive Inversion} is the most inconvenient one.
It finds the clues according to type definition of inverted hypothesis,
without telling which one it matches and the returned premises are not
introduced.
% >>>>>>> variant A
\compcert defines \coqdoctac{inv} as a combination of 
the standard \inversion with substitution and clearing.
% which makes the proof script short. 
So for a basic usage, it is not complicate to use.
% applicative complexity point of view, 
We think \coqdoctac{BasicElim} is easier to use 
than the two other built-in tactics.
New equalities hypotheses will be rewritten 
and existing premises of equation can be kept by a block.
It handles the recursive type definition. 
% % JF: Don't understand (syntax) but does not seem that important
% The name of the inverting target is passed to its inductive premise of the same type. 

If we take name control issues into account,
both \coqdoctac{Derive Inversion} and \coqdoctac{BasicElim} are hard 
to use. 
% That is because they are not able to find the corresponding
% constructor by only applying themselves to inverting an inductive
% type.
% This is because nothing is provided to To give names, we have
% to consider all cases of the inductive type.
Names have to be provided for all cases given by the constructors.
For example, we have to consider 16 cases for \coqvartt{eval\_expr}.
Even if we just use a wild-card in impossible cases,
15 wild-cards are still needed for them,
as well as extra tactics for concluding.

\medskip
% Due to the changes between different \compcert
% versions, we have to change all the proofs related to those
% changes first. That is a quite large work. But now we only have to change
% the \hcinv definitions.  During developing our project, it also saves
% time for less modification when you want to ameliorate the existing
% proof script.
The price we have to pay for gaining controllability and
accurate management of names is that \hcinv has to be updated
with each release of \compcert.
This requires some work.
But as expected, proof scripts themselves are robust, 
changes occur only in the definitions related to \hcinv.
In our developments,
after \hcinv became available,
proof scripts could also be improved much more easily,
achieving a good separation of concerns between the design of proofs
and technical issues on inversion.

% >>>>>>> variant B
% The CompCert defined \coqdoctac{inv} combines the standard \inversion with
% subs-traction and clearing, which makes the proof script shorter. From the
% applicative complexity point of view, it is not complicate to use.
% We think \coqdoctac{BasicElim} is better than the two.
% New equivalent hypotheses can be
% rewritten and existing premises of equation can be kept by a block.
% And it focuses on the recursive type definition. The name of the
% inverting target is passed to its inductive premise of the same type.

% Then we take naming control into account. Both \coqdoctac{Derive
%   Inversion} and \coqdoctac{BasicElim} are hard to apply then, because
% they are not able to find the corresponding constructor by only
% applying themselves to inverting an inductive type. To assign names, we
% have to consider all cases of the inductive type. For example, we have
% to consider all 16 cases for type \coqvartt{eval\_expr}. Even if we
% only place wild-card for impossible cases, there are 15 wild-cards to
% make.  And extra tactics are needed to deny the unmatched
% constructors.

% The other price to pay is the generated proof terms of the other
% three.  Comparing the compilation results (in Table ~\ref{t:size})
% shows the advantage of using our \hcinv. Although built-in tactic
% \coqdoctac{Derive Inversion} has better timing of type-checking, the
% compiled output (.vo file) is the biggest, four times comparing to the
% other built-in ones, ten times comparing to \hcinv.

% Especially for inverting the complex types in CompCert-C, using \hcinv
% is much more efficient. Due to the changes between different CompCert
% versions, we had to change all the proofs related to those
% changes first. That is a quite large work. But now we only have to change
% the \hcinv definitions.  During development of our project,
% it made it easier to maintain the existing proof scripts.

% We take one lemma from correctness proof of instruction ADC to compare three
% kinds of inversion tactics: \inversion, \coqdoctac{Derive Inversion\_clear},
% and our inversion \hcinv.
% This lemma discusses the memory state change between expressions including
% \coqvartt{Econdition}, \coqvartt{Ebinop}, \coqvartt{Evalof}, \coqvartt{Eval},
% and \coqvartt{Evar}~\ref{t:timing}.
% We compare the time used by performing each inversion and the size of output
% object files (.vo) by compilation ~\ref{t:size}.
% %%Test comparison results is needed here
% ======= end

\paragraph{Performance.}

\begin{table}[t]
\centering
\caption{Time costs (in seconds)}
\label{t:timing}
\begin{tabular}{|l|c|c|c|c|}
\hline
 & standard \inversion & \coqdoctac{Derive Inversion} & \coqdoctac{BasicElim} & our inversion \\
\hline
%Full example & 1.628102 & 0.976061 & 1.428089 & 0.31202 \\
Full example & 1.628 & 0.976 & 1.428 & 0.312 \\
\hline
%Ecall & 0.132009 & 0.076004 & 0.112007 &  0.028002\\
Ecall & 0.132 & 0.076 & 0.112 &  0.028\\
\hline
%Evalof &  0.132008 & 0.072004 & 0.092005 & 0.020001\\
Evalof &  0.132 & 0.072 & 0.092 & 0.020\\
\hline
%Evar &  0.128008 & 0.064004 & 0.084006 & 0.024001\\
Evar &  0.128 & 0.064 & 0.084 & 0.024\\
\hline
%Eaddrof &  0.140009 & 0.076005 & 0.104007 & 0.020001\\
Eaddrof &  0.140 & 0.076 & 0.104 & 0.020\\
\hline
\end{tabular}
\end{table}


\begin{table}[t]
\centering
\caption{Size of compilation results (in KBytes)}
\label{t:size}
\begin{tabular}{|l|c|c|c|c|}
\hline
 & standard \inversion & \coqdoctac{Derive Inversion} & \coqdoctac{BasicElim} & our inversion \\
\hline
%Full example & 191346 & 459613 & 170821 & 36712\\
Full example & 191 & 460 & 171 & 37\\
\hline
\end{tabular}
\end{table}

% % JF -> XM: take care of the order of your paragraphs!
% % You put conclusion and speak about tables, 
% % then say what are the tables about. I have to reverse that

% The other price to pay is the generated proof terms of the other
% three.  Comparing the compilation results (in Table ~\ref{t:size})
% shows the advantage of using our \hcinv. Although built-in tactic
% \coqdoctac{Derive Inversion} has the advantage in timing of
% type-checking, the compiled output (.vo file) is the biggest, four
% times comparing to the other built-in ones, ten times comparing to \hcinv.
Another clear advantage for our \hcinv is efficiency.
Proof terms generated by \hcinv are much smaller than by the three
built-in tactics, as shown on examples taken in \simsoccert,
see Tables~\ref{t:timing} and~\ref{t:size}).
The comparison is performed
on a lemma taken from the correctness proof of instruction ADC.
This lemma discusses how the memory state changes 
during the evaluation of expressions including
\coqvartt{Econdition}, \coqvartt{Ebinop}, \coqvartt{Evalof}, \coqvartt{Eval},
and \coqvartt{Evar}.
% We consider the three built-in inversion tactics
% (\inversion, \coqdoctac{Derive Inversion\_clear})
% and our inversion \hcinv.
We compare the time used for performing each inversion in Table~\ref{t:timing},
and the size of output object files (.vo) in Table~\ref{t:size}. 

We see that \hcinv consumes 4 to 5 times less space than \inversion and 
\coqdoctac{BasicElim}, and 10 times less than
\coqdoctac{Derive Inversion\_clear}.
%
Consistently, and more importantly for the user who heavily uses inversions,
\hcinv reacts much faster (3 to 6 times).
%
Note that, in our experiments, 
\coqdoctac{Derive Inversion} has a better response time
among the three built-in inversion tactics,
but it generates the biggest .vo files.

% % JF \hcinv may be much more efficient on less inductive types as well,
% % we don't know.
% Especially for inverting the complex types in \compcert-C, using \hcinv
% is much more efficient. 
%



%720KB derive_inversion.vo   84KB new_inversion.vo

%%%%%%%%%%%%%%%%%%%%%%%%%%%%%%%%%%%%%%%%%%%%%%%%%%%%%%%%%%%%%%%%%%%%%%%%%%%%%
\paragraph{Inversions out of reach of built-in tactics.}
% Beating \inversion
%\label{sec:beatinv}

Let us now consider a predicate defined on a dependent type.
We take intervals $[1...n]$, formalized as $t$ in the standard library \texttt{Fin},
and then we restrict them to have an odd length.

\begin{alltt}
Inductive t : nat -> Set :=
  | F1 : forall {n}, t (S n)
  | FS : forall {n}, t n -> t (S n).

Inductive odd : forall n : nat, t n -> Prop :=
  | odd\_1 : forall n, odd (S n) F1
  | odd\_SS : forall n i, odd n i -> odd \_ (FS (FS i)).
\end{alltt}

%\medskip
%\coqdocinput{chunk60}
%\medskip

\noindent
Finding the premises for the second constructor is a function 
similar to the one provided for $E2$ above:

%\medskip
%\coqdocinput{chunk61}
%\medskip

\begin{alltt}
Definition premises\_odd\_SS {n} {i: t n} (of: odd n i) :=
  let diag n i :=
    match i with
      | FS \_ (FS \_ y)  => forall (X: Prop), (odd \_ y -> X) -> X
      | \_ => True 
    end in
  match of in odd n i return diag n i with
      | odd\_SS n i o => fun X k => k o
      | \_ => I
  end.
\end{alltt}

\noindent
In particular we can easily prove:

%\medskip
%\coqdocinput{chunk62}
%\medskip

\begin{alltt}
Lemma odd_SS_inv: forall n i, odd _ (FS (FS i)) -> odd n i.
Proof.
  intros n i o. apply (premises_odd_SS o). trivial.
Qed.
\end{alltt}

\noindent
Standard \inversion happens to fail here.
Note that BasicElim may work (we actually could not succeed)
but would need an additional axiom related to John Major equality.


%%% Local Variables: 
%%% mode: latex
%%% TeX-master: "thesis"
%%% End: 


\chapter{Tests generator for the decoder}%{Other facilities}
\label{cpt:other}
% \jf{%
% Main idea:
% From each instruction I, decode I, generate an assembly instruction in txt,
% then using ARM assembly program, generate the binary gain,
% then check that the result is identical to the I.
% \\%
% Only $4.10^9$ instructions.
% Simsoc speed = $10^8$ instructions/s.
% Wrong (depends on decoding speed of Simsoc), to be completed by Vania
% Then finished in 40 s.
% \\%
% Not done for all $4.10^9$ cases but for all instructions
% in all their different modes.
% \\%
% Show some code resulting form the tools.
% }

Currently we do not have a correctness proof of our ARM instruction decoder.
Instead, we have built a decoder tests generator,
which can help to check the coverage and
correctness of the generated C decoder.
% which takes the instruction encoding tables and restrictions of having
% legal instruction as input and produces binary instruction


\selectlanguage{french}
\section*{Résumé}

\begin{resume}
Actuellement, nous n'avaons pas développé de preuve de correction
pour notre décodeur d'instructions ARM intégré à \simlight.
À la place, nous avons construit un générateur de tests pour ce décodeur,
permettant de tester sa couverture et de vérifier qu'il produit
des résultats corrects.
\end{resume}

\selectlanguage{english}
\bigskip

%\section

%\hide{
\begin{figure*}
\centering\footnotesize
\begin{tabular}{|c|c|c|c|c|c|c|c|c|c|}
\multicolumn{10}{c}{\small\em (a) binary encoding of the {\stt ADC} instruction}\\
\hline
31 $\ldots$ 28 & 27 26 & 25 & 24 \dotfill 21 & 20 & 19 $\ldots$ 16 & 15 $\ldots$ 12 & \multicolumn{3}{c|}{11 \dotfill 0} \\\hline
\stt cond & \stt 0~0 & \stt I & \stt 0~1~0~1 & \stt S & \stt Rn & \stt Rd & \multicolumn{3}{c|}{\stt shifter\_operand} \\
\hline
% \multicolumn{10}{c}{~}\\
\multicolumn{10}{c}{\small\em \phantom{\LARGE I}(b) binary encoding of the ``logical shift left by immediate'' operand\phantom{\LARGE I}}\\
\hline
31 $\ldots$ 28 & 27 26 & 25 & 24 \dotfill 21 & 20 & 19 $\ldots$ 16 & 15 $\ldots$ 12 & 11 \dotfill 7 & 6 $\ldots$ 4 & 3 $\ldots$ 0 \\\hline
\stt cond & \stt 0~0 & \stt 0 & \stt opcode & \stt S & \stt Rn & \stt Rd & \stt shift\_imm & \stt 0~0~0 & \stt Rm \\
\hline
% \multicolumn{10}{c}{~}\\
\multicolumn{10}{c}{\small\em \phantom{\LARGE I}(a+b) resulting binary encoding of the flattened instruction\phantom{\LARGE I}}\\
\hline
31 $\ldots$ 28 & 27 26 & 25 & 24 \dotfill 21 & 20 & 19 $\ldots$ 16 & 15 $\ldots$ 12 & 11 \dotfill 7 & 6 $\ldots$ 4 & 3 $\ldots$ 0 \\\hline
\stt cond & \stt 0~0 & \stt 0 & \stt 0~1~0~1 & \stt S & \stt Rn & \stt Rd & \stt shift\_imm & \stt 0~0~0 & \stt Rm \\
\hline
\end{tabular}
\end{figure*}
%}

In order to validate the generated C decoder, we have built an
automatic test generator that generates all possible instructions
excluding undefined and unpredictable ones.  We generate two kind of
files. The first file contains the test instructions binary encoding
in the ELF format. The second file contains the same instructions in the same
order in assembly code.

The decoder has been included in a program that generates, for each
instruction from the binary file, the assembly language of that
decoded instruction. The second generated file can then be compared
with that decoding result: the two files should be identical.

The parameter values are chosen with respect to the validity constraints to
ensure that the instruction is defined and predictable.
For example, the parameters of the {\stt ADC} instruction (see
Fig.~\ref{fig:flatten}) are {\stt Rd}, {\stt Rn}, and {\stt shift\_imm}.
Binary instructions are produced with different combinations of values for them.
From reading the \texttt{Syntax} and \texttt{Usage} part of each instruction,
we know there are several validity constraints for some instructions.
Some validity constraints are dealt with during the parameter generation.
For example, register {\stt Rn} in instruction {\stt LDRBT} cannot be
{\stt PC} ({\stt R15}).
Hence the test chooses directly values between 0 and 14 to be assigned to {\stt Rn}.
Some other validity constraints involve two or more
parameters at the same time.
Continuing the example of {\stt LDRBT}, another constraint states
that {\stt Rd} and {\stt Rn} must be different:
the generator must produce two different values
and assign them to {\stt Rd} and {\stt Rn}.
Similarly, we generate the corresponding assembly code. Under each encoding
table in the reference manual, the \texttt{Syntax} part explains the syntax
of the instruction, the instruction identifier, and the same parameters as in
the encoding table.
The contents of the generated files are shown in Figure~\ref{table:dectest}.
The left column is a group of binary test in hexadecimal format,
which are legal instantiation of \adc instruction.
The right column is their corresponding assembler code according to
the syntax:
$$\texttt{ADC\{<cond>\}\{S\} <Rd>, <Rn>, <shifter\_operand>}$$
They represent one group of \adc with under different combination of
condition of execution and the value of the S bit.

    % 8000:	d2a063ac 	adcle	r6, r0, #-1342177278	; 0xb0000002
    % 8004:	80ab3000 	adchi	r3, fp, r0
    % 8008:	80b0dd8f 	adcshi	sp, r0, pc, lsl #27
    % 800c:	b0b30618 	adcslt	r0, r3, r8, lsl r6
    % 8010:	80bb0da0 	adcshi	r0, fp, r0, lsr #27
    % 8014:	c0bedd31 	adcsgt	sp, lr, r1, lsr sp
    % 8018:	00a157ca 	adceq	r5, r1, sl, asr #15
    % 801c:	80b05251 	adcshi	r5, r0, r1, asr r2
    % 8020:	c0ad3268 	adcgt	r3, sp, r8, ror #4
    % 8024:	00b55574 	adcseq	r5, r5, r4, ror r5
    % 8028:	a0ad806e 	adcge	r8, sp, lr, rrx
\begin{table}[h]
  \centering
  \begin{tabular}{|l|l@{~}|}
    \hline
    binary tests & asm tests\\
    \hline
    52a063ac & ADCLE	R6, R0, \#0xb0000002\\
    80ab3000 & ADCHI	R3, R11, R0\\
    80b0dd8f & ADCHIS	SP, R0, PC, LSL \#27\\
    b0b30618 & ADCLTS	R0, R3, R8, LSL R6\\
    80bb0da0 & ADCHIS	R0, R11, R0, LSR \#27\\
    c0bedd31 & ADCGTS	SP, LR, R1, LSR SP\\
    00a157ca & ADCEQ	R5, R1, R10, ASR \#15\\
    80b05251 & ADCHIS	R5, R0, R1, ASR R2\\
    c0ad3268 & ADCGT	R3, SP, R8, ROR \#4\\
    00b55574 & ADCEQS	R5, R5, R4, ROR R5\\
    a0ad806e & ADCGE	R8, SP, LR, RRX\\
    \hline
  \end{tabular}
  \caption{Generated tests for C decoder}
  \label{table:dectest}
\end{table}

We use the generated binary instruction as input for our decoder.
It outputs the result in assembly code.
Then using the Unix command {\stt diff}, we can compare the decoder
results and the assembly tests.
Several minor issues have been detected and fixed in this way.


% % JF moved to chapt "formal"
% \section{Experimental result and validation}
% \label{ssc:vali}

% With the functional defined semantics, it is possible to extract the whole Coq
% specified ARMv6 simulator into OCaml code and compile it to an executable
% binary. And it is also interesting to see tests running on a formal specified
% simulator.
% We provide the tests for the C specified Simlight.
% In the mean time, we use these tests for our formal specification.
% On one side, we use the {\stt arm-elf-gcc} compiler to
% compile the C tests into ELF files to be used in Simlight.
% On the other side, we write an easy translator
% then to translate the tests to Coq representation
% and then extract to ML files.
% These small tests can be executed on the formal simulator in a sufficient
% simulation speed, less than 1000 instructions per second.
% Running a simple direct sum test takes around five minutes.
% A sorting program would then need one day to be completed.
% % % JF -> XM : should be much less, no? anyway no need to state anything.
% % wheras the correponding C code only needs 0.78 seconds.
% However, we are
% For such formal specification, we do not care the execution speed,
% which aims to perform formal proofs for the simulation correctness.

% The validation result of the C side is worth to mention.
% In SimSoC, there is already an ISS for ARMv5 architecture and corresponding
% tests exist. Thanks to backward compatibility, all these tests can be reused
% on new ARMv6 ISS. And the new ARMv6 ISS can pass these tests and new written
% tests for ARMv6 new instructions.
% % compare with ARMv5 (backward compatibility)
% % use 10 elf files and 3 computers.
% We measure the speed of the generated ARMv6 ISS and hand-written ARMv5 ISS to
% make comparison. We expect that the new ARMv6 ISS is competitve to the existing
% one for v5. And we cannot measure with the compilation to native code because
% it is not available in old ARMv5 ISS.
% We select three benchmarks ``loop'', ``sorting'', and ``crypto'' and
% run them first with optimization (-O3) then without (-O0) on three different
% computers:
% a 32-bit Linux, a 64-bit Linux, and a 64-bit MacBook. The result is shown in
% table~\ref{t:speed}.
% Generally, we can conclude that the two ISSes are runing at the same speed.
% However, the new ISS performs better on 64-bit machine.
% So we gain not only development cost but also an even better performance.

% %\hide{
% \begin{table}\centering
% \begin{tabular}{l|l|rr|r}
% % \cline{3-4}
% \multicolumn{2}{l|}{~} & \multicolumn{2}{|c|}{ARMv6} & \multicolumn{1}{c}{ARMv5} \\
% \multicolumn{2}{l|}{~} & \multicolumn{2}{|c|}{generated ISS} & \multicolumn{1}{c}{hand-written} \\
% \multicolumn{2}{l|}{~} & \multicolumn{2}{|c|}{speed and relative gain} & \multicolumn{1}{c}{speed} \\\hline
% \multirow{3}{*}{arm32-crypto-O0}  & Linux 64  & 104.78 Mi/s\hspace*{-2.5mm} & +2.6\% & 102.16 Mi/s \\
%                                   & MacOSX    & 89.08 Mi/s\hspace*{-2.5mm}  & +7.4\% & 82.98 Mi/s  \\
%                                   & Linux 32  & 76.74 Mi/s\hspace*{-2.5mm}  &\hspace*{-2.5mm}-10.8\% & 86.03 Mi/s  \\\hline

% \multirow{3}{*}{arm32-crypto-O3}  & Linux 64  & 89.97 Mi/s\hspace*{-2.5mm}  & +2.4\% & 87.89 Mi/s  \\
%                                   & MacOSX    & 74.65 Mi/s\hspace*{-2.5mm}  & +4.6\% & 71.39 Mi/s  \\
%                                   & Linux 32  & 70.91 Mi/s\hspace*{-2.5mm}  & -5.1\% & 74.70 Mi/s  \\\hline

% \multirow{3}{*}{arm32-loop}       & Linux 64  & 124.85 Mi/s\hspace*{-2.5mm} & -1.2\% & 126.38 Mi/s \\
%                                   & MacOSX    & 108.50 Mi/s\hspace*{-2.5mm} & +1.9\% & 106.52 Mi/s \\
%                                   & Linux 32  & 88.89 Mi/s\hspace*{-2.5mm}  & -5.8\% & 94.39 Mi/s  \\\hline

% \multirow{3}{*}{arm32-sorting-O0} & Linux 64  & 82.18 Mi/s\hspace*{-2.5mm}  & -0.5\% & 82.61 Mi/s  \\
%                                   & MacOSX    & 74.40 Mi/s\hspace*{-2.5mm}  & +8.6\% & 68.49 Mi/s  \\
%                                   & Linux 32  & 62.42 Mi/s\hspace*{-2.5mm}  &\hspace*{-2.5mm}-11.3\% & 70.37 Mi/s  \\\hline

% \multirow{3}{*}{arm32-sorting-O3} & Linux 64  & 106.41 Mi/s\hspace*{-2.5mm} & -1.0\% & 107.54 Mi/s \\
%                                   & MacOSX    & 97.51 Mi/s\hspace*{-2.5mm}  & +5.6\% & 92.35 Mi/s  \\
%                                   & Linux 32  & 83.39 Mi/s\hspace*{-2.5mm}  & -1.0\% & 84.27 Mi/s  \\\hline

% \multirow{3}{*}{\em average}      & Linux 64  & 107.26 Mi/s\hspace*{-2.5mm} & +4.1\% & 103.00 Mi/s \\
%                                   & MacOSX    &  92.24 Mi/s\hspace*{-2.5mm} & +4.5\% &  88.27 Mi/s \\
%                                   & Linux 32  &  77.90 Mi/s\hspace*{-2.5mm} & -6.8\% &  83.54 Mi/s \\\hline

% \multicolumn{2}{l|}{\em global average}       &  92.47 Mi/s\hspace*{-2.5mm} & +0.9\% &  91.60 Mi/s \\
% \end{tabular}
% \caption{Comparison of the simulation speeds}
% \label{t:speed}
% \end{table}


%%% Local Variables:
%%% mode: latex
%%% TeX-master: "thesis"
%%% End:

\chapter{Discussion and conclusion}
\label{cpt:concl}

% % JF->XM: general comment: self-defined ---> user-defined

We developed the certification of a part of
an ARM instruction set simulator called \simlight,
using
% the formal representation of the concrete C program
% according to
the operational semantics of the C language provided by the \compcert project.
Correctness proofs were performed under the interactive proof assistant
Coq.
A large part of the Coq specification and of the model of the simulator
were automatically produced from the pseudo-code available in the ARM reference manual.
A Coq proof technique for performing \emph{inversions} was introduced in
order to solve cumbersome proof steps in our work
in a better way than Coq built-in tactics.
Moreover, the size of proof terms generated by our our \hcinv
is much lower than with built-in Coq \inversion,
making Coq type checking and compilation more efficient.
Additionally, we have built a test generator for the ARM instruction decoder,
which generates massive tests covering all ARM instructions.

The following sections contains
an assessment on the usage of operational semantics
in proving the correctness of \simlight and the feasibility of
using this new approach for proving general C programs,
the overall development size of SimSoC-Cert and the TCB.
We conclude with prospects of future work.

\section{Using operational semantics for proving C programs}

The certification technique we applied for \simlight is based on the C operational
semantics provided by \compcert.
The Coq formal representation of the C programs of each ARM instruction
can be obtained from
the instruction pseudo-code intermediate representation AST in two ways:
either by translating it to \compcert C AST,
% JF: the important part of this pretty-printing is more the sharing of types,
% something that  you rightly don't want to explain here...
% moreover it is not part of the comparison since used in the 2 options
% -> let's just forget it
% and pretty-printing it in Coq,
or by translating it to a textual C program,
then parsing it to \compcert C AST using the \compcert C parser.
% and pretty-printing it in Coq.
%
In our experiments, no difference could be observed between the two approaches
-- no information was lost using \compcert C parser.
\compcert C supports a C subset which is rich enough to describe the
operations of ARM instructions.
% % JF -> XM: I see your point but you have to be more careful.
% % OK, we avoid unproved components as far a possible, and
% % the parser is not proved -- it is actually unclear how to get
% % a suitable specification.
% % That said, we are already confident that this parser is good.
% % Here we get additional evidence that this pareser can be trusted
% % but it is not a breakthrough.
% In the future, we can expect other C programs without their own intermediate
% representation AST to use the \compcert C parser directly.

Correctness proofs were performed using the Coq proof assistant.
In this approach to the certification of C programs,
the Coq proof steps in Coq are not simple.
However, we were actually able to consider C programs
having a large size and complex specification,
using the full expressive power of Coq.
%
% % JF: yes we could speak again about ax sem, but we have nothing
% %  more than claims staed inthe intro, since we did not (want to) try.
% than using axiomatic semantics.
%
Our work assesses the feasibility of using operational semantics for
certifying C programs.

Proof steps related to the \compcert C semantics can be simplified
a lot by defining Coq tactics with Ltac (the tactics language).
% % JF: we know we know...
% As an interactive proof assistant, the proof steps require the
% interaction with its user.
Our initial first proof script for ADC instruction contained thousands
of lines of code.  Then, we identified repetitive sequences and
started to define our own proof tactics in the Ltac language,
resulting into much shorter proof scripts.  The second version for ADC
correctness proof was approximately three times smaller than the
first one.
% Repetitive proof steps containing seqences of commands
% can be defined in one general Ltac definition.
In the design of these tactics, we did not seek for generality.
However, since ARM instructions within the same category often have
very similar statements and expressions, our tactics can actually be
reused.
% the user-defined tactics can be reused in all of them.

In Section~\ref{sec:tactic}, we have introduced more general tactics
implemented in SimSoC-Cert, like finding functions in the C memory
model, reusing load/store operations, etc.  Those tactics are not
specific to \simlight, they only deal with \compcert C semantics and
memory operations.  The same holds for our inversion technique: it was
implemented for the needs of SimSoC-Cert as a tactic \hcinv dedicated
to the inductive relations defined in \compcert (see
Section~\ref{ssec:invssc}).  However, these tactics can be reused in
other projects using the same approach to the correctness proof of
\compcert C programs, e.g. the CCCBIP project
which recently started in our group and aims at building a certifying compiler
from a high-level component-based language dedicated to embedded systems
(BIP), with \compcert C as its target. 


% To reason on the C operational
% semantics, \emph{inversions} of the evaluation rules are the essential steps.
% For other projects intend to use \compcert C operational semantics for
% C program certification, e.g. the CCCBIP project, it is possible to
% reuse the tactics defined in SimSoC-Cert.


\section{Hand-crafted inversion}
%%Copy from CPP12 need to be changed 

Our hand-crafted inversion presented in Chapter~\ref{cpt:inv}
was experimented on large proofs relying on big inductive relations
independently defined in the \compcert project.
It played a key role for the success of this approach to correctness proofs
of C programs, and
the extra flexibility provided by \hcinv inversions could be exploited to
produce smaller, more robust and manageable proofs.

It is not yet a fully automatic tactic, like the original \inversion. 
We think that
automation could be realized by interacting with the internals of Coq.
This would be done for efficiency concerns and would not harm
in the cases where the proof can be automatically completed,
or is followed by tactics which do not refer to names produced by inversion.

But in a project with a big size specification like \simsoccert,
where proofs require fine tuning,
interactions between the human and the proof assistant cannot be avoided.
In general, in such situations,
statements involve arbitrarily complex definitions,
so we cannot make the assumption that decision procedures can be used.
The issue is then to provide appropriate mechanisms,
so that writing proof scripts and interacting with the proof assistant
is made easy.
%
We think that our hand-crafted inversion technique is a good tool
in this respect:
it is flexible enough for the user,
practical situations can be managed
with a full control on the script and valuable improvements
of the script are easier to design.


Let us mention another possible application of the technique. Inversion is
sometimes needed to write a function whose properties will be established later
(as opposed to providing a monolithic and exhaustive Hoare-style specification
and along with a VC generator such as Program). In this context, simply using
the proof engine and the inversion tactic tends to generate unmanageably large
terms. We can expect our technique to be very helpful in such situations.

\section{Development size}
\begin{table}[ht]
  \centering
  \begin{tabular}{|l|r@{~}|}
    \hline
    Original ARM ref man (txt)           & 49655 \\
    ARM Parsing to an OCaml AST         & 1068 \\
    Generator (Simgen) for ARM         &   10675 \\
    Generator specifications for SH4      & 737 \\
    General C libraries on ARM         & 1852 \\
    General Coq libraries on ARM         & 1569 \\
    Generated C code for \simlight ARM operations   & 6681 \\
    Generated Coq code for ARM operations   & 2068 \\
    Generated Coq code for ARM decoding  & 592 \\
    Projection   & 857 \\
    Proof script for ADC (2011)    & 3171 \\
    Proof script for ADC (2012)    & 1204 \\
    Definition of \hcinv       &551\\
    Definition of other user-defined tactics      &185\\
%% Xiaomu: please complete
    Proof script for auxiliary functions   & 856 \\
    Proof script for BL (2012)   & 437 \\
    Proof script for LDRB (2012)   & 170 \\
    Proof script for MRS (2012)   & 322 \\
    \hline
  \end{tabular}
  \smallskip
  \caption{Sizes (in number of lines)}
  \label{tab:sizes}
\end{table}

Table~\ref{tab:sizes} shows the size of our development.
The size of the generator is almost the same as the total
number of lines of the generated part for ARMv6.
But note that this is the version redesigned by F. Tuong
in order to be more general,
so that it could be reused with other specific processors.
Currently, besides ARM, it is applied to the SH4 reference manual
where, instead of a specific pseudo-code,
instructions are described using a C syntax.
% There is a good hope that it could be used on other architectures
% \jf{Could be shortened because work by F Tuong.}
% According to different architecture, we will have different size of development.
% But thanks to the experiment of SH4, we are now certain that
% the framework is capable to be used for another processor
% architecture. The only requirement for the object processor is that
% its reference manual should be well structured and can be transformed
% to analyzable text. The most important thing is that it contains formal or
% semi-formal descriptions to be automatically translated to a specific
% intermediate representation.

One can note also that the generated code for the ISS
takes 50\:\% of the Coq formal model,
and almost 70\:\% of the C simulator.
Although the gain may be considered as not that large,
we think that it was worth taking this approach,
given the repetitive nature of instructions.
% Considering the development worthiness,
% the instruction set simulation should be complex enough;
% at least the generated code is more than the specification for such processor.
% As we mentioned in Section~\ref{ssc:arm}, the core of a processor
% simulator is ISS.

About the proof efforts,
the first experiment on the correctness of ADC
%(11 lines of pseudo-code)
costed one month.
The number of Coq lines for the proof script is quite large
(about 3200 for the first version),
especially if we compare with the 11 lines of the corresponding
pseudo-code in the reference manual.
% And the size of the experiment code is quite huge,
% with no optimization or user-defined tactics applied, 3171 loc.
At this stage, we did not develop user-defined tactics.
Now, using \hcinv and other user-defined tactics,
not only maintenability is much improved,
but the development time for a proof is much lower.
Less than one week is needed for an instruction as complicated as ADC.
% Compared to the size of the definition of \hcinv
% and other user-defined tactics,
% the number of lines of code
% of the new version of correctness proof script has been highly
% reduced. And using the new \hcinv with the experiment we obtained,
% the development time has been saved, too.
% To complete a correctness proof of an instruction,
% we need only less than a week,
% which is enough for a quite complicate instruction
% like ADC.
%
% % JF -> XM: you said 12 but I count 11 here!
% % Either the count is wrong, or you forgot an instruction
Until now, 11 instructions were proved correct,
one from each instruction category.
They are given in Table~\ref{tab:prvinst}.
% (the instruction we have proved
% in the category is inside the parenthesis) :
% branch ins truction (BL),
% data processing instruction (ADC),
% Multiply instruction (MUL),
% parallel arithmetic addition and subtraction instruction (QADD16),
% extended instruction (XTAB16),
% miscellaneous arithmetic instruction (CLZ),
% status register accessing instruction (MRS),
% load and store instruction (LDR),
% load and store multiple instruction (LDM),
% semaphore instruction (SWP),
% and exception generating instruction (BKPT).
\begin{table}[ht]
  \centering
  \begin{tabular}{|l|l|}
    \hline
    Category & Instruction name \\
    \hline
    branch & BL \\
    data processing & ADC \\
    multiply & MUL \\
    parallel arithmetic addition and subtraction & QADD16 \\
    extended instruction & UXTAB16 \\
    miscellaneous arithmetic & CLZ \\
    status register access & MRS \\
    load and store & LDR \\
    load and store multiple & LDM \\
    semaphore & SWP \\
    \hline
  \end{tabular}
  \smallskip
  \caption{ARM instructions having a correctness proof}
  \label{tab:prvinst}
\end{table}

\section{Trusted Code Base}

% The trust we may have in our result depends on the faithfulness of its
% statement with relation to the expected behavior of the simulation of
% \texttt{ADC} in \simlight.  It is mainly based on the manually written
% Coq and C library functions, the translators written in OCaml
% described in Section~\ref{sec:overall} (including the
% pretty-printer for Coq), the final phase of the Compcert compiler, and
% the formal definition of $\mathit{proc\_state\_related}$.

Our proofs depend on several tools developed elsewhere:
the Coq proof assistant,
the OCaml compiler and the \compcert C certified compiler.
The TCB of these external tools have to be considered independently.
Regarding Coq, the TCB is essentially its kernel.

Next, the TCB includes the formal version of the ARM reference manual
on which proofs are carried on:
hand-written and automatically produced Coq definitions,
as described in Figure~\ref{fig:arch}.
Alternatively,
automatically produced Coq definitions could be replaced by
the textual reference manual (patched by our bug fixes)
and Coq code generators.
% % JF: no, unless we consider simlight itself, and in that
% %  case we have to add gcc
% the \compcert C AST pretty-printer
% the hand-written Coq, %  and C library functions, %% NO they are proved
% the generated Coq and C representations,
The TCB also includes
the Coq projections from the \compcert C AST representation of \simlight code
to our abstract Coq model.
% Altogether, we have the TCB of
% the correctness statements for relating \simlight to the expected behavior
% as in the formal model.

% % JF -> XM What do you mean? Why the TCB should grow?
% % --> commented out unless something important has to be stated.
% In the future, the development size of correctness proofs will grow
% larger when all the work is done for 147 ARM instructions
% and all the auxiliary functions,
% approximately three times comparing to our TCB.


% \section{Validation}

% \jf{Optionally remove?}

% The validation of new ISS of ARMv6 integrated into SimSoC has been
% introduced in Section ~\ref{ssc:vali}. And a brief speaking on
% Evaluation of Coq specification has been mentioned there, too. In
% general, the C specification can be executed as fast as the older
% hand-written version of ARMv5, but we gain more confidence in the new
% version with the correctness proofs. Although the execution speed is
% very low, the extraction from formal model is still an executable
% and reliable model.

% Using a generator avoids many typo-like errors. However, other kinds
% of errors remain possible.
% Besides the bugs in the documentation which were reported before,
% there are the last bugs we found and fixed
% while trying to boot Linux on the SPEArPlus600 SoC simulator:
% \begin{itemize}
%  \item After the execution of an {\stt LDRBT instruction}, the content
%    of the base register ({\stt Rn}) was wrong. It was due to a bug in
%    the reference manual itself; the last line of the pseudo-code has
%    to be deleted\footnote{This error is fixed in the ARMv7 reference
%      manual, which is now the recommended manual for the ARMv6
%      architecture.}.
%  \item
%    Base register is where the base address stored.  The address may be
%    changed by load/store instructions write-back.  In the operation of
%    such instruction, the write-back to base register should ahead of the
%    certain memory access.  If the memory access fails, the base
%    register then must keep the original value. Such rule is only
%    explained informally.  Our generated ISS manages {\em data aborts}
%    using C++ exception mechanism. As a consequence, moving the
%    statement doing the write-back at the end of the instruction code
%    (and so after any possible {\stt throw}) is sufficient to keep the
%    base register unchanged in case of exception.
%    Some load/store in structure can modify the processor mode; in this
%    case the meaning of base register may be changed. Then write-back
%    must affect a banked version of it instead of current version.
%  \item Additionally, there is a half-word access to an odd address
%    while executing SPEArPlus600 specific code. In this case, the
%    manual indicates that the result is ``unpredictable''. % ...
%  \end{itemize}

\section{Future work}

% % JF -> XM, [GC] general comment:
% % I removed parts wich could be considered as weak or
% % not significant enough.

The next step would be to
extend the work done on \texttt{ADC} and other operations
given in Table~\ref{tab:prvinst} to the full ISS.
% different instruction categories. % JF It is already done!!
% % JF : said above
% The reused parts are the common functions and user-defined tactics,
% which can reduce much of the developing time during proving phase.
% This shorten a lot the current proof script and make it easy
% to understand and to be more generalize.
We are confident that the corresponding work on
the remaining ARM instructions can then be done much faster.
% % JF -> XM : added the following, it would make sense to say
% %  how many lib fun are available / to be done
In particular, a number of lemmas on 14 library functions
are already available.
71 library functions remain to be done.
% % JF: see [GC]
% And it will be also helpful to study the proofs between correctness of
% instructions in the same category.
% The instructions in the same category are constructed
% very similar to each other.
% The common parts then can be
% applied to all instructions in one category.

% Internal functions are described in an informal manner in the
% ARMv6 reference manual.
% No pseudo-code is available for them,
% which means that the corresponding library functions,
% both in the abstract Coq model and in \simlight,
% are written by hand.
% In order to get a suitable \compcert C AST to reason about,
% we use the parser provided in \compcert.
The hand written library functions in \compcert C ASTs are obtained
using the \compcert C parser.
Currently, they are merged with instructions by hand,
and identifiers used in these functions are added manually,
in order to solve a technical issue
stated on page~\pageref{page:libfunast}.
It would be better to build a ``hook'' which automatically finds
the called functions in the parsed ASTs and
generates unused block numbers for the corresponding identifiers.

We also attempted to write a Coq (functional) version of the decoder,
but strong improvements are required to make it usable.
The current version is based on a huge pattern-matching,
%The resulting decoding algorithm is quite weak,
which considers the 32 bits of a binary instruction
in a carefully designed order.
% depending on a roughly concluded decoding order.
% Decoding should be more specific,
% which is one of the very important parts of processor simulator.
We started to design a better version of this decoder,
considering the semantics of bit fields.
% A new version of decoder is planned to decode the binary instruction by the
% bits field separately.
% % [GC] and too technical for a conclusion.
% For example,
% the key bits field $[27:25]$ is $010$ and then
% the instruction must be one of the load/store instruction
% under immediate offset addressing mode;
% or when its value equals to $101$,
% the instruction is a branch instruction.
% After deciding which category it is,
% it should use the other bit field represented parameter
% to decode which concrete instruction it is.
Then, proofs for the decoder could be considered as well
-- automatic extraction tools from the ARM reference manual
are already available.
Finally, the simulation loop
(basically, repeat decoding and running operations) can be be proven.

In another direction, our methodology can be reused on other processors,
such as SH4.

% proof for simlight 2
In the future, \simlight 2 could be considered as well.
\simlight 2 has adopted several optimization methods for a higher simulation
speed. The most important difference is the ``flattening'' method applied
to the instruction set (see Subsection~\ref{sec:codegen}).
Some instructions are merged with their addressing mode,
and the \simlight 2 decoder decodes the instruction and its addressing
mode at the same time. 
Then the C definition is simpler than in \simlight with less function calls.
We expect the proof for this \simlight 2 decoder
to be less difficult than \simlight.
%
Instruction operations in \simlight 2 are essentially the same as for \simlight.
The main optimization used in \simlight 2 
is to specialize some of the parameters according to actually used values.
Therefore, one ARM instruction operation is implemented by
several functions in \simlight 2, instead of just one function in \simlight;
but the code of these functions is essentially the same,
so there is good hope that 
existing correctness lemmas for \simlight could be
restated and generalized in such a way that
instances of them would just be the expected correctness lemmas 
for the corresponding functions in \simlight 2.


% % JF [GC]
% based on automatic
% generation of simulation code and Coq specification for other
% processors.
% The one already considered is SH4.
% In fact, the same approach as the ARMv6 has been followed, and a similar Coq representation can currently be generated from the SH4 manual.
% The other core processor of SimSoC is PowerPC.
% To perform a similar work on PowerPC requires a pre-process replacing using the existing pdf-to-text
% step.
% Because the result of {\stt pdftotxt} is not complete, due to the two column structured documentation, lines are lost or contents
% are disordered.

% % JF : it is not future work
% Besides the correctness proofs on processor instruction operation, the
% other important achievement is building the hand craft version
% \hcinv.
% The technique introduced early in \cite{small_inv} on very small toy examples could be successfully used in a significant application,
% up to suitable extensions in order to conveniently get the premises of
% a constructor in non-absurd cases.
% As in \cite{2013itp}, we do not claim that we have a fully automated tactic, like \inversion.
% Our goal is more modest: providing a hand-crafted inversion technique
% which is flexible enough for the user, so that most practical
% situations can be managed with a full control on the script and
% valuable improvements on robustness.
% Moreover, the extra flexibility provided by hand-crafted inversions can be exploited to produce
% smaller, more manageable proof terms.

% % JF [GC]
% The experiment with SimSoC-Cert correctness proof is introduced in
% Chapter ~\ref{cpt:inv}, which relies on the big inductive relation
% representing the operational semantics of \compcert C defined by
% \compcert project.
Our group recently started another project aiming at
the implementation of certified software written in BIP,
a high-level component-based language dedicated to embedded systems,
with \compcert C as an intermediate target.
We expect the work presented is this thesis to be reused there.
More generally,
our implementation \hcinv dedicated to \compcert
can be re-used in any application of \compcert operational semantics
for proving C programs.
However, it has to be updated accordingly to the new releases of \compcert.

% % JF : it is not future work
% Let us mention another possible application of the technique.
% Inversion is sometimes
% needed to write a function whose properties will be established later (as
% opposed to providing a monolithic and exhaustive Hoare-style specification and
% along with a VC generator such as Program).
% In this context, simply using the proof engine and the \inversion tactic
% tends to generate unmanageably large terms.
% %We can expect our technique could be very helpful in such situations.
% We expect our technique to be very helpful in such situations.

% and the line-count metrics given at the end of \cite{small_inv} makes sense

% reduce developpment time
% ease refactoring a lot


%%% Local Variables:
%%% mode: latex
%%% TeX-master: "thesis"
%%% End:

\include{vf}

\begin{appendices}
\include{appendixADC}
\chapter{Example: the proof script related to instruction ADC}
\label{app:adc}

Here we present Coq code on the main theorem stated for ARM instruction ADC.
There are 3 memory state transitions for the concrete model.
First, from \texttt{m0} to \texttt{m1}, the parameters of
ADC is allocated.
Second, from \texttt{m1} to \texttt{m2}, the parameters are initialized.
From this memory state \texttt{m2}, we are able to build the projection to
the abstract model for processor state \texttt{proc} and other parameters.
Then, from \texttt{m2} to \texttt{mfin}, the statement of ADC function body
is executed.
The new abstract state is \texttt{S.ADC\_step sbit cond d n so (mk\_semstate nil true s)}. It is expected to be related to \texttt{mfin} in the concrete model.

\begin{alltt}

Theorem related_aft_ADC: forall e m0 m1 m2 mfin vargs s out sbit cond d n so,
  alloc_variables empty_env m0 (fun_internal_ADC.(fn_params) ++ 
                                fun_internal_ADC.(fn_vars)) e m1 ->
  bind_parameters e m1 fun_internal_ADC.(fn_params) vargs m2 ->
  (forall m ch b ofs, Mem.valid_access m ch b ofs Readable) ->
  proc_state_related proc m2 e (Ok tt (mk_semstate nil true s)) ->
  sbit_func_related m2 e sbit ->
  cond_func_related m2 e cond ->
  d_func_related m2 e d ->
  n_func_related m2 e n ->
  so_func_related m2 e so ->
  exec_stmt (Genv.globalenv prog_adc) e m2 fun_internal_ADC.(fn_body) 
    Events.E0 mfin out ->
  proc_state_related proc mfin e 
    (S.ADC_step sbit cond d n so (mk_semstate nil true s)). 

\end{alltt}

The proof script for theorem \texttt{related\_aft\_ADC} is too long to
be present here ($\sim 600$ loc).
Instead of showing the whole script,
we choose one of the lemmas used to support the proof
of \texttt{related\_aft\_ADC}: \texttt{same\_copy\_SR}.

Before stating a lemma, 
in order to shorten the proof script and its readability,
we give a name to the expression we focus on for the lemma.

The name \texttt{cp\_SR} is given to the ASTs of C expression:
\begin{alltt}
 copy_StatusRegister(&proc->cpsr, spsr(proc))
\end{alltt}
In this expression, we have two function calls to \texttt{spsr} 
and \texttt{copy\_StatusRegister}.

\begin{alltt}
Definition cp_SR :=
  (Ecall
    (Evalof (Evar copy_StatusRegister T32) T32)
    (Econs
      (Eaddrof
        (Efield
          (Evalof
            (Ederef 
              (Evalof (Evar proc T2) T2) T8)
            T8) cpsr T9) T25)
      (Econs
        (Ecall (Evalof (Evar spsr T33) T33)
          (Econs (Evalof (Evar proc T2) T2)
            Enil) T25) Enil)) T10).
\end{alltt}

The Lemma states that the evaluation results of expression \texttt{cp\_SR} in
the abstract model and the concrete model are equivalent.

\begin{alltt}          
Lemma same_copy_SR :
  forall e m l b s t m' v em,
    proc_state_related m e (Ok tt (mk_semstate l b s)) ->
    eval_expression (Genv.globalenv prog_adc) e m cp_SR t m' v ->
    proc_state_related m' e
      (Ok tt (mk_semstate l b
      (Arm6_State.set_cpsr s (Arm6_State.spsr s em)))).
Proof.
  intros until em. intros psrel cpsr.
  inversion cpsr. subst. rename H into ee,H0 into esrv. unfold cp_SR in ee.
  inv_eval_expr m m'.
  (* Function spsr, get spsr from the current state *)
  apply (same_spsr e l b s vf0 fd0 m vargs0 t10 m3 vres0 l b s)
    in psrel; [idtac|exact Heqff0|exact ev_funcall].
  (* Function copy_StatusRegister, copy the current spsr to cpsr*)
  apply (same_copy e l b s vf fd m3 vargs t2 m' vres l b
    (Arm6_State.set_cpsr s (Arm6_State.spsr s em))) in psrel;
  [idtac|exact Heqff|exact ev_funcall0].
  exact psrel.
Qed.
\end{alltt}




%%% Local Variables: 
%%% mode: latex
%%% TeX-master: "thesis"
%%% End: 

\end{appendices}

% \include{8/materials_methods}




% --------------------------------------------------------------
%:                  BACK MATTER: appendices, refs,..
% --------------------------------------------------------------

% the back matter: appendix and references close the thesis


%: ----------------------- bibliography ------------------------

% The section below defines how references are listed and formatted
% The default below is 2 columns, small font, complete author names.
% Entries are also linked back to the page number in the text and to external URL if provided in the BibTex file.

% PhDbiblio-url2 = names small caps, title bold & hyperlinked, link to page
%\begin{multicols}{2} % \begin{multicols}{ # columns}[ header text][ space]
%\begin{tiny} % tiny(5) < scriptsize(7) < footnotesize(8) < small (9)

%\bibliographystyle{Latex/Classes/PhDbiblio-url2} % Title is link if provided
%\renewcommand{\bibname}{References} % changes the header; default: Bibliography

% adjust this to fit your BibTex file

%\end{tiny}
%\end{multicols}

% --------------------------------------------------------------
% Various bibliography styles exit. Replace above style as desired.

% in-text refs: (1) (1; 2)
% ref list: alphabetical; author(s) in small caps; initials last name; page(s)
%\bibliographystyle{Latex/Classes/PhDbiblio-case} % title forced lower case
%\bibliographystyle{Latex/Classes/PhDbiblio-bold} % title as in bibtex but bold
%\bibliographystyle{Latex/Classes/PhDbiblio-url} % bold + www link if provided

%\bibliographystyle{Latex/Classes/jmb} % calls style file jmb.bst
% in-text refs: author (year) without brackets
% ref list: alphabetical; author(s) in normal font; last name, initials; page(s)

%\bibliographystyle{plainnat} % calls style file plainnat.bst
% in-text refs: author (year) without brackets
% (this works with package natbib)


% --------------------------------------------------------------

% according to Dresden med fac summary has to be at the end
%\include{0_frontmatter/abstract}

%: Declaration of originality
%\include{declaration}



\nocite{*}
\bibliographystyle{abbrv}
\bibliography{biblio}


\end{document}
