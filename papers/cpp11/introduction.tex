\section{Introduction}
\label{sec:intro}

\subsection{Simulation of Systems-on-Chip}

Systems-on-Chip (SoC), used in devices such as smart-phones, contain both
hardware and software. A part of the software is generic and can be used with
any hardware systems, and thus can be developed on any computer. In contrast,
developing and testing the SoC-specific code can be done only with this SoC, or
with a \emph{software executable model} of the SoC. 
To reduce the time-to-market, the software development must start
before the hardware is ready. Even if the hardware is available,
simulating the software on a model provides more debugging
capabilities.
%
The %most abstract and 
fastest simulators use native simulation.
%
The software of the \emph{target} system (i.e., the SoC) is compiled with
the normal compiler of the computer running the simulator, 
but linked with special system
libraries. Examples of such simulators are the Android and iOS
SDKs.

In order to develop low-level system code, one needs a simulator that
can take the real binary code as input. Such a simulator requires a
model of the processor and of its peripherals (such as UART, DMA,
Ethernet card, etc). When simulating a smart-phone SoC, this kind of
functional simulator can be from 1 to 100 times slower than the real
chip.
%
These simulators have other uses, for example, as
reference models for the hardware verification.
An error in the simulator can then mislead both the software and the hardware
engineers.
QEMU~\cite{qemu} is an open-source processor emulator
coming with a set of device models; it can simulate several
operating systems. Other open-source simulators include
UNISIM~\cite{unisim} (accurately-timed)
and
SimSoC~\cite{ossc09}, developed by some colleagues, %at LIAMA 
which is loosely-timed (thus faster).
Simics~\cite{simics} is a commercial alternative.
%
The usual language to develop such simulators is C++, combined with
the SystemC~\cite{systemc-lrm} and OSCI-TLM~\cite{tlm-osci}
libraries.

\medskip
The work reported here is related to SimSoC.

\subsection{The Need for Certification}

Altogether, a functional simulator is a complex piece of software.
SimSoC, which is able to simulate Linux both on ARM and PowerPC
architectures at a realistic speed
(over 10 Millions of instructions per second per individual core), 
includes about 60,000 lines of C++
code. The code uses complex features of the C++ language and of the
SystemC library. Moreover, achieving high simulation speeds requires
complex optimizations, such as dynamic translation~\cite{qemu}.

This complexity is problematic, because beyond speed,
\emph{accuracy} is required:
all instructions have to be simulated
exactly as described in the documentation.
There is a strong need to strengthen the confidence that
simulations results match the expected accuracy. 
Intensive tests are a first answer.
For instance, as SimSoC is able to run a Linux kernel
on top of a simulated ARM, we know that many situations are covered.
However it turned out, through further experiments, that it was not
sufficient: wrong behaviors coming from rare instructions
were observed after several months.
%
Here are the last bugs found and fixed by the SimSoC team while trying to boot
Linux on the SPEArPlus600 SoC simulator.
\begin{itemize}
\item After the execution of an \texttt{LDRBT} instruction, the contents of the
  base register (\texttt{Rn}) was wrong. It was due to a bug in the reference
  manual itself; the last line of the pseudo-code has to be deleted.
\item After a data abort exception, the base register write-back was not
  canceled.
\item Additionally, a half-word access to an odd address while
  executing some SPEArPlus600 specific code was not properly handled. 
\end{itemize}

Therefore we propose here to certify the simulator,
that is, to prove, using formal methods --
here: the Coq proof assistant \cite{coqmanual,coqart} --
that it conforms to the expected behavior.

This is a long term goal.
Before going to the most intricate features of a simulator such as SimSoC,
basic components have to be considered first.
We then decided to focus our efforts on a sensible and important 
component of the system: the CPU part of the ARMv6 architecture 
(used by the ARM11 processor family).
%
This corresponds to a specific component of the SimSoC simulator,
which was previously implementing the ARMv5 instruction set only.
Rather than certifying this component, it seemed to us more feasible
to design a new one directly in C, in such a way that it can be
executed alone, or integrated in SimSoC (by including the C code in
the existing C++ code).
%
We call this new component \simlight \cite{rapido11}. Combined with a small
\texttt{main} function, \simlight can simulate ARMv6 programs as long
as they do not access any peripherals (excepted the physical memory)
nor coprocessors. There is no MMU (Memory Management Unit) yet. Integrating it in SimSoC just
requires to replace the memory interface and to connect the interrupts
(IRQ and FIQ) signals.

\smallskip
The present paper reports our first efforts 
towards the certification of \simlight.
We currently have a formal description of the ARMv6 architecture,
a running version of \simlight,
and we are in the way of performing correctness proofs.
The standard way for doing this is to use Hoare logics
or a variant thereof.
Various tools exist in this area, 
for example Frama-C \cite{frama-c}.
We chose to try a more direct way,
based on an operational semantics of C;
more precisely the semantics of Compcert-C defined in the Compcert project
\cite{Leroy-Compcert-CACM}.
One reason
is that we look for a tight control on the formulation of proof obligations
that we will have to face.
Another advantage is that we can consider the use of
the certified compiler developed in Compcert,
and get a very strong guarantee on the execution of 
the simulator (but then, sacrificing speed to some extent\footnote{%
According to our first experiments, \simlight compiled with Compcert
is about 50~\% to 70~\% slower than \simlight compiled with \texttt{gcc -O0}.
%It is indeed interesting : by running natively without any simulation, 
%we got a faster executable with Compcert~!
}).

Another interesting feature of our work is that 
the most tedious (hence error prone) part of the formalization --
the specification of instructions --
is automatically derived from the reference manual.
It is well known that the formal specification of such big applications
is the main weak link in the whole chain.
Though our generators cannot be proved correct, because the statements
and languages used in the reference manual have no formal semantics,
we consider this approach as much more reliable than a manual formalization.
Indeed, a mistake in a generator will impact several or all operations,
hence the chances that it will be detected through a visibly wrong behavior
are much higher than with a manual translation,
where a mistake will impact only one (eventually rarely used) operation.

Note that after we could handle the full set of ARM instructions,
our colleagues of the SimSoC team decided to use the same technology
for the SimSoC itself:
the code for simulating instructions in \simlight,
i.e., the current component dedicated to the ARM v6 CPU in SimSoC,
is automatically derived using a variant of our generator,
whereas the previous version for ARM v5 was manually written \cite{rapido11}.


Fig.~\ref{fig:archi} describes the overall architecture. 
The contributions of the work presented in this paper are 
the formal specification of the ARMv6 instruction set
and the correctness proof of a significant operation.
More precise statements on the current achievements 
are given in the core of the paper.
 
\smallskip
\emph{Related Work.}
A fully manual formalization of the fm8501 and ARMv7 architectures are reported
in \cite{fm8501} and \cite{FoxM10}. 
The formal framework is respectively ACL2 and HOL4 instead of Coq, 
and the target is to prove that the hardware or microcode implementation of 
ARM operations are correct wrt the ARM specification.
Our work is at a different level:
we want to secure the simulation of programs using ARM operations.
% and no specific target is considered, whereas we are driven by the needs of
% simulation of ARMv6 and 
Another major difference is the use of automatic generation 
from the ARM reference manual in our framework,
as stated above.

\medskip
The rest of the paper is organized as follows.
Section \ref{sec:overallarchi}
presents the overall architecture of \simlight
and indicates for which parts of \simlight 
formal correctness is currently studied.
A informal statement of our current results is also
provided there.
Sections \ref{sec:armmodel} and \ref{sec:simlight}
present respectively our Coq formal reference model of ARM
and the (Coq model of) Compcert-C programs targeted for correctness.
A precise statement of our current results and indications on the proofs are 
given in Section \ref{sec:results}.
We conclude in Section \ref{sec:conclusion} with some hints on
our future research directions.
Some familiarity with Coq is assumed in 
Sections \ref{sec:armmodel}, \ref{sec:simlight} and~\ref{sec:results}.


%%% Local Variables: 
%%% mode: latex
%%% TeX-master: "cpp"
%%% End: 
