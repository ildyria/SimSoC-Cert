% \hide{
%    In general, patch functions are in sl2_patch.ml, and generation
%    functions are in this file.
   
%    At the beginning, we have one list of progs (cf ast.ml) and another list of
%    coding tables (cf codetype.ml). There is a third list describing the ASM
%    syntax (cf syntaxtype.ml). Each list item describes either an instruction or
%    an addressing mode case.
   
%    - MSR is splitted in MSRreg and MSRimm (cf file norm.ml)
%    - the pseudo-code is normalized (cf file norm.ml)
%    - the address write-back is disabled in the M2, M3, and M4 addressing modes
%    - the 3-uplets <instruction, addressing mode case, syntax> are flattened
%    (cf file flatten.ml)
%      o during flattening, some patches are applied:
%        patch_W, patch_SRS, patch SRS_RFE

%    From this point, we manipulate "flat programs" (cf flatten.ml). The
%    notion of addressing mode has disappeared.

%    - We remove the Thumb instruction MOV(3), because it is identical to CPY
%    - Where possible, we swap the conjunctions so that the CP15 U bit is tested
%      after the alignment. This is an optimization try, maybe useless.
%    - We improve the instructions that use the coprocessor
%      (+ disable LDC and STC).
%    - We fix a problem about "address of next instruction".
%    - We compute the list of parameters and variables used by the pseudo-code
%    - We replace some sub-expressions by "computed parameters"
%    - We remove the ConditionPassed tests
%    - We insert the writebacks at the end of the instructions
%      (previously removed from the addressing mode cases)
%    - We specialize the instructions for some boolean parameter values
%    - We compute the list of instruction groups. All instructions in a group
%      have the same list of parameters
   
%    From this point, we manipulate extended programs; i.e., flat program with 
%    symbol tables.

%    - We create the unconditional variants of the conditional instructions

%    Now, the code generation can start.

%    - We generate the instruction type and sub-types.
%    - We generate the tables containing, among other things, the instruction
%      names
%    - We generate the may_branch function
%    - We generate the 4 decoders: {thumb, arm32} x {decode_and_store,
%      decode_and_exec} (cf sl2_decoder.ml)
%    - We generate a small program, which is used only to control the size of the
%      instruction type
%    - We generate (part of) the ARM to LLVM translator
%    - We generate the ASM printers. We need 2 versions:
%      o one C version using fprintf and FILE*
%      o one C++ version using streams 
%    - We generate the semantics function (cf sl2_semantics.ml).
%      We need 2 versions:
%      o one version with an expanded list of atomic arguments
%      o one version taking an SLv6_Instruction* as argument
% }



%other generated info

%other hand-written specifications


% The two main components of a processor simulator are then:
 
% \begin{itemize}
% \item
%   The decoder, which decode from given a binary word, retrieves the
%   name of an operation and its potential arguments in assembly code.
%   In Section \ref{ssc:dec} we will explain how it is generated from
%   reference manual.
% \item
%   The precise description of transitions is the operation of instruction.
%   The definition contains operations on processor registers and memory,
%   thereby the processor state is changed.
%   In the ARMv6 reference manual, these algorithms are written in
%   ``pseudo-code`` which calls low-level primitives. For example,
%   some code indicates setting a range of bits of a register by a given
%   value. And there exists operations might lead the system to undefined
%   or forbidden situation. In ARM processor, this is called ``UNPREDICTABLE``
%   state. When the simulator meets these result, it is then stopped with
%   a message indicated the situation.
%  \hide{The precise description of transformations performed by an operation
%  on registers and memory.  In the reference manual of ARM, this is
%  defined by an algorithm written in “pseudo-code” which calls
%  low-level primitives for, e.g., setting a range of bits of a register
%  to a given value.  Some situations are forbidden or left
%  unspecified. For ARM processors, this results in a so-called
%  ``UNPREDICTABLE`` state. The best choice, for a simulator, is then to
%  stop with a clear indication of what happens.}
% \end{itemize}

% \subsection{Formal specification model of ARMv6}
% \label{ssec:fsmoa}

% The formal specification keep as close as the reference manual, and
% make it as simple as possible. In this way, we can obtain a trustable
% formal model and simple enough to reason about. Such formal model
% contains the basic library for integer representation and binary
% operations, memory model, main processor system control co-processor,
% the instruction set, the simulation looper, and also files to describe
% the initial simulation configuration.  As we explained before, except
% the instruction set, all the other descriptions are manually
% defined. And the bit vector definition and opertions reuse the integer
% module from \compcert. From the \compcert integer specification, we
% extend it to describe the 4-bit register number, 30-bit memory

% Each operation O from the reference manual is mechanically translated
% to a cor- responding Coq function named O\_Coq. First we define a type
% state, which is a record with two fields Proc and SCC (System Control
% Coprocessor) containing respectively the components related to the
% main processor (status register CPSR, SPSR, other registers...) and
% the corresponding components related to the copro- cessor, as well as
% the ARM memory model. Then we use a monadic style [14] in order to
% take the sequentiality of transformations on the state into
% account. Beyond the state st, two other pieces of information are
% handled: loc, which represent local variables of the operation and bo,
% a Boolean indicating whether the program counter should be incremented
% or not; they are registered in the following record which is used for
% defining our monad:

% \begin{alltt}
% (* A4.1.2 ADC *)
% Definition ADC_step (S : bool) (cond : opcode) (d : regnum) (n : regnum) (shifter_operand : word) : semfun _ := <s0>
%   if_then (ConditionPassed s0 cond)
%     ([ <st> set_reg d (add (add (reg_content s0 n) shifter_operand) ((cpsr st)[Cbit]))
%     ; If (andb (zeq S 1) (zeq d 15))
%        then         (<st> if_CurrentModeHasSPSR (fun em =>
%            (<st> set_cpsr (spsr st em))))
%        else         (if_then (zeq S 1)
%            ([ <st> set_cpsr_bit Nbit ((reg_content st d)[n31])
%            ; <st> set_cpsr_bit Zbit (if zeq (reg_content st d) 0 then repr 1 else repr 0)
%            ; <st> set_cpsr_bit Cbit (CarryFrom_add3 (reg_content s0 n) shifter_operand ((cpsr st)[Cbit]))
%            ; <st> set_cpsr_bit Vbit (OverflowFrom_add3 (reg_content s0 n) shifter_operand ((cpsr st)[Cbit])) ])) ]).
% \end{alltt}


% With a concrete instruction above, we explain how the formal
% specification of a state transition functiones. Through the left line
% of the generation chain we can machinically produce the Coq
% specification from original operation pseudo-code~\ref{ps:adc}
% \coqdocvar{ADC}.  Most of the instruction operations have argument
% \coqdocvar{cond} to check the current condition. Most of the ARMv6
% instructions can be executed conditionally, but a few executes
% unconditionally.  For conditionally executed instruction, the
% operation will be skipped if the condition is not fit.  The other
% arguments are: \coqdocvar{S} is a bit indicates if the instruction
% updates CPSR; \coqdocvar{Rn} is a register contains the first operand;
% \coqdocvar{Rd} is the destination rigister; \coqdocvar{shift\_operand}
% specifies the second operand, which here is the date-processing
% operands of the first addressing mode. And if the bit$[25]$ is $0$ and
% the bit$[7]$ and bit$[4]$ are both 1, then the instruction is no
% longer ADC but one in the exextension of instruction set.

% Note that in general, every \coqdocvar{O\_Coq}
% functions terminate with \coqdocvar{Ok} as value. However for 
% ``{\it UNPREDICTABLE}``
% states for example, errors are implicitely propagated with our monadic
% constructors for exceptions : Ko and Todo.  We present now the
% translation of ADC pseudocode. To this effect, we intro- duce get st,
% a monadic function giving access to the current state st in its body,
% represented by this notation:Each operation O from the reference
% manual is mechanically translated to a cor- responding Coq function
% named O\_Coq. First we define a type state, which is a record with two
% fields Proc and SCC (System Control Coprocessor) containing
% respectively the components related to the main processor (status
% register CPSR, SPSR, other registers...) and the corresponding
% components related to the copro- cessor, as well as the ARM memory
% model. Then we use a monadic style [14] in order to take the
% sequentiality of transformations on the state into account. Beyond the
% state st, two other pieces of information are handled: loc, which
% represent local variables of the operation and bo, a Boolean
% indicating whether the program counter should be incremented or not;
% they are registered in the following record which is used for defining
% our monad:

% \hide{ Let us illustrate this on a concrete example.  Here is the
%   original pseudo-code of the ADC (add with carry) operation of ARMv6.
%   As most operations of ARM, this operation has an argument called
%   cond which indicates whether the operation should be skipped or
%   not. CPSR (Current Program Status Register) and SPSR (Saved Program
%   Status Register, used for exception handling) are special registers
%   related to execution modes of ARM; they also contain flags (N, Z, C
%   and V) relevant to arithmetic instructions. The instruction has four
%   parameters: S is a bit which specifies that the instruction updates
%   CPSR, Rn is a register for the first operand, Rd is the destination
%   register, shifter operand specifies the second operand according to
%   a (rather complicated) addressing modes.  } 

% The monadic function \coqdocvar{get\_st} provides method to access the current state,notated
% as $<st>$, at any place of the operation sequences:

% %Definition semfun A := semstate -> @result A.  
% %\end{alltt}

% Note that in general, every O\_Coq
% functions terminate with Ok as value. However for “UNPREDICTABLE”
% states for example, errors are implicitely propagated with our monadic
% constructors for exceptions : Ko and Todo.  We present now the
% translation of ADC pseudocode. To this effect, we intro- duce get st,
% a monadic function giving access to the current state st in its body,
% represented by this notation:

% \begin{alltt}
% Notation "’<’ st ’>’ A" := (\_get\_st (fun st => A))
%   (at level 200, A at level 100, st ident).
% \end{alltt}

% \subsection{C model of ARMv6}
% \label{ssec:sma}

% The C specification continues to use the data structure for ARMv5 processor,
% because there is nothing changed on processor character between the two version.
% The processor state contains the most important members in the ARM model: a pointer to
% where stores the structure representing Memory Management Unit (MMU), 
% a status register structure for CPSR, an array for status register structure of SPSR,
% a structure for CP15 (SCC), a field of processor id (useful when there is more than one
% core), six array for registers of each processor mode, one field for PC, and the last
% boolean value for \texttt{jump} which indicate whether the instruction modifies the PC.
% In order to gain high speed for our simulator, the processor state has been designed
% complex like this.
% There are several fields redundanted, for example, the PC field is a pointer
% to the 15th register in user register array. Because, the PC field is a significant
% value to judge whether the execution is branched or not, in order to split the running
% program into code block.
% And the PC field will be called many times during execution.
% This part is related to a very important optimization method
% in \simsoc simulation ~\ref{sec:simsoc}. 
% The processor mode is not placed as a value directly under processor state data 
% structure. It is inside the structure for status register.
% And there is another way to represent what is the current processor mode,
% by looking at which register array we are using.
% For each processor mode there is a corresponding register array.
% This is because the simulator never use mode as a condition, it can go straight to the
% field it want to look for to save time.
% Contrast, the status register is not a simple 32-bit integer. It is also a data
% structure of every significant bit of the register. The bits $[31:28]$ are the famous
% N, Z, C, V(Negative, Zero, Carry and oVerflow) flags for condition code. 
% These flags in the Current Program Status Register (CPSR) are tested by most of the 
% instructions to determine wether the instruction 
% The condition code flags in the CPSR can be tested by most instructions to
% determine the instruction is going to be operated or skipped.
% All the other bits are seen as the reserved or \texttt{background}.
% Because they are reserved for future extansion and be read as zero, any written on them
% should be ignored. For future compatibility, they must be written with the same value
% from the same bits.
% The generated instructions have the similar shape of the following example of ADC
% instruction:


% Most of the instruction begins with the condition determination by asking
% the current condition, then a series of operations. 

% To be precise, \simlight (Version 1) is in \compcert C language, the
% large subset of C language.  Then the whole simulator is compiled
% under the certified compiler \compcert.  In this way, the erroneous
% from C compiler can be avoided and the compilation result is also
% reliable.  Generating directly \compcert C code makes sure that none
% unaccepted code has been ignored by the compiler.


% \section{Other faclilities}
% \subsection{Decoder}
% \label{ssc:dec}
% The instruction part of ARM reference manual is well structured, in which the
% two elements can be used as the input of the automatic generation chain. The
% first is the operation description we have introduced. The second is the 
% instruction encoding table. It summarized all possiblility of this kind of
% instruction 32-bit representation. All others will be decoded into unpredictable
% or undefined.
% Using the encoding table, it is able to build a intuitional ARM instruction
% decoder of ARMv6 architecture.
% The main body of the decoder is a big pattern match. Each constructor
% represents by 32 bits, either implicit or explicit.
% To define a functional ARM instruction decoder, the first thing to do is to 
% decide the matching order.
% Why there is this matching order. Because the pattern matching in Coq is ordered
% top to bottom, the value belongs to the $i$th constructor if and only if it is 
% not in any case of the previous one, otherwise the Coq type checker will
% consider such case is redundant. 
% The 147 instructions can be first grouped into two, conditional and
% unconditional.
% For ARM instructions, the condition field \texttt{cond} $[31:28]$ indicates
% the conditional execution of ARM instruction. 
% These four bits describe different meaning when they are set or cleared.
% And instruction under such condition has to be executed with the corresponding
% CPSR (Current Program Status Register) in which N, Z, C, V flags statisfy
% the given condition. If not so, the instruction will have no effect on processor
% state and go to the next instruction without notification.
% A small mount of instruction can only be executed unconditionally.
% These instruction will be checked first by matching the first four bits with
% $0b1111$ representing unconditional exection.
% The rest are grouped by ARM instruction categories. The instruction under the
% same category will not cover each other by the wildcard in general.
% And we also decide 5 level of order to group the conditional instructions.
% \begin{itemize}
% \item
% All the multiply instruction without an accumulator. They can be covered by
% the similar multiply instruction with accumulator. The instructions without
% accumulator fill the bits $[15:12]$ with $0b1111$ where the instruction with
% accumulator using them to refer to the register for accumulator operand.
% \item
% Some instructions from ARMv5 architecture use the notion \texttt{SBO} or
% \texttt{SBZ} to say that the instruction bit is read as one/zero whatever the
% value of the bit is and it can not be rewrite.
% These instruction need to be checked then, otherwise it is possible to hide by
% some of the new instruction in ARMv6.
% \item
% A few load/store instructions work under previlige mode of processor.
% They have two significant bit $P$ and $W$ assigned to special combination of value
% to indicate this kind of instructions.
% We have to put them in higher priority before
% the similar instruction working for the other processor modes to prevent the
% decoder from hiding them.
% \item
















% Instructions load/store from memory with format other than \texttt{word}.
% They have the similar shape as the load/store with \texttt{word} too,
% but use four bits $[7:4]$ refer to the load/store length, whether it is a
% \texttt{half}, \texttt{double word} or a \texttt{signed byte}.
% \item
% The last group contains all the instruction operations with addressing modes.
% Decoding this kind of instructions has to call the decoder for addressing mode
% too.

% For the executing version of formal simulator, we did not integret the extracted the Ocaml code
% from the decoder Coq specification. Because the extraction principle will
% translate a Coq pattern-matching which matches more than one terms into a
% complicated one separating the terms and match them one by one.
% Then during the compilation and excution,
% a lot of time will be spend on such decoder.
% A more efficient way is to generate independently a decoder in Ocaml code using
% the similar generation stradegy as the Coq decoder generator.
% A same affect is also done for C representation of \simlight.
% \end{itemize}

% For C representation decoder, we do not use such sequential algorithm.
% It lowers too much of the execution speed. Instead, a decoder based on
% {\it switch} is in used. It is also generated automatically
% using the same generating stradegy, and be put in
% the same file as instructions set descriptions in C language
% with cases in {\it switch} statement.


% In order to validate the \simsoc decoder,
% we built an automatic test generator that generates all possible instructions
% which exclude undefined and unpredictable ones.
% We generate two files. The first contains the instruction binary, in the ELF
% format.
% Another file which
% contains the same tests in assembly code is generated as well.
% It is going to be compared with the result of decoding the binary tests.
% Both files are generated according to the instruction encoding and syntax
% as extracted from the reference manual.

% The parameter values are chosen with respect to the validity constraints to
% ensure that the instruction is defined and predictable.
% For example, the parameters of the {\stt ADC} instruction (see
% Fig.~\ref{fig:flatten}) are {\stt Rd}, {\stt Rn} and {\stt shift\_imm}.
% Binary instructions are producted with different combinations of values for them.
% From reading the \texttt{Syntax} and \texttt{Usage} part of each instruction,
% we know there are several validity constraints for some instructions.
% Some validity constraints are dealt with during the parameter generation.
% For example, register {\stt Rn} in instruction {\stt LDRBT} cannot be
% {\stt PC} ({\stt R15}),
% then we chose directly a value between 0 and 14 to be assigned to {\stt Rn}.
% Some other validity constraints involving two or more
% parameters at the same time.
% Continuing the example of {\stt LDRBT}, another constraint states
% that {\stt Rd} and {\stt Rn} must be different:
% the generator must produces two different values
% and assigns them to {\stt Rd} and {\stt Rn}.
% Similarly, we generate the corresponding assembly code. Under each encoding
% table in the reference manual, the \texttt{Syntax} part explain the syntax
% of the instruction, the instruction identifier and the same parameters as in
% the encoding table.
% We use the generated binary instruction as input for our decoder.
% And it will output the decoding result in assembly code.
% Then using the Unix command {\stt diff}, it is able to compare the decoder
% results and the assembly tests. Minor issues have been detected and fixed in this
% way.

% \subsection{Experimental result and validation}
% \label{ssc:vali}

% With the functional defined semantics, it is possible to extract the whole Coq
% specified ARMv6 simulator into ocaml code and compile it to an executable
% binary. And it is also interesting to see tests running on a formal specified
% simultor.
% We provide the tests for the C specified Simlight.
% In the mean time, we use these tests for our formal specification.
% On one side, we use the {\stt arm-elf-gcc} compiler to
% compile the C tests into ELF files to be used in Simlight. On the other side, we
% write an easy translator then to translate the tests to Coq representation then
% extract to ML files.
% These small tests can be executed on the formal simultor in a sufficient
% simulation speed, less than 1000 instructions per second.
% Running an simple direct sum test costs around five minutes.
% A sorting test then costs almost one day to finish, which in C specification it
% only takes 0.78 second.
% For such formal specification we do not care the execution speed,
% it is aim to perform formal proofs for the simulation correctness. 



%%% Local Variables: 
%%% mode: latex
%%% TeX-master: "thesis"
%%% End: 
