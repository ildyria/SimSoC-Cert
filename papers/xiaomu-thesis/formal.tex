\chapter{Formal model of ARMv6}
\label{cpt:formal}

% \jf{Short introduction in 2 or 3 sentences}
In the beginning of this chapter,
we present a short introduction of the ARMv6 reference manual,
% XM
% It helps to understand which parts we have to specify in our model;
% [The line is right but can only be written by a reviewer!]
% the quantity and the complexity of the specification work.
% JF
with an emphasis on the parts we have to specify in our model.
Next we present our formal Coq model of ARMv6:
the main types, how instructions are formalized,
and the ARM instruction decoder.

\section*{Résumé}
\selectlanguage{french}
\begin{resume}
Ce chapitre commence par une introduction au manuel de référence de l'ARMv6,
qui sert de point de départ de notre travail.
Nous insistons plus particulièrement sur les parties que nous avons
formalisé en Coq.
Nous présentons ensuite notre modèle formel Coq de l'ARMv6:
les types principaux permettant de décrire l'état du processeur, 
la façon dont les instructions sont formalisées,
et enfin le décodeur.
Nous terminons par quelques remarques sur nos tentatives d'utilisation
de ce modèle comme spécification exécutable.
\end{resume}

\selectlanguage{english}

\section{The ARM reference manual}
\label{sec:ref}
% \jf{2-3 pages. The reader should understand the complexity of your work.
% \\%
% Ideal situation = they provide a formal model.
% Not the case.
% So make one from the ref man.
% \\%
% Similarly Compcert designed a formal model of C according
% to the informal ISO-C standard (check).
% \\%
% Explain the structure of ARM ref man.
% State = registers + memory + ...
% \\%
% Library of basic fct
% \\%
% 3 addressing modes, 11 operand modes, 15 condition modes, 2 post-operation modes
% \\%
% Instructions : encoding, pseudo-code, english
% \\%
% $\rightarrow$ Big number of possible instructions
% }

\newcommand{\state}{\textit{state}\xspace}
\newcommand{\armst}{\state}
\newcommand{\bit}[1]{\texttt{#1}\xspace}
\newcommand{\inst}[1]{\texttt{#1}\xspace}
\newcommand{\adc}{\inst{ADC}\xspace}
\newcommand{\armkw}[1]{\texttt{#1}\xspace}
\newcommand{\unpred}{\armkw{UNPREDICTABLE}\xspace}
\newcommand{\armvar}[1]{\emph{#1}}
\newcommand{\armrf}[1]{\texttt{#1}\xspace}
\newcommand{\coqcode}[1]{\texttt{#1}\xspace}
In order to certify the ARMv6 simulator, first we need to have a formal
model that can be referred to.
% XM
% We would expect the ARM provider provides a formal model,
% JF
In an ideal world, the ARM company would provide a formal model of their processors,
but it is not the case...
In fact, the only basis we can depend upon to obtain an ARMv6 formal model
is their reference manual~\cite{arm6refman}.
Similarly, \compcert project~\cite{ccc} designed their \compcert C language and Asm language
according to the informal ISO C 90 standard document~\cite{C90} and relevant parts of
the reference manuals for PowerPC, ARM, and IA32.

% JF: removed the next sentence which is obvious
% To understand how to design the formal model of ARMv6, we have to first understand its reference manual.
The ARMv6 reference manual
is structured in four main parts, {\bf CPU architecture}, {\bf Memory and system architecture},
{\bf Vector floating-point architecture} and {\bf Debug architecture}.
The useful contents for us to build our formal model is taken in the CPU architecture part.
There is another important section at the end of the document,
the {\bf Glossary},
which gives the detailed explanation of key words appearing in the
document using formulas and English.

The {\bf CPU architecture} part introduces the ARM programmers' model, the ARM instruction set,
the ARM addressing modes, and the Thumb instruction set.
The contents of the programmers' model helps to formalize a \armst representing the
structure of the ARMv6 processor.
Most of the contents are written in English.
% JF: removed the next sentence which is obvious
% Then we need to abstract this \armst by understanding the document.
The ARM processor state can be illustrated as in Figure~\ref{fig:armst}\,.

\begin{figure}[h]
\centering
%$state_{proc}\!=\textit{general-purpose}~registers\times status~registers\times exceptions\times processor~modes$
$\state_{proc} = \left\{
  \begin{array}{l}
    \textit{general-purpose~registers} ~\times\\
     \textit{status~registers} ~\times\\
     \textit{exceptions} ~\times\\
     \textit{processor~modes}
   \end{array}
 \right.$

\medskip

$\state_{scc}=\textit{registers}~\times~\textit{memory}$

$\state_{ARMv6}=\state_{proc}~\times~\state_{scc}$
\caption{ARM processor state}
\label{fig:armst}
\end{figure}

\sloppy
% JF "is contructed by" --> "contains";
% Rule of thumb: use simple words unless you are really sure of what you write,
% e.g. you found significant instances on google; is it the case here?
% Actually "is contructed by" may work but I'm not sure
% -- and I'm not confident in your English, sorry...
The ARM main processor contains thirty-one 32-bit general-purpose
registers including the program counter, and six 32-bit status
registers.  A particularity of ARM architecture is that the program
counter, register 15, can be used as any other general-purpose
registers (e.g one can XOR the program counter with another register...)
But it has many instruction-specific effects on instruction
execution.  If the program counter is used in a way that does not obey specified
restrictions, the instruction will yield to an \unpred state.  When \unpred
state is reached, the instruction results cannot be relied upon, but the system
will not halt or raise exception:  \unpred is part
of the system.

Access to the registers is decided by the current processor mode.  The
processor mode is encoded in 5 bits of the Current Program Status
Register (CPSR), which is accessible in all processor modes (user
mode, FIQ mode, IRQ mode, supervisor mode, abort mode, undefined mode,
system mode), see Figure~\ref{fig:armpm}.  Other than CPSR, the other
five status registers are the Saved Program Status Registers (SPSR)
corresponding to each mode.  When internal or external sources
generate exceptions, the processor will react as follows.  The
processor status in CPSR will first be preserved into an SPSR
according to the type of the exception.  The processor mode is switched
to the associated exception mode, the bits representing the processor
mode in CPSR then change to the corresponding exception mode.  Thus
each type of exception proceeds under the specific exception
mode. Eventually, the processor will return to the normal user mode
and the CPSR will be restored from the saved value.

\fussy

\begin{figure}[h]
\centering
$$exn\_mode=fiq~|~irq~|~svc~|~abt~|~und$$
$$proc\_mode=usr~|~exc\_mode~|~sys$$
\caption{ARM processor modes}
\label{fig:armpm}
\end{figure}

In our work, we only consider a simplified memory model without the
memory management unit (MMU) described in part {\bf Memory and system
  architecture}, in which only read and write functions are kept.  In
the ARM model, memory is controlled by the System Control Co-processor
(CP15). So we describe the memory model as a part of System Control
Co-processor (SCC), together with registers of SCC.  The state of SCC
and the state of the main processor are together to form the state of
ARMv6.

%3 addressing modes, 11 operand modes, 15 condition modes, 2 post-operation modes
The next part of {\bf CPU architecture} is for the ARM instructions
set.  This is where we spent most of our efforts.  For ARMv6
architecture, there are 147 kinds of instructions and five kinds of
addressing modes; with the different combinations of 15 condition
modes, 11 operands, and two post-operation modes; the combination of
which represents tens of thousands of instruction instances to
consider.

The reference manual describes the ARM instructions in a
well-structured way by providing their syntax and usage.  Each
instruction is specified by:
\begin{itemize}
\item The instruction encoding table
\item The instruction syntax
\item The validation under different versions of ARM architecture
\item The exceptions that may occur
\item The pseudo-code explaining the instruction operation
\item The usage or notes.
\end{itemize}

The description of instructions plays a very important role in our project.
%We have to understand each part precisely, in order to formalize them accurately.
As an example, let us take the instruction \adc from the
data-processing instruction set.  This instruction performs an add
with carry.  It adds (with carry) the value of a register with either
the value of another register or an immediate value.  Its encoding
table is shown in Figure~\ref{fig:adcet}\,.


\begin{figure}[h]
\centering
\begin{tabular}{|c|c|c|c|c|c|c|c|c|c|}
%\multicolumn{10}{c}{\small\em (a) binary encoding of the {\stt ADC} instruction}\\
\hline
31 $\ldots$ 28 & 27 26 & 25 & 24 \dotfill 21 & 20 & 19 $\ldots$ 16 & 15 $\ldots$ 12 & \multicolumn{3}{c|}{11 \dotfill 0} \\\hline
\stt cond & \stt 0~0 & \stt I & \stt 0~1~0~1 & \stt S & \stt Rn & \stt Rd & \multicolumn{3}{c|}{\stt shifter\_operand} \\
\hline
\end{tabular}
\caption{ADC instruction encoding table}
\label{fig:adcet}
\end{figure}

The bits $[31:28]$ encode the conditions (\armrf{cond}), under which the instruction is going to be executed.
$$\begin{array}{rl}
   \armrf{cond} = & \armrf{EQ~|~NE~|~CS~|~CC~|~MI~|~PL~|~VS~|~VC~|~HI~|~}\\
                  & \armrf{LS~|~GE~|~LT~|~GT~|~LE~|~AL}
 \end{array}
$$

When the condition is not satisfied, the instruction has no effect on
the processor state, acting like a \textit{No-Op} instruction, and the
program counter moves up to the next instruction.  Most ARM
instructions are conditional, only a small number can be executed
unconditionally, although in practice many instructions bear the
\textbf{always} code, which indicates that the instruction is always
executed.  The four bits of the condition code are related to the
condition flags in the Status Register, so that an instruction can be
executed only when the condition code matches.

The bits $[24:21]$ form what is called the \armkw{opcode}, which
specifies the operation. Here it contains the code for ``add with
carry''.  These bits are first checked by the decoder to recognize the
instruction kind.

The \bit{I} bit is an identifier which distinguishes the immediate
shifter operand from the register-based shifter operand, and the
\bit{S} bit indicates whether the instruction updates the flags in
CPSR.

\armkw{Rn} is the first source operand. According to the addressing
mode encoded by bits \bit{I},\bit{7} and \bit{4} (explained below in
this section), the second operand is one of the following basic cases:
\begin{itemize}
\item
An immediate operand, formed by rotating bits $[7:0]$
with a even value decided by the four bits $[11:8]$.
\item
A register operand, which refers to the bits $[3:0]$.
\item
A shifted register operand, which is a shifted or rotated value
  of a register.  The register is of bits $[3:0]$, and the five types
  of shift is indicated by the bits $[11:7]$.
\end{itemize}

However, because the ARM encoding is very dense, the same instruction
with a special combination of the three bits \bit{I}, \bit{7}, and \bit{4},
is no longer a data-processing instruction,
but it becomes an extension of the Load/Store instruction...

In the assembly language description for each instruction,
parameters wrapped by \armrf{<~>}
refer to corresponding bit fields in the encoding table.
For example in Figure~\ref{fig:adcas},
the value of the parameters in the assembler syntax of \adc,
\armkw{cond}, \armkw{S}, \armkw{Rd}, \armkw{Rn},
and \armkw{shifter\_operand} must be encoded precisely by the bit fields
from the \adc encoding table as Figure~\ref{fig:adcet}.
% JF: the next unfinished sentence seems to be here by mistake
% The symbol \armrf{<~>} wraps the value that is

\begin{figure}[h]
\centering
$\armrf{ADC\;\{<cond>\}\{S\}~<Rd>,~<Rn>,~<shifter\_operand>}$
\caption{ADC assembler syntax}
\label{fig:adcas}
\end{figure}

The binary decoder in the simulator must extract the bit fields from
the binary instruction and initialize C variables or data structures
with the appropriate value of the parameters, which sometimes requires
an additional operation like sign extension.

The C like code in Figure~\ref{fig:adcpc} is the piece of pseudo-code
given in the document to describe what the instruction ADC does.  It
first checks if it is conditionally executed. If not, the execution
will finish. If so, it assigns the result of adding the contents in
register \armrf{Rn}, the value of \armrf{shifter\_operand} and the
carry (\armrf{C} flag in CPSR).  Then, considering whether the
\armrf{S} bit is set or not, the CPSR is updated or not.  If the
\armrf{S} bit is set and register \armrf{Rd} is the program counter,
it means that the ARM processor is currently running under exception
mode and the status in SPSR needs to be restored in CPSR.  Before
writing to the CPSR, a check is performed to determine if the current
mode is an exception mode, because only such a mode possesses an
SPSR. Otherwise the instruction returns \unpred.  If the register
\coqdockw{Rn} is some other general-purpose register and the
\coqdockw{S} bit is set, CPSR has to be updated according to the
result of the addition and the presence of a carry.

\begin{figure}[h]
\begin{alltt}
if ConditionPassed(cond) then
   Rd = Rn + shifter_operand + C Flag
   if S == 1 and Rd == R15 then
      if CurrentModeHasSPSR() then
         CPSR = SPSR
      else UNPREDICTABLE
   else if S == 1 then
      N Flag = Rd[31]
      Z Flag = if Rd == 0 then 1 else 0
      C Flag = CarryFrom(Rn + shifter_operand + C Flag)
      V Flag = OverflowFrom(Rn + shifter_operand + C Flag)
\end{alltt}
\caption{ADC instruction operation Pseudo-code}
\label{fig:adcpc}
\end{figure}

\newcommand{\Rd}{\armrf{Rd}\xspace}
\newcommand{\Rn}{\armrf{Rn}\xspace}

% JF: again, meaningless sentence
% When reading the pseudo-code,
% we have to understand the different meaning of symbols.
% JF: new sentence instead
There are some pitfalls related to the different meaning of symbols in
the pseudo-code.  The same notation on different sides of '$=$' can be
different.  For example, let us consider the assignment 
``$\armrf{Rd = Rn + shifter\_operand + C Flag}$''.
On the left-hand side of '$=$',
the meaning of \Rd is the address of \Rd (the result will be assigned
to address of \Rd).  On the other hand, the same \Rd on the right-hand
side, e.g., in ``$\armrf{N Flag = Rd[31]}$'', means the content of
register \Rd.  More subtle, the meaning of \Rd is different from the
meaning of \Rn -- so such names are more like keywords than
identifiers: on the right-hand side of an equation, \Rn represents the
\emph{original} contents of \Rn, whereas \Rd represents the
\emph{current} contents of \Rd.  And when \Rd and \Rn happen to be the
same register, the value of \Rn on the right hand-side must stick to
the original contents, not to the updated result.
% JF: again, remove obvious sentence
% Accurately understand the meaning of pseudo-code, we are able to
% formalize the ARMv6 instructions correctly.

As we explained for instruction \adc, the second operand is encoded by
addressing mode 1 -- data-processing operand.
There are five groups of addressing mode:
\begin{itemize}
\item Addressing mode 1 -- Data-processing operand
\item Addressing mode 2 -- Load and store word or unsigned byte
\item Addressing mode 3 -- Miscellaneous loads and stores
\item Addressing mode 4 -- Load and store multiple
\item Addressing mode 5 -- Load and store coprocessor.
\end{itemize}

Each of them contains several formats used to calculate the value used
in the instruction operation.  For example, in addressing mode 1,
there are eleven formats to encode the \armrf{shift\_operand}, and in
addressing mode 2, there are nine.  The reference manual gives for
each format an encoding table and an assembler syntax, and its
operation in pseudo-code, which are similar to the description of ARM
instructions. With the usage and notes of each instruction, we are
able to match the instruction with its own addressing mode.

\section{Formalization in Coq}

\newcommand{\statedef}{\coqcode{state}}

We want the formal specification to be as close as possible to the
reference manual.  We also want it to be as simple as possible.  In
this way, our formal model can be reliable and simple enough to reason
about.

Our formal model contains the basic library for integer representation
and binary operations, a memory model, the main processor, the system
control co-processor, the instruction set, the simulation loop, and a
description of the initial configuration. % As explained above,
% except the instruction set, all the other descriptions are manually defined.
The bit vector definition and operations reuse the integer
module from \compcert,
instantiated to 32-bit words, 4-bit words for register numbers
and 30-bit integers for memory addresses.

The core part of the ARM processor is the ARM instruction set.  Its
formalization is a rather heavy piece of work.  In particular, the
pseudo-code of each instruction given by the ARM reference manual has
to be formalized.  As a result, we get a formal semantics for ARMv6
instructions.

According to the structure presented in Figure~\ref{fig:armst},
we formalize a \texttt{Record} type \statedef by composing
the \statedef of the main processor
and the \statedef of the system control coprocessor:

\begin{alltt}
Record state : Type := mk_state \{
  (* Current program status register *)
  cpsr : word;
  (* Saved program status registers *)
  spsr : exn_mode -> word;
  (* Registers *)
  reg : register -> word;
  (* Raised exceptions *)
  exns : list exception;
  (* Processor mode *)
  mode : proc_mode
\}.

Record state : Type := mk_state \{
  (* registers *)
  reg : regnum -> word;
  (* memory *)
  mem : address -> word
\}.

Record state : Type := mk_state \{
  (* Processor *)
  proc : Arm6_Proc.state;
  (* System control coprocessor *)
  scc : Arm6_SCC.state
\}.
\end{alltt}

Considering the whole simulation system, we need another state
representing not only the processor but also the execution status.
We introduce a new type named \coqcode{semstate} to distinguish it from
the \statedef for processors.
The specification follows the monadic style \cite{wadjfp09}
to represent calculations on the ARM processor states.

This style takes the sequentiality of transformations on the state
into account.  In the state monad, functions take a state as input and
return a value combined with a new state.  Beyond the \statedef, two
other pieces of information are handled: \coqcode{loc}, which
represents local variables of the operation, and \coqcode{bo}, a
Boolean indicating whether the program counter should be incremented
or not; they are registered in the following record which is used for
defining our monad.

\begin{alltt}
Record semstate := mk\_semstate \{ loc : local ;  bo : bool ; st : state \}.

Inductive result \{A\} : Type :=
    | Ok (\_ : A) (\_ : semstate) | Ko (m : message) | Todo (m : message).

Definition semfun A := semstate -> @result A.
\end{alltt}

Note that in most cases, % every operation in our formal model % \coqdocvar{O\_Coq}
functions will return an \coqdocvar{Ok} value.
The value \coqdocvar{Ko} is used for \unpred states
and is implicitly propagated with our monadic constructors for exceptions.
The value \coqdocvar{Todo} is used in a similar way for unimplemented operations
-- currently, it is still the case for coprocessor instructions.

The simulation fetches the binary code at a given address; decodes it to corresponding
assembly instruction; invokes the operation in library and executes it; and at last
includes the computation of the address of the next instruction.
The ARMv6 behavior semantics is described by functional
rather than relational definitions.
This means our specification is consistent and deterministic.
The two main components of a processor simulator are then:

\begin{itemize}
\item
  The decoder, which decodes a given binary word, retrieves the
  name of an operation and its potential arguments in assembly code.
  In Section \ref{ssc:dec} we will explain how it is generated from
  the reference manual.
\item
  The precise description of transitions is the operation of instruction.
  The definition contains operations on processor registers and memory;
  thereby, the processor state is changed.
  In the ARMv6 reference manual, these algorithms are written in
  a ``pseudo-code'' syntax which calls low-level primitives. For example,
  some code indicates setting a range of bits of a register by a given
  value. And some operations might lead to unspecified
  or forbidden results. In ARM processor, this is called \unpred. %``UNPREDICTABLE``
  When the simulator meets these result,
  it then returns a \coqcode{Ko} or \coqcode{Todo} state with
  a message specific to the situation.
 % \hide{The precise description of transformations performed by an operation
 % on registers and memory.  In the reference manual of ARM, this is
 % defined by an algorithm written in “pseudo-code” which calls
 % low-level primitives for, e.g., setting a range of bits of a register
 % to a given value.  Some situations are forbidden or left
 % unspecified. For ARM processors, this results in a so-called \unpred
 % state. The best choice, for a simulator, is then to
 % stop with a clear indication of what happens.}
\end{itemize}

\subsection{Running an instruction}
\label{ssec:fsmoa}

Each instruction operation (\textit{O}) from the reference manual,
for example in Figure~\ref{fig:adcpc}, gives a semi-formal
description on how a instruction evaluates.
Here we show how to specify it in by
a corresponding Coq function (\textit{O\_coq}).

Taking the instruction \adc as an example,
its formalization in Coq is showed in the following function
\coqcode{ADC\_step}, which operates on the parameter \coqcode{st}
of type \statedef. Other parameters are found by searching
for unspecified variables in the pseudo-code.
Not all variables are declared globally. Variables which are
assigned during the execution are local variables except the
output registers, for example, \Rd.
The body of the function is kept as close as possible to the pseudo-code
by using notations, like $\texttt{<st>}$, as explained below:

\begin{alltt}
\small
(* A4.1.2 ADC *)
Definition ADC_step (S : bool) (cond : opcode) (d : regnum) (n : regnum)
                    (shifter_operand : word) : semfun _ := <s0>
  if_then (ConditionPassed s0 cond)
    ([ <st> set_reg d (add (add (reg_content s0 n) shifter_operand)
                    ((cpsr s0)[Cbit]))
    ; If (andb (zeq S 1) (zeq d 15))
       then (<st> if_CurrentModeHasSPSR (fun em =>
                                          (<st> set_cpsr (spsr st em))))
       else         (if_then (zeq S 1)
          ([ <st> set_cpsr_bit Nbit ((reg_content st d)[n31])
          ; <st> set_cpsr_bit Zbit
                  (if zeq (reg_content st d) 0
                   then repr 1
                   else repr 0)
          ; <st> set_cpsr_bit Cbit
                  (CarryFrom_add3 (reg_content s0 n)
                                  shifter_operand ((cpsr s0)[Cbit]))
          ; <st> set_cpsr_bit Vbit
                  (OverflowFrom_add3 (reg_content s0 n)
                                     shifter_operand ((cpsr s0)[Cbit])) ])) ]).
\end{alltt}

\newcommand{\cpsrdef}{\coqcode{CPSR}}

% JF: redundant sentence
% The concrete example of \adc instruction helps to explain how
% the formal specification of a state transition functions.
For most of the ARMv6 instructions, executions are conditional.  These
conditionally executed instructions must first check if the current
condition (argument \coqcode{cond}) fits the required
condition. Otherwise, the following operations will be skipped, and
then go to the next instruction.  The \coqcode{S} bit argument
indicates the instruction must update the status register \cpsrdef. If
the argument \Rd refers to the program counter (R15), the updating of
\cpsrdef is going to preserve the value of \coqcode{SPSR} when the
current processor mode is one of the exception mode. If \Rd is one of
the other general-purpose register, updating of \cpsrdef is done by
updating the significant flags in \cpsrdef. The values are calculated
by operations on argument \Rn which contains the first operand, and
\coqcode{shift\_operand} which specifies the second operand.

% JF: HERE

\newcommand{\getst}{\texttt{\_get\_st}\xspace}

Here we explain the notation \texttt{<st>}. It is the notation
for function \getst, a monadic function that provides access to
the current state \coqcode{st} at any place of the operation sequences:

\begin{alltt}
  Definition bind \{A B\} (m : semfun A) (f : A -> semfun B) : semfun B :=
    fun lbs0 =>
    match m lbs0 with
      | Ok a lbs1 => f a lbs1
      | Ko m => Ko m
      | Todo m => Todo m
    end.

  Definition bind_s \{A\} fs B (m : semfun unit) (f : A -> semfun B) :
    semfun B :=
    bind m (fun _ lbs1 => f (fs lbs1) lbs1).

  Definition _get_st \{A\} := bind_s st A (Ok tt).

  Notation "’<’ st ’>’ A" := (\_get\_st (fun st => A))
    (at level 200, A at level 100, st ident).
\end{alltt}

In general, every operation function terminates with
\coqcode{Ok} state. However, errors are implicitly propagated with
our monadic constructors: \coqcode{Ko} and \coqcode{Todo}.

The other notations to keep the formalization well structured are the case
statements \coqcode{If}, \coqcode{then}, \coqcode{else}, and \coqcode{if\_then},
also the sequence statement denoted by brackets and semicolons:

\begin{alltt}
  Definition if_then_else \{A\} (c : bool) (f1 f2 : semfun A) : semfun A :=
    if c then f1 else f2.

  Notation "'If' A 'then' B 'else' C" := (if_then_else A B C)
                                                      (at level 200).

  Definition if_then (c : bool) (f : semfun unit) : semfun unit :=
    if_then_else c f (Ok tt).
\end{alltt}

\begin{alltt}
  Definition _set_bo b lbs := ok_semstate tt (loc lbs) b (st lbs).\

  Definition block (l : list (semfun unit)) : semfun unit :=
    let next_bo f1 f2 := next f1 (_get_bo f2) in
    List.fold_left (fun f1 f2 =>
      next_bo f1 (fun b1 =>
      next_bo f2 (fun b2 => _set_bo (andb b1 b2))) ) l (Ok tt).

  Notation "[ a ; .. ; b ]" := (block (a :: .. (b :: nil) ..)).
\end{alltt}

\subsection{Decoder}
% \xm{Need to rewrite. Also in chapter 4}
\label{ssc:dec}
Now we consider the formalization of decoding instructions.
An instruction encoding table, e.g. like in Figure~\ref{fig:adcet},
summarizes all possibilities for this instruction in 32-bits representation.
All the others will be decoded into \unpred or undefined.
Then we can build an ARM instruction decoder for the ARMv6 architecture
using all of the instruction encoding tables.

% The instruction part of ARM reference manual is well structured, in which the
% two elements can be used as the input of the automatic generation chain. The
% first is the operation description we have introduced. The second is the
% instruction encoding table.

The main body of the decoder is a big pattern matching program. Each constructor
is represented by 32 bits, either implicit or explicit.
The Coq code in Figure~\ref{fig:coqdec} shows a thumbnail of the formal decoder,
for the constructor corresponding to \adc.

\begin{figure}[h]
\begin{alltt}
  Definition decode_conditional (w : word) : decoder_result inst :=
    match w28_of_word w with
    ...
      (*4.1.2 - ADC*)
      | word28 0 0 I_ 0 1 0 1 S_ _ _ _ _ _ _ _ _ _ _ _ _ _ _ _ _ _ _ _ _
      =>
        decode_cond_mode decode_addr_mode1 w
          (fun add_mode condition =>
            ADC add_mode S_ condition (regnum_from_bit n12 w)
                                      (regnum_from_bit n16 w))
    ...
    end.
\end{alltt}
\caption{Formalized decoder of conditional executed instructions}
\label{fig:coqdec}
\end{figure}

% JF->XM: you missed the main point, now introduced in a new sentence.
The decoding of ARM instructions is difficult because some bit
configurations are ambiguous at first sight: they could be interpreted
as different kind of operations.  Such ambiguities are solved in the
reference manual, which specifies a precedence order between the
interpretations.  In our formalization, this precedence order is
implemented by the order used for the different bit patterns, in the
global pattern-matching construct.
% To define a functional ARM instruction decoder, the first thing to do is to
% decide the matching order.
% The matching order is important because
In Coq, the pattern matching is considered from
top to bottom: a value belongs to the $i$th constructor if and only if it
could not match any previous pattern;
a pattern covered by previous patterns is considered as redundant
by the Coq type checker.

The 147 instructions are first partitioned into two groups, conditional and
unconditional instructions.
For ARM instructions, the condition field \texttt{cond} $[31:28]$ indicates
the conditional execution of ARM instruction.
% These four bits describe different meaning when they are set or cleared.
% And instruction under such condition has to be executed with the corresponding
% CPSR (Current Program Status Register) in which N, Z, C, V flags statisfy
% the given condition. If not so, the instruction will have no effect on processor
% state and go to the next instruction without notification.
% A small mount of instruction can only be executed unconditionally.
These instructions will be checked first by matching the first four bits with
$0b1111$ representing an unconditional execution.
The others are grouped by ARM instruction categories.
Instructions belonging to the same category do not conflict with each other
by the wild-card mechanism.
We also define 5 levels for grouping conditional instructions.
\begin{itemize}
\item
All multiply instructions without accumulator. They can be covered by
similar multiply instruction with an accumulator. Instructions without
accumulator fill the bits $[15:12]$ with $0b1111$, whereas instructions with
an accumulator using them refer to the register for accumulator operand.
\item
Some instructions from ARMv5 architecture use the notion \texttt{SBO} or
\texttt{SBZ} to express that the instruction bit is read as one/zero whatever the
value of the bit is and it cannot be rewritten.
These instructions need to be checked then, otherwise they could be hidden by
some of the new ARMv6 instructions.
\item
A few load/store instructions work under the privileged mode.  % of processor
Two significant bits $P$ and $W$ are assigned to a special combination of values
to indicate this kind of instructions.
We have to put them in higher priority, before
the similar instructions working for the other processor modes.
\item
Instructions load/store from memory with a format other than \texttt{word}
have a shape similar to the load/store with \texttt{word},
but the four bits $[7:4]$ are used to refer to the load/store length --
indicating
whether it is a \texttt{half}, \texttt{double word}, or a \texttt{signed byte}.
\item
The last group contains all the operations with addressing modes.
For decoding this kind of instructions,
the decoder for addressing mode has to be called first.
The addressing mode decoders are introduced below.
\end{itemize}

In Section~\ref{sec:ref} we mentioned that
the ARM instruction set admits five kinds of addressing mode.
They are used to encode the specific values appearing
in the instruction pseudo-code.
% JF: next sentence redundant with last item of the previous itemize
% The decoding of instructions with addressing mode must call first the
% decoder of addressing mode in order to know value fits to which format.
The encoding tables for addressing mode are in the same form as the ones for
ARM instructions, the formalization of addressing mode decoders are
similar to the instruction decoder.
The following definition shows one clause of the decoder for the addressing
mode 1 -- Data-processing operands, to indicate that the \armrf{shift\_operand}
is calculated by an immediate logical shift left. % JF->XM: Check this line

\begin{alltt}
 Definition decode_addr_mode1 (w : word) : decoder_result mode1:=
  match w28_of_word w with
  ...
  (*5.1.5 - Data processing operands - Logical shift left by immediate*)
  | word28 0 0 0 _ _ _ _ S_ _ _ _ _ _ _ _ _ _ _ _ _ _ 0 0 0 _ _ _ _ =>
    DecInst _ (M1_Logical_shift_left_by_immediate
                                     (regnum_from_bit n0 w) w[n11#n7])
   ...
  end.
\end{alltt}

\section{Experimentations}
\label{ssc:vali}

Altogether, we get an executable formal model of ARMv6 architecture,
which can be translated to OCaml code by extraction of Coq code.
However, for the executable version of the formal simulator,
we could not integrate this extracted OCaml code
because the extraction mechanism translates a Coq pattern,
which matches more than one terms, into many OCaml patterns,
which mention all possibilities one by one.

More precisely,
from the Coq model of ARMv6,
it is possible to extract OCaml code and compile it to an executable simulator
and perform some tests.
% And it is also interesting to see tests running on a formal specified
% simulator.
% We provide the tests for the C specified Simlight.
% In the mean time, we use these tests for our formal specification.
The {\stt arm-elf-gcc} compiler is already used in our group to
compile C tests into ELF files to be used in \simlight.
These tests could be translated
to a Coq representation, then extracted to OCaml.
% % JF -> XM I think that you speak about simlight C code, not OCaml.
% These small tests can be executed on the formal simulator in a sufficient
% simulation speed, less than 1000 instructions per second.
Running a simple direct sum test takes around five minutes.
A sorting program would then need one day to be completed.
% % JF -> XM : should be much less, no? anyway no need to state anything.
% wheras the correponding C code only needs 0.78 seconds.


% % JF -> XM : read carefully
Directly simulating the ARM in Coq would even be worse.
However, execution speed is not a concern in formal proofs,
as far as no heavy computations steps are involved.
% For such formal specification, we do not care the execution speed,
% which aims to perform formal proofs for the simulation correctness.
%
% JF : right but commented out to avoid confusion at this stage
% Similarly, the Coq decoder
% can be used as a suitable formal model for correctness proofs in future work.


%\xm{Need to be moved to chapter 5}
% A more efficient way is to generate independently a decoder in OCaml code using
% the similar generation stradegy as the Coq decoder generator.

%\jf{So what? What is the actual situation?}

% A same affect is also done for C representation of \simlight.
% For C representation decoder, we do not use such sequential algorithm.
% It lowers too much of the execution speed. Instead, a decoder based on
% {\it switch} is in used. It is also generated automatically
% using the same generating stradegy, and be put in
% the same file as instructions set descriptions in C language
% with cases in {\it switch} statement.

%%% Local Variables:
%%% mode: latex
%%% TeX-master: "thesis"
%%% End:
